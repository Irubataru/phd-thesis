% --------------------- %
% Consistency functions %
% --------------------- %

%%Referencing

\newcommand{\chaptoken}{\textsc{chapter}}
\newcommand{\sectoken}{\textsc{section}}
\newcommand{\secstoken}{\textsc{sections}}
\newcommand{\apxtoken}{\textsc{appendix}}
\newcommand{\apxstoken}{\textsc{appendices}}
\newcommand{\eqtoken}{\textsc{eq}}
\newcommand{\eqstoken}{\textsc{eqs}}
\newcommand{\figtoken}{\textsc{figure}}
\newcommand{\tabletoken}{\textsc{table}}
\newcommand{\ruletoken}{\textsc{rule}}

%%Naming

\newcommand{\qstat}{Q_{\text{stat}}}
\newcommand{\qkin}{Q_{\text{kin}}}

% ----------------- %
% Pretty references %
% ----------------- %

\newcommand{\chapref}[1]{\hyperref[#1]{\mbox{\chaptoken{} \ref*{#1}}}}
\newcommand{\secref}[1]{\hyperref[#1]{\mbox{\sectoken{} \ref*{#1}}}}
\newcommand{\meqref}[1]{\hyperref[#1]{\mbox{\eqtoken{} (\ref*{#1})}}}
\newcommand{\apxref}[1]{\hyperref[#1]{\mbox{\apxtoken{} \ref*{#1}}}}
\newcommand{\figref}[1]{\hyperref[#1]{\mbox{\figtoken{} \ref*{#1}}}}
\newcommand{\tabref}[1]{\hyperref[#1]{\mbox{\tabletoken{} \ref*{#1}}}}
\newcommand{\ruleref}[1]{\hyperref[#1]{\mbox{\ruletoken{} \ref*{#1}}}}


% ----------------------------------------------------------------- %
% Horribly complicated command to give a list of section references %
% ----------------------------------------------------------------- %

\newcommand*{\refprevious}{}
\newcommand*{\refseparator}{}

% Prints the previously stored reference, and stores the current
\newcommand{\printoneref}[2][, ]{%
  \refprevious%
  \refseparator%
  \renewcommand*{\refprevious}{%
    \ref{#2}}% store the current element in the \refprevious variable
  \renewcommand*{\refseparator}{#1}% update the separator after first element
}

\newcommand{\secrefs}[1]{%
  \renewcommand*{\refprevious}{}%
  \renewcommand*{\refseparator}{}%
  \mbox{\secstoken{}}
  \forcsvlist{\printoneref[, ]}{#1}%
  and \refprevious{}%
}

\newcommand{\apxrefs}[1]{%
  \renewcommand*{\refprevious}{}%
  \renewcommand*{\refseparator}{}%
  \mbox{\apxstoken{}}
  \forcsvlist{\printoneref[, ]}{#1}%
  and \refprevious{}%
}

\newcommand{\meqrefs}[2][,]{%
  \renewcommand*{\refprevious}{}%
  \renewcommand*{\refseparator}{}%
  \mbox{\eqstoken{} (}%
  \forcsvlist{\printoneref[#1]}{#2}%
  \refprevious{})%
}


% -------------- %
% tikz functions %
% -------------- %

\tikzset{
    skip loop/.style={
      to path={-- ++(0,#1) -| (\tikztotarget)}
    },
    %position/.style args={#1:#2 from #3}{
        %at=(#3.#1), anchor=#1+180, shift=(#1:#2)
    %}
}

% ---------------------- %
% Customized mathematics %
% ---------------------- %

\newcommand\scalemath[2]{\scalebox{#1}{\mbox{\ensuremath{\displaystyle #2}}}}

\newcommand{\minus}{\raisebox{.16ex}{\scalebox{0.75}{$-$}}}
\newcommand{\mminus}{\raisebox{.25ex}{\scalebox{0.5}{$-$}}}

\newcommand\moment[1][\,]{{\bm{\langle}} #1 {\bm{\rangle}}}
%\newcommand\cumulant[1][\,]{{\bm{[}} #1 {\bm{]}}}
\newcommand\cumulant[1][\;]{\mathbold{[} #1 \mathbold{]}}

% To get "eqnarray" centred alignment
% See: http://tex.stackexchange.com/a/102845/39313
\newcommand*\centermathcell[1]{\omit\hfil$\displaystyle#1$\hfil\ignorespaces}

% Get a raised \chi
\DeclareRobustCommand{\rchi}{{\mathpalette\irchi\relax}}
\newcommand{\irchi}[2]{\raisebox{\depth}{$#1\chi$}}

%\theoremstyle{defintion}
\newtheorem{definition}{Definition}[chapter]

%\theoremstyle{ruledef}
\newtheorem{ruledef}{Rule}[chapter]

% New definition of square root:
% it renames \sqrt as \oldsqrt
\let\oldsqrt\sqrt
% it defines the new \sqrt in terms of the old one
\def\sqrt{\mathpalette\DHLhksqrt}
\def\DHLhksqrt#1#2{%
  \setbox0=\hbox{$#1\oldsqrt{#2\,}$}\dimen0=\ht0
  \advance\dimen0-0.2\ht0
  \setbox2=\hbox{\vrule height\ht0 depth -\dimen0}%
  {\box0\lower0.4pt\box2}}

% xoverline function that has the width of the overline as an additional
% argument, from http://tex.stackexchange.com/a/22101/39313
\makeatletter
\newsavebox\@xoverlineboxA
\newsavebox\@xoverlineboxB
\newlength\@xoverlinelen

\newcommand*\xoverline[2][0.75]{%
    \sbox{\@xoverlineboxA}{$\m@th#2$}%
    \setbox\@xoverlineboxB\null% Phantom box
    \ht\@xoverlineboxB=\ht\@xoverlineboxA%
    \dp\@xoverlineboxB=\dp\@xoverlineboxA%
    \wd\@xoverlineboxB=#1\wd\@xoverlineboxA% Scale phantom
    \sbox\@xoverlineboxB{$\m@th\overline{\copy\@xoverlineboxB}$}%  Overlined phantom
    \setlength\@xoverlinelen{\the\wd\@xoverlineboxA}%   calc width diff
    \addtolength\@xoverlinelen{-\the\wd\@xoverlineboxB}%
    \ifdim\wd\@xoverlineboxB<\wd\@xoverlineboxA%
       \rlap{\hskip 0.5\@xoverlinelen\usebox\@xoverlineboxB}{\usebox\@xoverlineboxA}%
    \else
        \hskip -0.5\@xoverlinelen\rlap{\usebox\@xoverlineboxA}{\hskip 0.5\@xoverlinelen\usebox\@xoverlineboxB}%
    \fi}
\makeatother

% ----------------- %
% mathlap functions %
% ----------------- %

\def\clap#1{\hbox to 0pt{\hss#1\hss}}
\def\mathllap{\mathpalette\mathllapinternal}
\def\mathrlap{\mathpalette\mathrlapinternal}
\def\mathclap{\mathpalette\mathclapinternal}

\def\mathllapinternal#1#2{%
	\llap{$\mathsurround=0pt#1{#2}$}}

\def\mathrlapinternal#1#2{%
	\rlap{$\mathsurround=0pt#1{#2}$}}

\def\mathclapinternal#1#2{%
	\clap{$\mathsurround=0pt#1{#2}$}}
