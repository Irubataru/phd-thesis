% --------------------- %
% Consistency functions %
% --------------------- %

\newcommand{\sectoken}{\textsc{section}}
\newcommand{\secstoken}{\textsc{sections}}
\newcommand{\apxtoken}{\textsc{appendix}}
\newcommand{\apxstoken}{\textsc{appendices}}
\newcommand{\eqtoken}{\textsc{eq}}
\newcommand{\eqstoken}{\textsc{eqs}}
\newcommand{\figtoken}{\textsc{figure}}
\newcommand{\tabletoken}{\textsc{table}}

% ----------------- %
% Pretty references %
% ----------------- %

\newcommand{\secref}[1]{\hyperref[#1]{\mbox{\sectoken{} \ref*{#1}}}}
\newcommand{\meqref}[1]{\hyperref[#1]{\mbox{\eqtoken{} (\ref*{#1})}}}
\newcommand{\apxref}[1]{\hyperref[#1]{\mbox{\apxoken{} \ref*{#1}}}}
\newcommand{\figref}[1]{\hyperref[#1]{\mbox{\figtoken{} \ref*{#1}}}}
\newcommand{\tabref}[1]{\hyperref[#1]{\mbox{\tabletoken{} \ref*{#1}}}}

% ----------------------------------------------------------------- %
% Horribly complicated command to give a list of section references %
% ----------------------------------------------------------------- %

\newcommand*{\secprevious}{}
\newcommand*{\secseparator}{}

% Prints the previously stored reference, and stores the current
\newcommand{\printoneref}[1]{%
  \secprevious%
  \secseparator%
  \renewcommand*{\secprevious}{%
    \ref{#1}}% store the current element in the \secprevious variable
  \renewcommand*{\secseparator}{, }% update the separator after first element
}

\newcommand{\secrefs}[1]{%
  \renewcommand*{\secprevious}{}%
  \renewcommand*{\secseparator}{}%
  \mbox{\secstoken{}}
  \forcsvlist{\printoneref}{#1}%
  and \secprevious{}%
}

\newcommand{\apxrefs}[1]{%
  \renewcommand*{\secprevious}{}%
  \renewcommand*{\secseparator}{}%
  \mbox{\apxstoken{}}
  \forcsvlist{\printoneref}{#1}%
  and \secprevious{}%
}

% ---------------------- %
% Customized mathematics %
% ---------------------- %

\newcommand\scalemath[2]{\scalebox{#1}{\mbox{\ensuremath{\displaystyle #2}}}}
\newcommand{\minus}{\raisebox{.16ex}{\scalebox{0.75}{$-$}}}

% To get "eqnarray" centred alignment
% See: http://tex.stackexchange.com/a/102845/39313
\newcommand*\centermathcell[1]{\omit\hfil$\displaystyle#1$\hfil\ignorespaces}

% Get a raised \chi
\DeclareRobustCommand{\rchi}{{\mathpalette\irchi\relax}}
\newcommand{\irchi}[2]{\raisebox{\depth}{$#1\chi$}}

% New definition of square root:
% it renames \sqrt as \oldsqrt
\let\oldsqrt\sqrt
% it defines the new \sqrt in terms of the old one
\def\sqrt{\mathpalette\DHLhksqrt}
\def\DHLhksqrt#1#2{%
  \setbox0=\hbox{$#1\oldsqrt{#2\,}$}\dimen0=\ht0
  \advance\dimen0-0.2\ht0
  \setbox2=\hbox{\vrule height\ht0 depth -\dimen0}%
  {\box0\lower0.4pt\box2}}

% ----------------- %
% mathlap functions %
% ----------------- %

\def\clap#1{\hbox to 0pt{\hss#1\hss}}
\def\mathllap{\mathpalette\mathllapinternal}
\def\mathrlap{\mathpalette\mathrlapinternal}
\def\mathclap{\mathpalette\mathclapinternal}

\def\mathllapinternal#1#2{%
	\llap{$\mathsurround=0pt#1{#2}$}}

\def\mathrlapinternal#1#2{%
	\rlap{$\mathsurround=0pt#1{#2}$}}

\def\mathclapinternal#1#2{%
	\clap{$\mathsurround=0pt#1{#2}$}}
