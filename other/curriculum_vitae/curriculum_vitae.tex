%%%%%%%%%%%%%%%%%%%%%%%%%%%%%%%%%%%%%%%%%
% "ModernCV" CV and Cover Letter
% LaTeX Template
% Version 1.2 (25/3/16)
%
% This template has been downloaded from:
% http://www.LaTeXTemplates.com
%
% Original author:
% Xavier Danaux (xdanaux@gmail.com) with modifications by:
% Vel (vel@latextemplates.com)
%
% License:
% CC BY-NC-SA 3.0 (http://creativecommons.org/licenses/by-nc-sa/3.0/)
%
% Important note:
% This template requires the moderncv.cls and .sty files to be in the same 
% directory as this .tex file. These files provide the resume style and themes 
% used for structuring the document.
%
%%%%%%%%%%%%%%%%%%%%%%%%%%%%%%%%%%%%%%%%%

%----------------------------------------------------------------------------------------
%	PACKAGES AND OTHER DOCUMENT CONFIGURATIONS
%----------------------------------------------------------------------------------------

\documentclass[11pt,a4paper,sans]{moderncv}

\moderncvstyle{casual}
\moderncvcolor{red}

\usepackage[
  scale=0.75,
  bmargin=20mm,
  tmargin=30mm,
]{geometry}

%----------------------------------------------------------------------------------------
%	NAME AND CONTACT INFORMATION SECTION
%----------------------------------------------------------------------------------------

\firstname{Jonas Rylund}
\familyname{Glesaaen}

\title{Curriculum Vitae}
\address{Haakon Tveters Vei 26 d}{0682, Oslo, Norway}
\mobile{+47 95 29 99 82}
\phone{+49 171 1814 329}
\email{jonas@glesaaen.com}
%\homepage{github.com/irubataru}{github.com/irubataru}

%----------------------------------------------------------------------------------------

\begin{document}

\makecvtitle % Print the CV title

%----------------------------------------------------------------------------------------
%	EDUCATION SECTION
%----------------------------------------------------------------------------------------

\section{Education}

\cventry{\scalebox{0.9}{08.2013-present}}{Ph.D.-Student}%
  {Johann Wolfgang Goethe-Universit\"et}{Frankfurt am Main}{}{%
    Phase diagram of QCD, lattice QCD, analytic methods, cold and dense, heavy
    QCD, \\hopping parameter expansion, strong coupling expansion, graph theory.
  }
\cventry{\scalebox{0.9}{08.2009-05.2015}}{Master student}%
  {Norwegian University of Science and Technology (NTNU)}{\hskip 1cmTrondheim}{Graduate of excellence}{%
    Phase diagram of QCD, effective models, quark extended linear sigma model,\\
    the renormalisation group.}

\section{Theses}

\cventry{\scalebox{0.9}{Ph.D.-Thesis}}{%
    Heavy Quark QCD at Finite Temperature and Density Using an Effective Theory%
  }{%
    Supervisors: Prof. Dr. Owe Philipsen, and Prof. Dr.  Dirk-Hermann Rischke
  }{}{}{%
    In this work I used an effective theory approach to study the cold and dense
    limit of heavy QCD from first principles. I developed both computational and
    graphical tools and methods for carrying out the necessary mathematics to
    high order. Much of the work is also dedicated to the methods of resummation
    and their advantages.
  }

\cventry{\scalebox{0.9}{Master Thesis}}{%
    The Chiral Phase Transition in QCD: Mean-Field Versus the Functional Renormalisation Group%
  }{%
    Supervisor: Prof. Jens Oluf Andersen
  }{}{}{%
    For my masters work I made use of low energy effective theories, more
    specifically the quark extended linear sigma model, to analyse the
    properties of the breaking of chiral symmetry in QCD, and the emerging phase
    diagram. Also made use of the functional renormalisation group to extract
    the correct low energy properties of the quantum field theory.
  }

%----------------------------------------------------------------------------------------
%	WORK EXPERIENCE SECTION
%----------------------------------------------------------------------------------------

\section{Experience}

\subsection{Teaching, Goethe Universit\"et}

\cventry{10.2015}{Introductory Course to the C++ Programming Language}{}{}{}{%
  Taught a full week intensive course on the C++ programming language, which is a
  prerequisite for the students who wish to attend the numerical physics course
  taught by Prof. Lindenstruth}

\cventry{SS15}{Quantum Field Theory II}{Jun Prof. Dr. Marc Wagner}{}{}{}
\cventry{WS14}{Statistical Physics}{Prof. Dr. Owe Philipsen}{}{}{}
\cventry{SS14}{Introductory Quantum Mechanics}{Prof. Dr. Owe Philipsen}{}{}{}
\cventry{WS13}{Programming For Physicists}{Jun Prof. Dr. Marc Wagner}{}{}{}


\subsection{Teaching, NTNU}

\cventry{autumn 2012}{Statistical Physics}{Prof. Kåre Olaussen}{}{}{}
\cventry{spring 2012}{Computational Physics}{Prof. Alex Hansen}{}{}{%
  Created and published solutions to the homework and lectured Once a week.}
\cventry{autumn 2011, autumn 2012}{Quantum Mechanics II}{Prof. Jan Myrheim}{}{}{}

%\newpage
\cventry{autumn 2010}{Wave Physics}{Prof. Jon Andreas Støvneng}{}{}{}
\cventry{spring 2010}{Vector Calculus}{Førsteamenuensis Heidi Dahl}{}{}{}
\cventry{spring 2010}{Electricity and Magnetism}{Prof. Jon Andreas Støvneng}{}{}{}
\cventry{autumn 2009}{Mechanical Physics}{Prof. Arne Mikkelsen}{}{}{}

\subsection{Other}

\cventry{summer 2011, summer 2012, winter 2012}{Service desk administrator}%
  {Petroleum Geo-Services}{Oslo, Norway}{}{%
    Management and administration of user databases, PC repair, general support
    tasks.%
  }

\cventry{summer 2010}{Internship, material science}%
  {National Institute of Material Science (NIMS)}{Tsukuba, Japan}{Supervision of Prof. Kenji Sakurai}{%
    Carried our x-ray diffraction laboratory work on thin film materials
    sputtered with ytterbium crystals. Held and took courses on various material
    science related subjects. Observed synchrotron experiments at the KEK Photon
    Factory synchrotron.
  }

%----------------------------------------------------------------------------------------
%	COMPUTER SKILLS SECTION
%----------------------------------------------------------------------------------------

\section{Skills}

\cvitem{C++}{Experienced in OO programming, Template Meta Programming, MPI, the C++14 standard,
  boost, GSL, and other scientific libraries}
\cvitem{Other}{Mathematica, JavaScript, Python, LISP}
\cvitem{Typography}{\LaTeX{}, HTML, CSS}
\cvitem{Tools}{Vim, Gnuplot, GDB}

%----------------------------------------------------------------------------------------
%	LANGUAGES SECTION
%----------------------------------------------------------------------------------------

\section{Languages}

\cvitemwithcomment{Norwegian}{Mother tongue}{}
\cvitemwithcomment{English}{Excellent}{IELTS 8.5}
\cvitemwithcomment{Japanese}{Basic}{Basic communication, reading and writing skills}

\section{Publications}

\cvitem{}{%
  J. Glesaaen, M. Neuman, O. Philipsen, Heavy dense QCD from a 3D effective
  lattice theory. In {\itshape Proceedings, 33rd International Symposium on
    Lattice Field Theory (Lattice 2015)}, 2015, arXiv: 1511.00967
  }

\cvitem{}{%
  J. Glesaaen, M. Neuman, O. Philipsen, Equation of state for cold and dense
  heavy QCD. {\itshape JHEP}, 03:100, 2016, doi: 10.1007/JHEP03(2016)100
  }

\end{document}
