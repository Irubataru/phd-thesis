\chapter{Analytic Evaluation of the Effective Theory} \label{chap5}

In \chapref{chap4} we introduced the dimensionally reduced effective theory for
heavy quarks at strong coupling. We ended the chapter with a section on the
numerical handling of the theory and its advantages over full lattice gauge
theory simulations. Although a lot of progress has been made evaluating the
predictions of the theory \citep{Fromm:2011qi,Fromm:2012eb,Langelage:2014vpa},
we see from the convergence plots in \figref{fig:numerical_convergence} that
convergence is slow and other approaches should be considered.

It was observed in one of the previous studies of the effective theory
\citep{Langelage:2014vpa} that it is possible to treat the effective theory
purely analytical which provides a plethora of useful methods which we will
explore in this chapter. First and foremost it lends insight into the
mathematical and physical structure of the effective theory and serves as a
cross check for the numerical methods. In \secref{sec:cluster_expansion} we will
present the linked cluster expansion which will provide the building blocks for
a systematic study of the analytic evaluation. We will see how one can translate
between the language of spin statistics and nearest neighbour systems and the
strong coupling, heavy quark formalism. In \secref{sec:analytic_resummation} we
will introduce a new resummation scheme to the analytic evaluation which is
inaccessible to numerical methods. To do this we will exploit the care we put
into the section on effective theory combinatorics.

In \secref{sec:evaluation} we will utilise the full power of the analytic
expressions to study the various aspects of the theory at hand, comparing with
numerical results, studying lattice artefacts and more.

Finally in \secrefs{sec:large_nc_study,sec:yang_lee_zeros} we carry out two
exploratory studies in which the analytic evaluation is paramount. Although
these still pose a lot of open questions, we will build foundations on which
future studies can be performed.

\section{Linked cluster expansion} \label{sec:cluster_expansion}
\section{Analytic resummation} \label{sec:analytic_resummation}
\section{Observables} \label{sec:evaluation}
\section{Large \texorpdfstring{$N_c$}{Nc} limit} \label{sec:large_nc_study}
\section{Yang Lee Zeros} \label{sec:yang_lee_zeros}
