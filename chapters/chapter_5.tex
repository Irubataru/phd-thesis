\chapter{Analytic Evaluation of the Effective Theory} \label{chap5}

In \chapref{chap4} we introduced the dimensionally reduced effective theory for
heavy quarks at strong coupling. We ended the chapter with a section on the
numerical handling of the theory and its advantages over full lattice gauge
theory simulations. Although a lot of progress has been made evaluating the
predictions of the theory \citep{Fromm:2011qi,Fromm:2012eb,Langelage:2014vpa},
we see from the convergence plots in \figref{fig:numerical_convergence} that
convergence is slow and other approaches should be considered.

It was observed in one of the previous studies of the effective theory
\citep{Langelage:2014vpa} that it is possible to treat the effective theory
purely analytical which provides a plethora of useful methods which we will
explore in this chapter. First and foremost it lends insight into the
mathematical and physical structure of the effective theory and serves as a
cross check for the numerical methods. In \secref{sec:cluster_expansion} we will
present the linked cluster expansion which will provide the building blocks for
a systematic study of the analytic evaluation. We will see how one can translate
between the language of spin statistics and nearest neighbour systems and the
strong coupling, heavy quark formalism. In \secref{sec:analytic_resummation} we
will introduce a new resummation scheme to the analytic evaluation which is
inaccessible to numerical methods. To do this we will exploit the care we put
into the section on effective theory combinatorics.

In \secref{sec:evaluation} we will utilise the full power of the analytic
expressions to study the various aspects of the theory at hand, comparing with
numerical results, studying lattice artefacts and more.

Finally in \secrefs{sec:large_nc_study,sec:yang_lee_zeros} we carry out two
exploratory studies in which the analytic evaluation is paramount. Although
these still pose a lot of open questions, we will build foundations on which
future studies can be performed.

\section{Linked cluster expansion} \label{sec:cluster_expansion}

We start once more on a more fundamental level by introducing the linked cluster
formalism for scalar fields with nearest neighbour interactions. Although the
following section to a sense should be complete, we refer to introductory texts
on the subject for more details, e.g. \citep{Wortis:1980zb,Reisz:1995ag}, and
\citep{Domb:1980zb,Martin:1980zb} for a physics focused presentation of simple
graph theoretical practices. After the fundamentals are out of the way we study
how one can expand the formalism to also include $n$-point interactions with
finite spatial extent, a formalism in which the effective theory can then be
expressed.

\subsection{Classical linked cluster expansion for nearest neighbour interactions}
\label{sec:classical_lce_nn}

To introduce the framework we consider a scalar field with a $2$-point coupling
%
\begin{equation} \label{eq:scalar_field_Z}
  \mathcal{Z} = \int [\mathrm{d} \phi] e^{-S_0[\phi] + \frac{1}{2} \sum_{x,y}
    \phi(x) v(x,y) \phi(y)},
\end{equation}
%
where function $v(x,y)$ encodes the coupling information.We will assume that the
coupling strength is small enough to justify an expansion around the free
theory. To facilitate the expansion we introduce source fields $J(x)$ and define
the generating functional
%
\begin{equation}
  \mathcal{Z}[J] = \int [\mathrm{d} \phi] e^{-S[\phi] + \sum_x J(x) \phi(x)}.
\end{equation}
%
Since our goal is to study thermodynamic quantities, we shift our attention to
the computation of the grand canonical potential $\mathcal{W}$, or the generating functional
of connected correlation functions
%
\begin{equation}
  \mathcal{W}[J,v] = \log \mathcal{Z}[J,v].
\end{equation}
%
A linked cluster expansion (\emph{LCE}) of the grand canonical potential is then defined as
the Taylor expansion with respect to the coupling $v(x,y)$ around the free
theory
%
\begin{equation} \label{eq:cluster_expansion_def}
  \mathcal{W}[J,v] = \bigg( \exp \bigg(\frac{1}{2} \sum_{x,y} v(x,y)
    \frac{\delta}{\delta \hat{v}(x,y)} \bigg) \bigg) \mathcal{W}[J,\hat{v}]
    \,\Bigg|_{\hat{v}=0}.
\end{equation}
%
One can rewrite the derivative of $\mathcal{W}$ with respect to the couplings in
terms of derivatives with respect to the sources
%
\begin{equation}
  \frac{\delta \mathcal{W}}{\delta v(x,y)} = \frac{\delta^2 \mathcal{W}}{\delta
    J(x) \delta J(y)} + \frac{\delta \mathcal{W}}{\delta J(x)} \frac{\delta \mathcal{W}}{\delta J(y)}.
\end{equation}
%
We also know that $\mathcal{W}[J]$ is the generating functional of the
connected $n$-point functions 
%
\begin{equation}
  \frac{\delta \mathcal{W}}{\delta J(x)} \bigg|_{J=\mathrlap{0}} 
    = \frac{1}{\mathcal{Z}} \int [\mathrm{d} \phi] \, \phi(x) \, e^{-S[\phi]}
    \equiv \langle \phi(x) \rangle,
\end{equation}
%
which for higher order derivatives produces the cumulants
%
\begin{equation}
  \frac{\delta^2 \mathcal{W}}{\delta J(x) \delta J(y)} \bigg|_{J=\mathrlap{0}} 
    = \langle \phi(x) \phi(y) \rangle - \langle \phi(x) \rangle \langle \phi(y) \rangle.
\end{equation}
%
To second order the expansion in \meqref{eq:cluster_expansion_def} is
%
\begin{multline} \label{eq:linked_cluster_2nd_order}
  \mathcal{W}[J, v] = \mathcal{W}[J,0]
    + \frac{1}{2} \sum_{x,y} v(x,y) \frac{\delta \mathcal{W}[J,\hat{v}]}{\delta \hat{v}(x,y)} \bigg|_{\hat{v}=0} \\
    + \frac{1}{8} \sum_{x,y} \sum_{z,w} v(x,y) v(z,w) \frac{\delta^2
      \mathcal{W}[J,\hat{v}]}{\delta \hat{v}(x,y) \delta \hat{v}(z,w)}
    \bigg|_{\hat{v}=0} + \dots
\end{multline}
%
We define the coupled $n$-point functions by
%
\begin{equation}
  \mathcal{M}_n(x_1, x_2, \dots, x_n) = \frac{\delta^n \mathcal{W}[J,v]}{\delta
    J(x_1) \delta J(x_2) \cdots \delta J(x_n)}
\end{equation}
%
which in turn define the free theory $n$-point functions
%
\begin{equation}
   \mathcal{M}_n(x_1, x_2, \dots, x_n) \big|_{v = 0}
   = M_n(x_1) \delta(x_1, x_2, \dots, x_n)
\end{equation}
%
where the Kronecker deltas naturally arise in the free theory
%
\begin{equation}
  x \neq y \hskip .2cm\Rightarrow\hskip .2cm
    \langle \phi(x) \phi(y) \rangle \big|_{v=0} = \langle \phi(x) \rangle \langle \phi(y)
    \rangle.
\end{equation}
%
We can easily see from the deltas that the cluster expansion constitutes an
expansion in connected graphs as everything disconnected would give vanishing
contributions. We can rewrite the second order derivative in $v$ in 
\meqref{eq:linked_cluster_2nd_order} in terms of derivatives w.r.t. the sources,
and thus the free $n$-point functions, which gives
%
\begin{multline} \label{eq:free_energy_before_graph}
  \mathcal{W}[v] = \mathcal{W}[0] + \frac{1}{2} \sum_{x,y} M_1(x) \,v(x,y)\, M_1(y)
  + \frac{1}{4} \sum_{x,y} M_2(x) \,v^2(x,y)\, M_2(y)\\+ \frac{1}{2} \sum_{x,y,z}
  M_1(x) \,v(x,y)\, M_2(y) \,v(y,z)\, M_1(z) + \dots
\end{multline}

\subsection{Graphical definitions}

Although the coefficients for the $v^n$ term can be computed systematically from
\meqref{eq:cluster_expansion_def}  as we showed to second order in
\meqref{eq:linked_cluster_2nd_order}, the process is tedious. However there
exists a formalism in which the terms and their prefactors can be written down
immediately in an intuitive way.
%
%\theoremstyle{definition}%
\begin{definition}{Connected graph}\label{def:graph}\\
  A graph is a set of vertices and bonds where every bond connects two distinct
  vertices. An $n$-rooted graph has $n$ fixed, distinguishable, external
  vertices, while all non-rooted vertices are free. A vertex is said to be
  \emph{$n$-valent} if it has $n$ bonds attached to it.

  A connected graph has the property that one can always move from one vertex to
  another through a continuous set of movements along the graph's bonds. A graph
  which is not connected is disconnected.

  Two $n$-rooted graphs are \emph{isomorphic} if there exists a labelling of the
  bonds and vertices so that the bonds and vertices of the two graphs can be
  made identical. The number of distinct isomorphic labellings of a graph is
  called the graph's \emph{symmetry factor}.
\end{definition}
%
\noindent{}%
To compute $\mathcal{W}$ one simply takes the set of all topologically distinct
$0$-rooted connected graph. The order counting is on the bond level, meaning
that at $\mathcal{O}(v^3)$ we only take $0$-rooted connected graphs with three
or fewer bonds. The final ingredient is a rule for translating between the
graphical representation and the mathematical expression for the grand canonical
potential
%
%\theoremstyle{ruledef}%
\begin{ruledef}{Grand canonical potential $\mathcal{W}$} \label{rule:free_energy}
  \begin{enumerate}
    \item Assign a symbol, $x_1, x_2, \dots, x_n$ to every vertex
    \item To every bond connecting vertices $x_i$ and $x_j$ add a factor
      $v(x_i,x_j)$
    \item For every vertex $x_i$ with valence $p$, add a factor $M_p(x_i)$
    \item Add a sum over the entire lattice for every vertex symbol $x_i$
    \item Divide by the symmetry factor of the graph
  \end{enumerate}
\end{ruledef}
%
\noindent{}%
Using this rule we can write the grand canonical potential in
\meqref{eq:free_energy_before_graph} using a set of graphs
%
\begin{equation}
  \mathcal{W}[v] \:=\:
  \tikz[graph style] \node[skeleton node] {};
  \:+\: \textstyle\frac{1}{2}  \:
  \begin{tikzpicture}[graph style]
    \node[skeleton node] (n1) {};
    \node[skeleton node] (n2) [below=of n1] {};
    \draw[skeleton bond] (n1) -- (n2);
  \end{tikzpicture}
  \:+\: \textstyle\frac{1}{2}  \:
  \begin{tikzpicture}[graph style]
    \node[skeleton node] (n1) {};
    \node[skeleton node] (n2) [position=60 degrees from n1]  {};
    \node[skeleton node] (n3) [position=300 degrees from n2] {};
    \draw[skeleton bond] (n1) -- (n2);
    \draw[skeleton bond] (n2) -- (n3);
  \end{tikzpicture} 
  \:+ \textstyle\frac{1}{4} \:
  \begin{tikzpicture}[graph style]
    \node[skeleton node] (n1) {};
    \node[skeleton node] (n2) [below=of n1] {}
      edge[skeleton bond, bend left=45]  (n1)
      edge[skeleton bond, bend right=45] (n1);
  \end{tikzpicture}
  + \dots \,.
\end{equation}
%
The equality between \ruleref{rule:free_energy} and the grand canonical
potential LCE \meqref{eq:cluster_expansion_def} is in no way trivial and proofs
for the equality are given in e.g.  \citep{Englert:1963pr,Bloch:1965jp}. The
topology of the interaction is still yet to be specified and the sums over the
symbols \{$x_1$,\dots,$x_n$\} run over the entire lattice. Further
simplification can be achieved by choosing e.g. a uniform nearest neighbour
coupling
%
\begin{equation}
  v(x,y) = 
    \begin{cases}
      \hskip .1cm v & \text{ if $x$ and $y$ are nearest neighbours}\\
      \hskip .1cm 0 & \text{ else}.
    \end{cases}
\end{equation}
%
We have chosen nearest neighbour interactions here, although it has been shown
that a graphical expansion non nearest neighbour couplings can be reordered in
such a way that the class of graphs are identical to the nearest neighbour case
\citep{Pordt:1996it}. For nearest neighbour interactions the grand canonical
potential simplifies further
%
\begin{equation} \label{eq:free_energy_with_embedding}
  \mathcal{W}[v] = N M_0 + \frac{q}{2} v N M_1^2 + \frac{q}{4} v^2 N M_2^2
    + \frac{q^2}{2} v^2 N M_1^2 M_2 + \dots
\end{equation}
%
where $q$ is the $2 d$ for a $d$-dimensional square lattice. This factor is
called the embedding number\footnote{Also referred to as the \emph{lattice
    constant} in the lattice community} of a graph onto the lattice, and will
differ from lattice to lattice. For every graph in with a uniform nearest
neighbour interaction the sum over the symbols / coordinates will result in $N$,
the number of lattice sites, times the embedding number of the graph. Although
it might seem wrong to include all possible embeddings as e.g. the $q^2$ in the
$M_1^2 M_2$ term will natually include the embedding corresponding to the
$M_2^2$ term. This is however not a problem as the $M_n$ factors are given in
terms of commutators and applying the methods of moments and cumulants from
\secref{sec:moments_cumulants} we see that the system resolves these issues
automatically. The embedding number is dependent on the lattice, so that e.g.
the three bond graph
%
\begin{equation}
  \begin{tikzpicture}[graph style]
    \node[skeleton node] (n1) {};
    \node[skeleton node] (n2) [right=of n1] {}
      edge[skeleton bond]  (n1);
    \node[skeleton node] [position=60 degrees from n1] {}
      edge[skeleton bond] (n1)
      edge[skeleton bond] (n2);
  \end{tikzpicture}
\end{equation}
%
has an embedding number of zero on a square lattice as there is no way to
resolve the nearest neighbour requirement of \ruleref{rule:free_energy}. On a
triangular lattice on the other hand it would have a non-zero embedding number.
A table of the four bond graphs with symmetry factors and embeddings on a square
lattice can be found in \tabref{tab:graphs_embeddings}. When we later
extrapolate the graphical methods to the effective theory we will see that these
are all graphs needed to carry out a computation to order $\kappa^8$.

\begin{table}[ht]
  \begin{center}
    \includegraphics{embedding_and_symmetry}
  \end{center}
  \caption{Graphs with up to four bonds with symmetry factor and the embeddings
  on a $d$ dimensional square lattice.}
  \label{tab:graphs_embeddings}
\end{table}

In the next section we will see how to map the effective theory onto an LCE
framework and the additional condsiderations that has to be taken into account.

\subsection{LCE for the effective theory at LO}

We start working with the leading order action to establish corresponding
quantities in the effective action to the ones introduced in the previous
section. In the dense regime ($\mu \gg T$) the LO effective partition function
is
%
\begin{equation}
  \mathcal{Z}_2 = \int [ \mathrm{d} U ]_0 \det{}^{N_f} \qstat \exp \bigg(-\frac{1}{2}
  h_2 N_f \sum_{\langle x,y \rangle} W_{11} (x) W_{11} (y) \bigg),
\end{equation}
%
as we showed in \chapref{chap4}. Comparing the above equation with the scalar
field partition function we used to introduce the LCE,
\meqref{eq:scalar_field_Z}, we see that there is a close to one-to-one
correspondence between the two systems
%
\begin{equation}
  \phi \Leftrightarrow W_{11}, \hskip .5cm v \Leftrightarrow h_2 N_f, \hskip .5cm
  e^{-S_0[\phi]} \Leftrightarrow \mathcal{J} (U_0, W_{11}) \det \qstat,
\end{equation}
%
where $\mathcal{J} (U_0, W_{11})$ is the Jacobian determinant for the variable
change. There is however no need to compute $S_0[W_{11}]$ explicitly as the free
energy only depends on the $n$-point functions $M_n$, which in turn depends on
expectation values of the free theory. We define the $n$-point functions in
terms of $z$-functions, which are (for $N_f = 1$)
%
\begin{subequations}
\begin{alignat}{99}
  \centermathcell{z_{0}} &= &&\int \mathrm{d} W \det \qstat
    &&= 1 + 4 h_1^3 + h_1^6, \\
  \centermathcell{z_{(11)}} &=  &&\int \mathrm{d} W \det \qstat
    W_{11} &&= 6 h_1^3 + 3 h_1^6, \\
  \centermathcell{z_{(11)^2}} &= &&\int \mathrm{d} W \det
    \qstat W_{11}^2 &&= 4 h_1^3 + 9 h_1^6, \\
  \centermathcell{z_{(11)^3}} &= &&\int \mathrm{d} W \det
    \qstat W_{11}^3 &&= h_1^3 + 17 h_1^6 + h_1^9, \\
  \centermathcell{z_{(11)^4}} &= &&\int \mathrm{d} W \det
    \qstat W_{11}^4 &&= 21 h_1^6 + 6 h_1^9,
\end{alignat}
\end{subequations}
%
while for $N_f = 2$ they take the values
%
\begin{subequations}
\begin{alignat}{99}
  \centermathcell{z_{0}} &= &&\int \mathrm{d} W \det{}^2 \qstat \,
    &&= 1 + 20h_1^3 + 50 h_1^6 + 20 h_1^9 + h_1^{12}, \\
  \centermathcell{z_{(11)}} &=  &&\int \mathrm{d} W \det{}^2 \qstat \,
    W_{11} &&= 15 h_1^3 + 75 h_1^6 + 45 h_1^9 + 3 h_1^{12}, \\
  \centermathcell{z_{(11)^2}} &= &&\int \mathrm{d} W \det{}^2
  \qstat \, W_{11}^2 &&= 6 h_1^3 + 95 h_1^6 + 96 h_1^9 + 9 h_1^{12}, \\
  \centermathcell{z_{(11)^3}} &= &&\int \mathrm{d} W \det{}^2 
  \qstat \, W_{11}^3 &&= h_1^3 + 90 h_1^6 + 188 h_1^9 + 27 h_1^{12}, \\
  \centermathcell{z_{(11)^4}} &= &&\int \mathrm{d} W \det{}^2
  \qstat \, W_{11}^4 &&= 60 h_1^6 + 312 h_1^9 + 81 h_1^{12}.
\end{alignat}
\end{subequations}
%
The $z$'s have a fairly convoluted naming scheme. The reason for this is
that when we include more orders in the effective action we will need to
put sets of $W_{\{nm\}}$ in the integrand. The $z$'s follow the naming scheme
%
\begin{equation} \label{eq:lce_z_definition}
  z_{(n_1 m_1)^{k_1}\cdots(n_p m_p)^{k_p}} = \int \mathrm{d} W \det{}^{N_f} \qstat
  W_{n_1 m_1}^{k_1} \cdots W_{n_p m_p}^{k_p}.
\end{equation}
%
A list of all the integrated $z$'s needed to compute the results we will present
later is given in \apxref{apx:z_functions}. The $n$-point functions are then
given by
%
\begin{subequations}
\begin{align}
  M_0 &= \log z_0, \\
  M_1 &= \frac{z_{(11)}}{z_0}, \\
  M_2 &= \frac{z_{(11)^2}}{z_0} - \frac{z_{(11)}^2}{z_0^2}, \\
  M_3 &= \frac{z_{(11)^3}}{z_0} - 3 \frac{z_{(11)^2} z_{(11)}}{z_0^2} + 2 \frac{z_{(11)}^3}{z_0^3}, \\
  M_4 &= \frac{z_{(11)^4}}{z_0} - 4 \frac{z_{(11)^3} z_{(11)}}{z_0^2} - 3
  \frac{z_{(11)^2}^2}{z_0^2} + 12 \frac{z_{(11)^2} z_{(11)}^2}{z_0^3} - 6 \frac{z_{(11)}^4}{z_0^4}.
\end{align} 
\end{subequations}
%
With the full analytic result for the $\mathcal{Z}_2$ partition function at hand
we can start comparing with results from the numerical evaluation. However,
first we need to establish the observables.

\subsection{Observables}

We introduced the definition of the observables in \secref{sec-stat-mech}, but
just as we defined them on the lattice in \secref{sec-thermal-lattice-theory},
we have to give them in terms of the parameters we are working with. We use the
fact that  $\mathcal{W}$ is linear in volume in the thermodynamic
limit, which means that the pressure is given simply as
%
\begin{equation}
  \mathcal{P} = T \bigg( \frac{\partial}{\partial V} \log \mathcal{Z}
  \bigg)_{T\mathrlap{,z}} = \frac{T}{V}\,\mathcal{W}.
\end{equation}
%
Similarly, taking the derivative with respect to fugacity is also straight
forward, we can simplify it further using
%
\begin{equation}
  z\, \frac{\partial}{\partial z} \bigg|_{T\mathrlap{,V}} = h_1
  \frac{\partial}{\partial h_1} \bigg|_{T,V}
\end{equation}
%
which means we can define the baryon number density as
%
\begin{equation}
  n_B = \frac{1}{3} n_q = h_1 \frac{1}{3} \bigg( \frac{\partial}{\partial h_1} \frac{\mathcal{W}}{V} \bigg)_{T,V}.
\end{equation}
%
To compute the energy density $e$ we need to compute the
derivative
%
\begin{equation}
  e = T^2 \bigg( \frac{\partial}{\partial T} \frac{\log \mathcal{Z}}{V}
    \bigg)_{z,V}.
\end{equation}
%
We know that $\frac{\log \mathcal{Z}}{V}$ is volume independent, and therefore
the requirement of keeping $V$ constant is automatically fulfilled. We replace
the derivative in $T$ by a derivative in $a$, and therefore have
%
\begin{equation}
  e = -\frac{1}{N_t} \bigg( \frac{\partial}{\partial a} \frac{\mathcal{W}}{V}
    \bigg)_z.
\end{equation}
%
The derivative with respect to the lattice spacing must be handled with a bit of
care. One option is to define the derivative at constant baryon mass
%
\begin{equation}
  a \frac{\partial}{\partial a} (a m_B) = a m_B.
\end{equation}
%
We base the baryon and meson masses on the full heavy quark results from
\citep{Smit:2002introduction}, and later with the additional gauge corrections
from \citep{Langelage:2014vpa}
%
\begin{alignat}{99}
  a m_M &{}={}& \,\acosh &\bigg(1 + \frac{(M^2 - 4)(M^2 - 1)}{2M^2 - 3}\bigg) &&- 24 \kappa^2 \frac{u}{1-u},\\
  a m_B &{}={}& \log &\bigg(\frac{M^3(M^3 - 2)}{M^3 - \frac{5}{4}}\bigg) &&- 18
  \kappa^2 \frac{u}{1-u},
\end{alignat}
%
where $M = \frac{1}{2 \kappa}$. In the strong coupling limit we can use this
to determine $\frac{\partial \kappa}{\partial a}$
%
\begin{multline}
  a \frac{\partial \kappa}{\partial a} = a m_B \Big/
    \:\frac{\partial a m_B}{\partial \kappa} \\
  = \scalemath{0.9}{%
    -\frac{%
        a m_B e^{-a m_B} \big( e^{a m_B} - 8 + 4 \sqrt{4 + e^{a m_B} (e^{a m_B} - 1)} \,\big)%
      }{%
        6 \scalemath{0.8}{\times} 20^{1/3} \sqrt{4 + e^{a m_B} (e^{a m_B} - 1)}
        \big(e^{-a m_B} \big( 2 + e^{a m_B} - \sqrt{4 + e^{a m_B} (e^{a m_B} - 1)} \big) \big)^{2/3}
      }}.
\end{multline}
%
Away from strong coupling one has to define the derivative in $a$ as a
derivative in $\beta$ through the use of implicit functions
%
\begin{equation}
  a \frac{d \beta}{d a} \frac{d \kappa}{d \beta} = - a \frac{d \beta}{d a}
  \frac{\partial a m_B}{\partial \beta} \,\Big/ \:
  \frac{\partial a m_B}{\partial \kappa}.
\end{equation}
%
Assuming strong coupling for now we see that the energy density is given in
terms of the two parameters of the theory
%
\begin{equation} \label{eq:energy_dens_mid_calc}
  e =  -\frac{1}{a N_t} a\frac{\partial \kappa}{\partial a}
    \frac{\partial h_1}{\partial \kappa} \frac{\partial}{\partial h_1}
    \frac{\mathcal{W}}{V} \bigg|_z
  -\frac{1}{a N_t} a\frac{\partial \kappa}{\partial a}
    \frac{\partial h_2}{\partial \kappa} \frac{\partial}{\partial h_2}
    \frac{\mathcal{W}}{V} \bigg|_z.
\end{equation}
%
We use the fact that
%
\begin{equation}
  \frac{\partial h_1}{\partial \kappa} \bigg|_z = \frac{N_t}{\kappa} h_1
\end{equation}
%
to see that the first part of \meqref{eq:energy_dens_mid_calc} is
%
\begin{equation}
  -\frac{1}{a N_t} a\frac{\partial \kappa}{\partial a}
    \frac{\partial h_1}{\partial \kappa} \frac{\partial}{\partial h_1}
    \frac{\mathcal{W}}{V} \bigg|_z
  = - \frac{1}{a} \frac{a}{\kappa} \frac{\partial \kappa}{\partial a} h_1
  \frac{\partial}{\partial h_1} \frac{\mathcal{W}}{V}
  = -3 \frac{1}{a} \frac{a}{\kappa} \frac{\partial \kappa}{\partial a} n_B
\end{equation}
%
after inserting the definition of $n_B$. Using that to first order in $\kappa$
that $\frac{\partial \kappa}{\partial a} \sim \minus{}\kappa \frac{m_B}{3}$ we see that
this first term is somehow related to the rest energy of the system. We subtract
the energy density by this shift, the resulting quantity should give a good estimate for the
binding energy of the system at low temperatures where thermal fluctuations are
suppressed. We define the dimensionless ratio of the binding energy density to
the rest energy to be
%
\begin{equation}
  \epsilon = \frac{e - m_{B,\text{eff}}\, n_B}{m_{B,\text{eff}}\, n_B},
\end{equation}
%
where
%
\begin{equation}
  m_{B,\text{eff}} = -3 \frac{1}{\kappa} \frac{\partial \kappa}{\partial a}.
\end{equation}
%
\begin{figure}[t]%
  {\centering%
    \includegraphics[width=.65\textwidth]{mb_eff_comparison}\par}
  \caption{Comparison of the effective and full baryon mass used in the definition of $\epsilon$.}%
  \label{fig:mb_eff_comparison}%
\end{figure}%
%
We plot the effective baryon mass vs the real baryon mass in the strong coupling
limit for different values of $\kappa$ in \figref{fig:mb_eff_comparison}, and
see that they mostly agree all the way to $\kappa_{\mathrm{crit}}$, which is far
enough away from the parameter range in this study that the two can be
interchanged.

In \figref{fig:convergence_cluster_Z2} we plot the baryon number density in the
strong coupling limit at varying coupling parameter $h_2$, which can be used to
test the convergence of the expansion, similar to what we did in \chapref{chap4}
with \figref{fig:numerical_convergence} (left). We see that the higher order
linked cluster contribution has a small effect on the convergence, and that it
agrees with the results from the simulations.

\begin{figure}
  {\centering
    \includegraphics[width=.65\textwidth]{nb_convergence_cluster_Z2}\par}
  \caption{Convergence of LCE carried out on the LO effective theory
    $\mathcal{Z}_2$, compared to numerical data for the same parameters.}
  \label{fig:convergence_cluster_Z2}
\end{figure}

\begin{figure}
  {\centering
    \includegraphics[width=.48\textwidth]{binding_energy_Z2_k2}
    \includegraphics[width=.48\textwidth]{binding_energy_Z2_k8}
    \par}
  \caption{Binding energy $\epsilon$ for three different values of $\kappa$ with
    the $\mathcal{O}(\kappa^2)$ LCE of $\mathcal{Z}_2$ on the left and with the
    $\mathcal{O}(\kappa^8)$ LCE of $\mathcal{Z}_2$ on the right.}
  \label{fig:binding_energy_Z2}
\end{figure}

At this point a note on the multi level expansions is in order. Using the
methods outlined in \chapref{chap4} we compute the effective theory to some
order in the smallness parameters $\kappa$ and $u$. This defines a unique system
with a unique action, which can then be simulated with Monte Carlo or Langevin
algorithms. Another alternative is to analyse the parition function at the given
expansion order and determine the new expansion parameters (in this case $h_2$).
A second expansion can then be carried out to evaluate this specific system
order by order in this expansion parameter. If the simulation converges to the
correct result it is expected to reproduce the full (all order) linked cluster
expansion result of the effective theory at a given order.

Finally we plot the binding energy ratio $\epsilon$ as a function of the
chemical potential for various values of $\kappa$ in
\figref{fig:binding_energy_Z2}. We see that the binding energy decrease the
constituent quark masses (increase $\kappa$), which is what one would expect.
The plot on the left is to leading order in $h_2$, while the plot on the right
shows the fourth order, $h_2^4$, of the expansion. As one can see higher order
solution develops some interesting behaviour at the higher values of $\kappa$ as
we pass the $\mu_B / m_B = 1$ line. It is however beyond convergence, and we
need more orders in the effective theory before we can say anything about the
behaviour at higher chemical potential.

\subsection{Generalisation of the LCE to polynomial interactions}

Before we can apply the cluster expansion formalism to the full $\mathcal{O}(u^5
\kappa8)$ effective action we need see how to handle interactions that are more
complicated than two point interactions. We start with a generalised partition
function for $n$ component fields $\phi_i$
%
\begin{multline} \label{eq:multi_point_scalar_Z}
  \mathcal{Z} = \int [\mathrm{d} \phi_i] \exp \bigg(-S_0[\phi_i]
  +\frac{1}{2!} \sum_{x,y} \sum_{i,j} v_{ij}(x,y) \phi_i(x) \phi_j(y) \\
  +\frac{1}{3!} \sum_{x,y,z} \sum_{i,j,k} u_{ijk}(x,y,z) \phi_i(x) \phi_j(y)
  \phi_k(z) + \dots \bigg).
\end{multline}
%
We introduce the sources $J_i$ in a similar fashion to how they were introduced
in \secref{sec:classical_lce_nn}. This gives us a linked cluster expansion for
the grand canonical potential
%
\begin{align}
  \mathcal{W}[v,u] = \bigg[ \exp&\bigg( \frac{1}{2!} \sum_{x,y} \sum_{i,j} v_{ij}(x,y) 
    \frac{\delta}{\delta \tilde{v}_{ij}(x,y)} \bigg)  \nonumber \\
  \times\exp&\bigg( \frac{1}{3!} \sum_{x,y,z} \sum_{i,j,k} u_{ijk}(x,y,z)
    \frac{\delta}{\delta \tilde{u}_{ijk}(x,y,z)} \bigg) \cdots
    \bigg] \mathcal{W}[\tilde{v},\tilde{u}] \Bigg|_{\substack{\tilde{v}=0\\\tilde{u}=0\\\cdots}}\;.
\end{align}
%
The derivative with respect to the 3-point coupling $u$ can be expressed in
terms of derivates w.r.t. the sources
%
\begin{multline}
  \frac{\delta \mathcal{W}}{\delta u_{ijk}(x,y,z)} = \frac{\delta^3 \mathcal{W}}{\delta J_i(x) \delta J_j(y) \delta J_k(z)}
    + \frac{\delta \mathcal{W}}{\delta J_i(x)} \frac{\delta^2 \mathcal{W}}{\delta J_j(y) \delta J_k(z)}
    + \frac{\delta \mathcal{W}}{\delta J_j(y)} \frac{\delta^2 \mathcal{W}}{\delta J_i(x) \delta J_k(z)} \\
  \hspace{2cm} + \frac{\delta \mathcal{W}}{\delta J_k(z)} \frac{\delta^2 \mathcal{W}}{\delta J_i(x) \delta J_j(y)}
    + \frac{\delta \mathcal{W}}{\delta J_i(x)} \frac{\delta \mathcal{W}}{\delta J_j(y)}
    \frac{\delta \mathcal{W}}{\delta J_k(z)}\;,
\end{multline}
%
the same is also true for all the higher $n$-point interactions. To
$\mathcal{O}(v^2, u)$ (two bonds) we get
%
{\allowdisplaybreaks%
\begin{multline} 
  \mathcal{W}[v,u] = \mathcal{W}[0] + \frac{1}{2} \sum M_i(x) \,v_{ij}(x,y)\, M_j(y)
  + \frac{1}{4} \sum M_{ij}(x) \,v_{ik}(x,y)v_{jl}(x,y)\, M_{jl}(y)\\
  + \frac{1}{2} \sum M_i(x) \,v_{ij}(x,y)\, M_{jk}(y) \,v_{kl}(y,z)\, M_1(z)\\
  + \frac{1}{3!} \sum u_{ijk}(x,y,z) M_i(x) M_j(y) M_k(z) \\
  + \frac{1}{2} \sum u_{ijk}(x,y,y) M_i(x) M_{jk}(y) + \dots
\end{multline}}%
%
where the sums have been shortened for the sake of brevity and we have assumed
that the 3-point coupling is cyclic. Just as with the 2-point LCE one can
systematically carry out the linked cluster expansion, rewriting the derivatives
in the couplings $v, u, \dots$ order by order and evaluate $\mathcal{W}$.
However a graphical method is desired as it would greatly benefit the
expansion. To do this we need to further specify the geometry of the 3-point
interaction. One natural choice that is compatible with the nearest neighbour
interaction $v$ is a set of two nearest neighbour interactions
%
\begin{equation} \label{eq:udef_wedge}
  u(x,y,z) = 
    \begin{cases}
      \hskip .1cm u & \text{ if $\langle x, y \rangle$ and $\langle y, z \rangle$ are nearest neighbours},\\
      \hskip .1cm u & \text{ if $\langle x, y \rangle$ and $\langle x, z \rangle$ are nearest neighbours},\\
      \hskip .1cm u & \text{ if $\langle x, z \rangle$ and $\langle y, z \rangle$ are nearest neighbours},\\
      \hskip .1cm 0 & \text{ else},
    \end{cases}
\end{equation}
%
where we have reverted to the one-components fields for simplicity. Using this
definition for $u$, $\mathcal{W}$ evaluates to 
%
\begin{multline}
  \mathcal{W}[v,u] = N M_0 + \frac{q}{2} v N M_1^2 + \frac{q}{4} v^2 N M_2^2 \\
    + \frac{q^2}{2} v^2 N M_1^2 M_2 + \frac{q^2}{2} u N M_1^3 + \frac{q}{2} u N
    M_1 M_2 + \dots
\end{multline}
%
Graphically we can represent the above expression as
%
\begin{equation} \label{eq:lce_uv_nn}
  \mathcal{W}[v] \:=\:
  \tikz[graph style] \node[skeleton node] {};
  \:+\: \textstyle\frac{1}{2}  \:
  \begin{tikzpicture}[graph style]
    \node[inner 1] (n1) {};
    \node[inner 1] (n2) [below=of n1] {};
    \draw[bond 1] (n1) -- (n2);
  \end{tikzpicture}
  \:+\: \textstyle\frac{1}{2}  \:
  \begin{tikzpicture}[graph style]
    \node[inner 1] (n1) {};
    \node[inner 1] (n2) [right=of n1]  {};
    \node[inner 1] (n3) [position=60 degrees from n1] {};
    \node[outer 1] (n3o) at (n3) {}
      edge[bond 1] (n1)
      edge[bond 1] (n2);
  \end{tikzpicture} 
  \:+ \textstyle\frac{1}{4} \:
  \begin{tikzpicture}[graph style]
    \node[inner 1] (n1) {};
    \node[outer 1] (n1o) at (n1) {};
    \node[inner 1] (n2) [below=of n1] {};
    \node[outer 1] at (n2) {}
      edge[bond 1, bend left=45]  (n1o)
      edge[bond 1, bend right=45] (n1o);
  \end{tikzpicture}
  \:+\: \textstyle\frac{1}{2}  \:
  \begin{tikzpicture}[graph style]
    \node[inner 2] (n1) {};
    \node[inner 2] (n2) [position=60 degrees from n1]  {};
    \node[inner 2] (n3) [position=300 degrees from n2] {};
    \draw[bond 2] (n1) -- (n2);
    \draw[bond 2] (n2) -- (n3);
  \end{tikzpicture} 
  \:+ \textstyle\frac{1}{2} \:
  \begin{tikzpicture}[graph style]
    \node[inner 2] (n1) {};
    \node[inner 2] (n2) [below=of n1] {};
    \node[outer 2] at (n2) {}
      edge[bond 2, bend left=45]  (n1)
      edge[bond 2, bend right=45] (n1);
  \end{tikzpicture}
  + \dots \,.
\end{equation}
%
Where the $u$ bonds are coloured \ColBaseText{}, the $v$ bond pair is coloured
\ColHlIText{} the nodes are coloured based on the number of fields from which
interaction touches it. This is needed to distinguish e.g. the nodes in the
final term.  An alternative three point coupling could be a triangular nearest
neighbour coupling
%
\begin{equation}
  u(x,y,z) = 
    \begin{cases}
      \hskip .1cm u & \text{ if $\langle x, y \rangle$, $\langle y, z \rangle$,
                      and $\langle x, z \rangle$ are nearest neighbours},\\
      \hskip .1cm 0 & \text{ else},
    \end{cases}
\end{equation}
%
in which case the final term in \meqref{eq:lce_uv_nn} would be excluded as it is
not geometrically realisable. It would actually be better to represent this
particular version of the three point interaction with a three bond diagram
%
\begin{equation}
  \begin{tikzpicture}[graph style]
    \node[inner 2] (n1) {};
    \node[inner 2] (n2) [right=of n1] {}
      edge[bond 2]  (n1);
    \node[inner 2] [position=60 degrees from n1] {}
      edge[bond 2] (n1)
      edge[bond 2] (n2);
  \end{tikzpicture}
\end{equation}
%
as the bonds are intended to encode the nearest neighbour restriction.
Regardless of which geometry the $n$-point interactions have there are 
different paths forward. One option is to compute the derivatives explicitly
order by order and sum over the coordinates explicitly. Alternatively one can
write down all mixed graphs that respect the chosen geometry and compute the
modified symmetry factor. E.g. we see that the two bond nearest neighbour graph
for the $v$ and $u$ interactions
%
\begin{equation}
  \textstyle\frac{1}{4} \:
  \begin{tikzpicture}[graph style]
    \node[inner 1] (n1) {};
    \node[outer 1] (n1o) at (n1) {};
    \node[inner 1] (n2) [below=of n1] {};
    \node[outer 1] at (n2) {}
      edge[bond 1, bend left=45]  (n1o)
      edge[bond 1, bend right=45] (n1o);
  \end{tikzpicture}, \hskip 1cm
  \textstyle\frac{1}{2} \:
  \begin{tikzpicture}[graph style]
    \node[inner 2] (n1) {};
    \node[inner 2] (n2) [below=of n1] {};
    \node[outer 2] at (n2) {}
      edge[bond 2, bend left=45]  (n1)
      edge[bond 2, bend right=45] (n1);
  \end{tikzpicture}
\end{equation}
%
have different symmetry factors due to the fact that the node relabelling
symmetry is broken. A third alternative, and the one we will focus on, is
through a second embedding step
%
\begin{ruledef}{Grand canonical potential $\mathcal{W}$ for $n$ point interactions}%
  \label{rule:free_energy_n_point}%
  \begin{enumerate}
    \item Represent the geometry of the $n$-point interaction $v_n(x_{n_1},
          x_{n_2}, \dots, x_{n_n})$ as a graph according to \defref{def:graph}
    \item Construct all graphs with the necessary number of bonds and geometry
      to the desired order, we refer to these as \emph{skeleton graphs}
    \item Embed all $n$-point interaction graphs onto the skeleton graphs
    \item For every embedded $n$-point graph that visits $x_{n_1}, x_{n_2},
      \dots, x_{n_n}$, add a factor $v_n(x_{n_1}, x_{n_2}, \dots, x_{n_n})$
    \item For every vertex that with \emph{modified valence}\footnote{%
        Modified valence is the number of bonds connecting a vertex originating
        from \emph{different} $n$-point interactions} $p$, add a factor $M_p(x_i)$
    \item The correct symmetry factor will be the symmetry factor of the skeleton
          graph times the number of unique isomorphic embeddings 
  \end{enumerate}%
\end{ruledef}
%
\noindent{}%
Let us consider the embedding of a $v$-link and a $u$-"wedge" from \meqref{eq:udef_wedge}
on the three bond graph
%
\begin{equation}
  \textstyle\frac{1}{2} \:
  \begin{tikzpicture}[graph style]
    \node[skeleton node] (n1) {};
    \node[skeleton node] (n2) [position=60 degrees from n1] {}
      edge[skeleton bond, bend left=45]  (n1) 
      edge[skeleton bond, bend right=45] (n1);
    \node[skeleton node] [position=300 degrees from n2] {}
      edge[skeleton bond] (n2);
  \end{tikzpicture}
\end{equation}
%
This can be done in four different ways
%
\begin{equation}
  \textstyle\frac{1}{2} \:
  \begin{tikzpicture}[graph style]
    \node[inner 1] (n1) {};
    \node[outer 2] (n1o) at (n1) {};
    \node[inner 1] (n2) [position=60 degrees from n1] {};
    \node[outer 2] (n2o) at (n2) {}
      edge[bond 1, bend left=45]  (n1o) 
      edge[bond 2, bend right=45] (n1o);
    \node[inner 2] [position=300 degrees from n2] {}
      edge[bond 2] (n2o);
  \end{tikzpicture}
  +\textstyle\frac{1}{2} \:
  \begin{tikzpicture}[graph style]
    \node[inner 1] (n1) {};
    \node[outer 2] (n1o) at (n1) {};
    \node[inner 1] (n2) [position=60 degrees from n1] {};
    \node[outer 2] (n2o) at (n2) {}
      edge[bond 2, bend left=45]  (n1o) 
      edge[bond 1, bend right=45] (n1o);
    \node[inner 2] [position=300 degrees from n2] {}
      edge[bond 2] (n2o);
  \end{tikzpicture}
  +\textstyle\frac{1}{2} \:
  \begin{tikzpicture}[graph style]
    \node[inner 2] (n1) {};
    \node[outer 2] (n1o) at (n1) {};
    \node[inner 1] (n2) [position=60 degrees from n1] {};
    \node[outer 2] (n2o) at (n2) {}
      edge[bond 2, bend left=45]  (n1o) 
      edge[bond 2, bend right=45] (n1o);
    \node[inner 1] [position=300 degrees from n2] {}
      edge[bond 1] (n2o);
  \end{tikzpicture}
  +\textstyle\frac{1}{2} \:
  \begin{tikzpicture}[graph style]
    \node[inner 2] (n1) {};
    \node[inner 1] (n2) [position=60 degrees from n1] {};
    \node[outer 2] (n2o) at (n2) {};
    \node[outerouter 2] (n2oo) at (n2) {}
      edge[bond 2, bend left=45]  (n1) 
      edge[bond 2, bend right=45] (n1);
    \node[inner 1] [position=300 degrees from n2] {}
      edge[bond 1] (n2oo);
  \end{tikzpicture}
\end{equation}
%
where the first two are isomorphic and can be collected to one term. Hence there
are three unique embeddings of this combination onto the skeleton graph in
question, where one of the embeddings has a non-unit embedding number, and thus
a modified symmetry factor. With these two ingredients we present the
full three bond free energy
%
%\begin{multline} \label{eq:lce_uv_3b}
  \mathcal{W}[v] \:=\:
  \tikz[graph style] \node[skeleton node] {};
  \:+\: \textstyle\frac{1}{2}  \:
  \begin{tikzpicture}[graph style]
    \node[inner 1] (n1) {};
    \node[inner 1] (n2) [below=of n1] {};
    \draw[bond 1] (n1) -- (n2);
  \end{tikzpicture}
  \:+\: \textstyle\frac{1}{2}  \:
  \begin{tikzpicture}[graph style]
    \node[inner 1] (n1) {};
    \node[inner 1] (n2) [right=of n1]  {};
    \node[inner 1] (n3) [position=60 degrees from n1] {};
    \node[outer 1] (n3o) at (n3) {}
      edge[bond 1] (n1)
      edge[bond 1] (n2);
  \end{tikzpicture} 
  \:+ \textstyle\frac{1}{4} \:
  \begin{tikzpicture}[graph style]
    \node[inner 1] (n1) {};
    \node[outer 1] (n1o) at (n1) {};
    \node[inner 1] (n2) [below=of n1] {};
    \node[outer 1] at (n2) {}
      edge[bond 1, bend left=45]  (n1o)
      edge[bond 1, bend right=45] (n1o);
  \end{tikzpicture}
  \:+\: \textstyle\frac{1}{2}  \:
  \begin{tikzpicture}[graph style]
    \node[inner 2] (n1) {};
    \node[inner 2] (n2) [position=60 degrees from n1]  {};
    \node[inner 2] (n3) [position=300 degrees from n2] {};
    \draw[bond 2] (n1) -- (n2);
    \draw[bond 2] (n2) -- (n3);
  \end{tikzpicture} 
  \:+ \textstyle\frac{1}{2} \:
  \begin{tikzpicture}[graph style]
    \node[inner 2] (n1) {};
    \node[inner 2] (n2) [below=of n1] {};
    \node[outer 2] at (n2) {}
      edge[bond 2, bend left=45]  (n1)
      edge[bond 2, bend right=45] (n1);
  \end{tikzpicture}
  \:+ \textstyle\frac{1}{3!} \:
  \begin{tikzpicture}[graph style,node distance=.4]
    \node[inner 1] (n1) {};
    \node[outer 1] at (n1) {};
    \node[outerouter 1] (n1o) at (n1) {};
    \node[inner 1] (n2) [position=330 degrees from n1] {}
      edge[bond 1] (n1o);
    \node[inner 1] (n3) [position=90 degrees from n1] {}
      edge[bond 1] (n1o);
    \node[inner 1] (n4) [position=210 degrees from n1] {}
      edge[bond 1] (n1o);
  \end{tikzpicture}\\
  \:+ \textstyle\frac{1}{2} \:
  \begin{tikzpicture}[graph style]
    \node[inner 1] (n1) {};
    \node[inner 1] (n2) [position=60 degrees from n1] {};
    \node[outer 1] (n2o) at (n2) {}
      edge[bond 1] (n1);
    \node[inner 1] (n3) [position=300 degrees from n2] {};
    \node[outer 1] (n3o) at (n3) {}
      edge[bond 1] (n2o);
    \node[inner 1] (n4) [position=60 degrees from n3] {}
      edge[bond 1] (n3o);
  \end{tikzpicture}
  \:+ \textstyle\frac{1}{6} \:
  \begin{tikzpicture}[graph style]
    \node[inner 1] (n1) {};
    \node[outer 1] (n1o) at (n1) {};
    \node[inner 1] (n2) [position=60 degrees from n1] {};
    \node[outer 1] (n2o) at (n2) {}
      edge[bond 1] (n1o);
    \node[inner 1] (n3) [position=300 degrees from n2] {};
    \node[outer 1] (n3o) at (n3) {}
      edge[bond 1] (n2o)
      edge[bond 1] (n1o);
  \end{tikzpicture}
  \:+\textstyle\frac{1}{2} \:
  \begin{tikzpicture}[graph style]
    \node[inner 1] (n1) {};
    \node[inner 1] (n2) [position=60 degrees from n1] {};
    \node[outer 1] (n2o) at (n2) {};
    \node[outerouter 1] (n2oo) at (n2) {}
      edge[bond 1, bend left=45]  (n1) 
      edge[bond 1, bend right=45] (n1);
    \node[inner 1] [position=300 degrees from n2] {}
      edge[bond 1] (n2oo);
  \end{tikzpicture}
  \:+ \textstyle\frac{1}{2{\scriptstyle\times}3!} \:
  \begin{tikzpicture}[graph style]
    \node[inner 1] (n1) {};
    \node[outer 1] at (n1) {};
    \node[outerouter 1] (n1o) at (n1) {};
    \node[inner 1] (n2) [below=of n1] {};
    \node[outer 1] at (n2) {};
    \node[outerouter 1] at (n2) {}
      edge[bond 1]  (n1o)
      edge[bond 1, bend right=45] (n1o)
      edge[bond 1, bend left=45] (n1o);
  \end{tikzpicture}
  \:+ \textstyle\frac{1}{2} \:
  \begin{tikzpicture}[graph style,node distance=.4]
    \node[inner 1] (n1) {};
    \node[outer 2] (n1o) at (n1) {};
    \node[inner 2] (n2) [position=330 degrees from n1] {}
      edge[bond 2] (n1o);
    \node[inner 1] (n3) [position=90 degrees from n1] {}
      edge[bond 1] (n1o);
    \node[inner 2] (n4) [position=210 degrees from n1] {}
      edge[bond 2] (n1o);
  \end{tikzpicture} \\
  \:+ 
  \begin{tikzpicture}[graph style]
    \node[inner 2] (n1) {};
    \node[inner 2] (n2) [position=60 degrees from n1] {}
      edge[bond 2] (n1);
    \node[inner 1] (n3) [position=300 degrees from n2] {};
    \node[outer 2] (n3o) at (n3) {}
      edge[bond 2] (n2);
    \node[inner 1] (n4) [position=60 degrees from n3] {}
      edge[bond 1] (n3o);
  \end{tikzpicture}
  \:+ \textstyle\frac{1}{2} \:
  \begin{tikzpicture}[graph style]
    \node[inner 1] (n1) {};
    \node[outer 2] (n1o) at (n1) {};
    \node[inner 2] (n2) [position=60 degrees from n1] {}
      edge[bond 2] (n1o);
    \node[inner 1] (n3) [position=300 degrees from n2] {};
    \node[outer 2] (n3o) at (n3) {}
      edge[bond 2] (n2)
      edge[bond 1] (n1o);
  \end{tikzpicture}
  \:+
  \begin{tikzpicture}[graph style]
    \node[inner 1] (n1) {};
    \node[outer 2] (n1o) at (n1) {};
    \node[inner 1] (n2) [position=60 degrees from n1] {};
    \node[outer 2] (n2o) at (n2) {}
      edge[bond 2, bend left=45]  (n1o) 
      edge[bond 1, bend right=45] (n1o);
    \node[inner 2] [position=300 degrees from n2] {}
      edge[bond 2] (n2o);
  \end{tikzpicture}
  \:+\textstyle\frac{1}{2} \:
  \begin{tikzpicture}[graph style]
    \node[inner 2] (n1) {};
    \node[outer 2] (n1o) at (n1) {};
    \node[inner 1] (n2) [position=60 degrees from n1] {};
    \node[outer 2] (n2o) at (n2) {}
      edge[bond 2, bend left=45]  (n1o) 
      edge[bond 2, bend right=45] (n1o);
    \node[inner 1] [position=300 degrees from n2] {}
      edge[bond 1] (n2o);
  \end{tikzpicture}
  \:+\textstyle\frac{1}{2} \:
  \begin{tikzpicture}[graph style]
    \node[inner 2] (n1) {};
    \node[inner 1] (n2) [position=60 degrees from n1] {};
    \node[outer 2] (n2o) at (n2) {};
    \node[outerouter 2] (n2oo) at (n2) {}
      edge[bond 2, bend left=45]  (n1) 
      edge[bond 2, bend right=45] (n1);
    \node[inner 1] [position=300 degrees from n2] {}
      edge[bond 1] (n2oo);
  \end{tikzpicture}
  \:+ \textstyle\frac{1}{2} \:
  \begin{tikzpicture}[graph style]
    \node[inner 1] (n1) {};
    \node[outer 2] at (n1) {};
    \node[outerouter 2] (n1o) at (n1) {};
    \node[inner 1] (n2) [below=of n1] {};
    \node[outer 2] at (n2) {}
      edge[bond 2]  (n1o)
      edge[bond 1, bend right=45] (n1o)
      edge[bond 2, bend left=45] (n1o);
  \end{tikzpicture}
  \:+\dots\:.
\end{multline}

\begin{equation} \label{eq:lce_uv_3b}
  \tikz \node[anchor=base, baseline, minimum width=.9\textwidth, draw, minimum height=2cm, ColourHl1,thick] {Dummy equation}; 
\end{equation}

\subsection{Application to the effective theory, graph embeddings}

\section{Analytic resummation} \label{sec:analytic_resummation}
\section{Observables} \label{sec:evaluation}
\section{Large \texorpdfstring{$N_c$}{Nc} limit} \label{sec:large_nc_study}
\section{Yang Lee Zeros} \label{sec:yang_lee_zeros}
