\chapter{Introduction}
\markright{Introduction}

The current state of human knowledge suggests that majority of the visible
matter in the universe is made up of hadrons which in turn consist of
the more fundamental \emph{quarks}. We have so far discovered six species, or
flavours, of quarks, namely the up, down, strange, charm, top and bottom. These
fundamental particles carry three sets of charges: electric charge, flavour
charge and colour charge. The latter manifests itself
through the confinement process that binds the quarks together into inseparable
hadrons, and the resulting binding energy is responsible for almost 99 \% of the
mass of these bound particles. For example the proton has a mass of 938.27 MeV,
while its constituents, two up quarks and a single down quark, have a total rest
mass of no more than 9.8 (1.9) MeV \citep{Agashe:2014kda}. It is therefore of
great importance to understand the dynamics governing interactions between
particles carrying colour charges and the force's mediators, the \emph{gluons}. 

Throughout the various stages of discoveries in the world of particle physics,
the theories we use to describe nature have evolved. The current reigning model
of the universe is called the Standard Model of particle physics, and
categorises the known world consisting of 6 leptons, 18 quarks, 13 mediators,
and their antiparticles into symmetry groups. This theory describes three of
the four established fundamental forces as a quantum theory of fields,
electromagnetism, the strong- and weak nuclear forces, and it is
the most successful theory to date, predicting experimental values with
astonishing accuracy.

The subset of the Standard Model that describes the interaction between quarks
and gluons is called \emph{Quantum Chromodynamics} (QCD). The advent of QCD came
in the 1960's, in a period where a great number of "fundamental" particles were 
discovered. Both \cite{GellMann:1962xb} and \cite{Ne'eman:1961cd} found
structure and symmetry in this zoo of new particles, and to explain this
phenomenon, the existence of quarks was suggested \citep{GellMann:1964nj}.
Due to the fermionic nature of the quarks, an additional quantum number was
needed to allow for the quarks inside baryons to occupy the required spin
and flavour quantum states. This new quantum number was named \emph{colour},
hence \emph{chromo}dynamics, from ancient greek χρῶμα \citep{Greenberg:1964pe}.

QCD has an extremely rich structure, being confining at low energies, while also
possessing \emph{asymptotic freedom}. It has an internal energy scale,
$\Lambda_{\mathrm{QCD}} \approx 200 \:\mathrm{MeV}$, which arise from
dimensional transmutation, and gives the defining length scales of the theory.
It also has a non-trivial topological structure, resulting in instanton
configurations in the vacuum \citep{'tHooft:1976up}.

\begin{figure}[t]
  \begin{center}
    \begin{adjustbox}{max width=\textwidth}
      \raisebox{.135cm}{\includegraphics{qcd_phase_diag}} \hskip .2cm
      \includegraphics{columbia_plot}
    \end{adjustbox}
  \end{center} \vskip -.5cm
  \caption{Left: A simplified, conjectured phase diagram of QCD with path
    symbolising the scan lines of various current and future experiments. Right:
    The Columbia plot, showing the deconfinement transition order for various
    quark masses.}
  \label{fig:phase_diags}
\end{figure}

Of particular interest is the phase diagram of QCD in the temperature-chemical
potential plane, sketched in \figref{fig:phase_diags} (left). At low
temperatures and densities, QCD is confining, and thus the effective degrees of
freedom are the bound states, baryons an mesons. Moving along the temperature
axis, one eventually passes into the quark-gluon-plasma state, a state in which
the thermodynamic energies are high enough that the quarks and gluons deconfine.
In this state the quarks and gluons move around semi-freely, weakly interacting
with each other within the bulk of the plasma. This \emph{deconfinement
  transition} is in fact not a phase transition at all, but has been shown to be
a crossover transition, with a pseudo-critical temperature $T_c \approx 150-170
\:\mathrm{MeV}$ \citep{Aoki:2006we}. It is still an open question whether moving
along this line of crossover transitions eventually will result in a phase
transition and an accompanying critical endpoint. The nature of the crossover
transition depends on the constituent quark masses and the number of quark
flavours, degenerate or otherwise. This dependence is often represented in a
\emph{Columbia} plot, \figref{fig:phase_diags} (right). Although the physical
point clearly sits in the crossover area, the existence of a critical end-point
depends on whether the lower $Z(2)$ critical line shrinks or expands at
increasing chemical potential.  It is also an open question whether this
critical line actually touch the chiral axis in the left or not
\citep{Philipsen:2016hkv}.

Back to the $T$-$\mu$ phase diagram, moving along the chemical potential axis at
zero temperature, one finds a first order liquid-gas phase transition at chemical
potentials around the proton mass.  Following this transition curve to higher
temperature, it will eventually end in a critical endpoint. Continuing to even
higher chemical potentials, Cooper's theorem tells us that we should find
various colour superconducting phases. This has been shown for asymptotically
large densities. However, no first principle proof for intermediate values
exists. Due to the nature of QCD, and the enlarged number of quantum numbers,
there are multiple superconducting phases. These phases differ in which
quantum numbers are fixed to create Cooper pairs. One has the $2$-flavour
superconducting phase and the colour-flavour-locked superconducting phase, where
the colour indices play different roles in the two cases. For a discussion on the
phase structure of QCD see e.g.  \citep{Rajagopal:2000wf,Rischke:2003mt}.

The goal of this thesis is to attempt a study of QCD at low temperatures and
high densities, close to the liquid-gas phase transition. Due to the confining
nature of QCD, such a study cannot be carried out with the use of perturbation
theory. A common approach to this is using low energy effective models
for QCD, such as hadron gas resonant models, and meson models. We will be more
ambitious, and will undertake a first principle study through the medium
of lattice gauge theories. As we will see in the introductory chapters, the
lattice formalism is ill-suited for carrying out simulations in the cold and
dense regime of QCD due to the sign problem, which we will surmount by the use
of an effective lattice theory that has matching series coefficients to full QCD
in a specific parameter region.

The outline of the thesis is as follows. We introduce the necessary
formalism in \chapref{chap2}. This will include the quantum field theoretical
description of a special type of quantified local symmetries, and a short
introduction to the group theory of continuous groups. We then introduce lattice
gauge theory through the discretisation of space-time, and finally an overview
of important lattice concepts.

In \chapref{chap3} we proceed to introducing the necessary formalism for
transitioning our quantum field theory, and with it lattice gauge theory, into
the realm of thermodynamics and statistical mechanics. In this section we take
great care to properly define difficulties that arise in high density lattice
simulations, namely the sign problem. We also present a handful of "remedies",
and discuss their applicability.

This is followed by \chapref{chap4} in which we introduce our method of choice
for dealing with the sign problem. This is a systematic series expansion
approach around the dynamics of heavy quarks ($m_q \to \infty$) and strong
coupling ($g \to \infty$). The derivation of the effective theory is carried out
in great detail, and various important steps for convergence is discussed. We
round off the chapter with a discussion on the numerical evaluation of this
simplified lattice theory.

The thesis culminates in the purely analytical treatment of this effective
lattice theory. \hyperref[chap5]{\mbox{\textsc{Chapter} \ref*{chap5}}} begins
with the introduction of the linked cluster expansion, which enter as a bedrock
on which we build the remainder of the analysis, and is the de factor method for
studies of thermodynamics through series expansions. We then develop a
generalisation to the cluster expansion for more intricate theories, and name it
the \emph{polymer linked cluster expansion}. We then introduce powerful
resummation schemes to extend the region of relevance of our theory further.
Finally, the equation of state for cold and dense, heavy QCD is studied and
discussed.
