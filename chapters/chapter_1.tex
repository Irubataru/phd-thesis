\chapter{Introduction}

The current state of human knowledge suggests that the majority of the visible
matter in the universe is made up of hadrons that are themselves built up from
the more fundamental \emph{quarks}. We have so far discovered six species, or
flavours, of quarks, namely the up, down, strange, charm, top and bottom. These
fundamental particles carry three sets of charges: electric charge, flavour
charge and colour charge.  It is the latter of these that manifest itself
through the confinement process that binds the quarks together into inseparable
hadrons, and the resulting binding energy is responsible for almost 99 \% of the
mass of these bound particles. For example the proton has a mass of 938.27 MeV,
while its constituents, two up quarks and a single down quark, have a total rest
mass of no more than 9.8 (1.9) MeV \citep{Agashe:2014kda}. It is therefore of
great importance to understand the dynamics governing interactions between
particles carrying colour charges and the force's mediators, the \emph{gluons}. 

Through the various stages of discoveries within the world of particle physics,
the theories we use to describe nature has evolved. The current reigning model
of the universe is called the Standard Model of particle physics which
categorises the known world consisting of 6 leptons, 18 quarks, 13 mediators,
and their anti-particles into symmetry groups. This theory describes three of
the four established fundamental forces as a quantum theory of fields, and is
the most successful theory to date, predicting experimental values with
astonishing accuracy.
