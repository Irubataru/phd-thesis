\chapter{Statistical Mechanics and Phase Transitions}

In this chapter we will give a short overview of some aspects of statistical
mechanics that are important to the present work in \secref{sec-stat-mech}.
Then the field theoretical equivalent is given in
\secref{sec-thermal-field-theory}, before we once more study the discretised
theory in \secref{sec-thermal-lattice-theory}. We tie up the chapter 
in \secrefs{sec-finite-density-lattice,sec-sign-problem} with a
thorough discussion of simulations of finite density systems, and the hurdles
that need to be overcome.

Several volumes of the standard literature has been consulted with respect to
the contents of this chapter. These include, but are not limited to,
\cite{landau2013statistical,pathria2011statistical} for introductions to
statistical mechanics, \cite{kapusta2006finite} with an introduction to Thermal
Field Theory (\emph{TFT}), and \cite{montvay1997quantum,Philipsen:2010gj} for
the lattice formulation of finite temperature theory.

\section{Statistical mechanics} \label{sec-stat-mech}

Every quantity of interest to anyone studying equilibrium thermodynamics is
encoded into the previously mentioned partition function
%
\begin{equation} \label{eq-sm-partition-function} \mathcal{Z} = \sum_i
  \mathcal{Z}_i = \sum_i e^{-\beta E_i} \,.  \end{equation}
%
The sum goes over all states of the degrees of freedom of the system with $E_i$
being the energy cost of the configuration, and $\beta = 1/T$ is the reciprocal
temperature\footnote{This should not be confused with the lattice gauge
  coupling, which was introduced in the previous chapter. $\beta$ will
  exclusively be used to indicate $1/T$ in this chapter only}. How one chooses
to define the scope of these \emph{states} depends on the physics one is
interested in obtaining, as all choices should in the end be independent,
although different physical quantities are easier to extract in some rather than
others. Three common descriptions are the \emph{microcanonical}-,
\emph{canonical}- and \emph{grand canonical} ensembles.  They differ in their
scope as the microcanonical ensemble includes systems at a fixed energy shell,
the canonical ensemble a system of fixed temperature and particle number while
the grand canonical ensemble also allows for fluctuations in particle numbers.

Both the canonical- and grand canonical ensembles have associated partition
functions, and the two can be related through a simple relation
%
\begin{equation}
  \mathcal{Z}_{\text{GC}}(z, V, T) = \sum_{N=0}^{\infty} z^N \mathcal{Z}_{\text{C}}(N, V, T),
\end{equation}
%
where $z = e^{\beta \mu}$ is the fugacity, $\mu$ being the chemical
potential. The partition function can in turn be used to calculate a multitude
of thermodynamic quantities
%
\begin{align}
  \mathcal{P} &= \frac{1}{\beta} \bigg( \frac{\partial}{\partial V} \log \mathcal{Z}
    \bigg)_{\beta,z}, \\
  \mathcal{E} &= \,\minus\, \bigg( \frac{\partial}{\partial \beta}
    \log \mathcal{Z} \bigg)_{z,V}, \\
  \mathcal{N} &= \;z\; \bigg( \frac{\partial}{\partial z}
    \log \mathcal{Z} \bigg)_{\beta,V},
\end{align}
%
pressure, average energy and average particle number respectively.

\subsection{Phases and phase transitions}

The phase of a system is linked to the characteristic behaviour of one or more
of its physical quantities, such as the magnetisation of spin glasses, the
average positions of atoms in a crystal and the molecular composition of
solutions. We refer to this defining physical physical quantity as the
\emph{order parameter} of the phase.  The transition between two or more phases
are known as \emph{phase transitions}, and due to the very nature of phases,
happen through the induction of mathematical singularities. 

The order of a phase transition is exactly described by which derivative of the
free energy with respect to the order parameter of choice diverge. First order
transitions have discontinuous first derivatives, while only higher order
derivatives diverge for second order transitions.

However, if the order parameter tends towards zero, but never quite reaches it,
one can define a pseudo-phase transition, known as a \emph{crossover}, to be the
point of the most rapid change. While this is not a real phase transition,
following its trajectory in a phase diagram will result in one if one exists,
linking the two concepts.

Phase transitions is an inherently macroscopic concept, and does not know about
the microscopic details of the theory. Theories with different microscopic
details might therefore still behave alike on the macroscopic level. These
categorise into \emph{universality classes}, catalogued by the singular
behaviour of their physical quantities close to the transition. If we denote the
order parameter by $m$, and the ordering field by $h$\footnote{a characteristic
variable which define the transition, $m\to0$ as $h\to0$}, we can define some of
the critical exponents by
%
\begin{align}
  m \;&\sim (T - T_c)^{\beta}, \\
  \bigg(\frac{\partial m}{\partial h}\bigg)_{\mathrlap{T}} &\sim (T - T_c)^{-\gamma}, \\
  C_V = - \beta^2 &\bigg(\frac{\partial^2 \mathcal{F}}{\partial
    \beta^2}\bigg)_{\mathrlap{V}} \sim (T - T_c)^{-\alpha}.
\end{align}
%
Systems which have the same critical exponents fall into the same universality
class. A notable example is the $4D$ Ising model whose critical exponents are
$\alpha = 0$, $\beta = 1/2$, and $\gamma = 1$.

\subsection{Yang Lee zeros}

An alternative approach to a rigorous study of the phase transitions was
proposed by Yang and Lee \citep{Yang:1952be,Lee:1952ig}. They suggested that one
can analyse the properties of the thermodynamic functions around a transition by
studying the zeros of the grand canonical partition function in the complex
fugacity-plane.

Seeing as the partition function can be interpreted as the normalisation of a
statistical probability function, it can never be zero for real and positive
values of $z$. However, as one increases the volume of the system, the number of
zeros increases, and tends to a continuous curve in the thermodynamic limit,
$V\to\infty$. As this curve forms, zeros on the positive and negative imaginary
axis could tend towards coinciding at the real axis, signalling the onset of a
phase transition.

\section{Thermal field theory} \label{sec-thermal-field-theory}

The quantum partition function is defined similar to the classical one, but with
the energy and particle numbers promoted to operators
%
\begin{equation}
  \mathcal{Z}_{\text{GC}} = \sum_{\mathclap{[\phi_i, \pi_i]}}
  \big\langle [\phi_i, \pi_i] \big|
    z^{\hat{N}} e^{-\beta \hat{H}} \big| [\phi_i, \pi_i] \big\rangle \,.
\end{equation}
%
This sum reduces to a form similar to \meqref{eq-partition} after introducing
the second quantisation and evaluating the integrals over the conjugate momentum
fields $\pi_i$.

% TODO is the interval of t_E really half open?

Focussing on the canonical ensemble, there are two amendments which need to be
made to the Euclidean action in \meqref{eq-action-def}. First the Euclidean time
$\tau_E$ have to be integrated over the half open interval $[0, \beta)$
%
\begin{equation}
  \mathcal{S}_E[\phi_i] = \int_{\mathrlap{0}}^{\mathrlap{\beta}} \mathrm{d} \tau_E \int \mathrm{d}^3 x \,
    \mathcal{L}_E\big[\phi_i(x)\big].
\end{equation}
%
The second rectification is a bit more subtle and concerns the periodic boundary
conditions of the fields. For the fields to respect their statistical properties
(Bose- vs Fermi statistics) they need to obey the boundary conditions
%
\begin{equation} \label{eq-boundary-conditions}
  \phi(\tau_E + \beta) = \begin{cases}
    \hphantom{\minus}\phi(\tau_E), \hskip1em \text{for bosonic fields}, \\
    \minus\phi(\tau_E), \hskip1em \text{for fermionic fields}.
  \end{cases}
\end{equation}

We can now return to the grand canonical ensemble and the particle number
operator $\hat{N}$. A suitable operator can be created from the conserved
Noether charges, such as the one from \meqref{eq-noether-charge}. For every
distinct conserved charge one wishes to study the dynamics of, there should be a
separate chemical potential. Two common systems of interest are those of baryon-
and isospin chemical potential. Baryon chemical potential is connected to the
global U$(1)$ symmetry of the fermion fields. The associated Noether charge is
that of total fermion number
%
\begin{equation}
  N_f = \int \mathrm{d}^3 x \, \bar{\psi}_f(x) \gamma^0 \psi_f(x),
\end{equation}
%
with either separate or degenerate $\mu_f$ for each of the fermion species. The
isospin charge is related to the Cartan generator of the flavour mixing SU$(2)$
symmetry of a two flavour system (the third Pauli matrix)
%
\begin{equation}
  N_I  = \int \mathrm{d}^3 x \, \bar{\psi}_i(x) \gamma^0 (\sigma^3)_{ij}
    \psi_j(x),
\end{equation}
%
which for QCD gives the up and down quarks chemical potentials that only differ
in sign. By tuning the value of $\mu_I$, one can study the effects of a $u/d$
asymmetry, while adjusting $\mu_f$ induces a particle/anti particle imbalance.

\section{Thermal fields on the lattice} \label{sec-thermal-lattice-theory}

Following the same path as in \secref{sec-lattice_intro}, discretising the
expression for the partition function $\mathcal{Z}$ is straightforward. We see
that lattice theory naturally describe systems at non zero temperature, as the
temporal extent always will be finite. The temperature is given in terms of the
number of temporal lattice sites, $N_t$,
%
\begin{equation}
  T = \frac{1}{a N_t}.
\end{equation}
%
Therefore, when going to continuum physics we only want to take the proper
infinite volume limit for the spatial direction, and in the time direction send
$a\to0$ and $N_t\to\infty$ in such a way that $a N_t$ remains finite.

At a first glance it might seem as if when integrating lattice theories we can
only vary the temperature in finite increments, seeing as $N_t$ has to be an integer.
It is however much more common to keep $N_t$ fixed while varying $a$, which one
varies by tuning the gauge coupling $g$, as we saw in \secref{sec-scale_setting}.

Other than imposing the proper boundary conditions for the variables, no other
amendments need to be made for the finite temperature simulations. These follow
directly from \meqref{eq-boundary-conditions}
%
\begin{align}
  \psi(n_t + N_t) &= \minus \psi(n_t), \\
  U_{\mu}(n_t + N_t) &= \hphantom{\minus} U_{\mu}(n_t).
\end{align}

\subsection{Thermodynamic quantities}

In \secref{sec-stat-mech} we stated the thermodynamic quantities $\mathcal{P}$,
$\mathcal{E}$ and $\mathcal{N}$ in terms of derivatives of the partition
function. However one cannot sample the partition function through Monte Carlo
simulations, and we therefore have to device a different scheme to calculate
these quantities. By interchanging the order of the derivatives and the
integrals we get 
%
\begin{equation}
  \mathcal{E} = \minus\frac{\partial}{\partial \beta} \log \mathcal{Z}
    = \frac{1}{\mathcal{Z}} \int \prod_i \mathrm{d} \phi_i \, 
    \bigg( \frac{\partial\mathcal{S}}{\partial \beta} \bigg) e^{-\mathcal{S}}
    \equiv \bigg\langle \frac{\partial\mathcal{S}}{\partial \beta} \bigg\rangle
\end{equation}
%
However, there is still a problem in defining the derivative with respect to
$\beta$. As we wish to keep $N_t$ fixed, we have to vary $\beta$ by varying $a$,
but we only want to vary the lattice spacing in the temporal direction, not the
spatial ones. By naively varying $a$, we would not only change the temperature,
but also the volume, which is supposed to stay fixed in the definition of
$\mathcal{E}$.

To rectify this we have to introduce an anisotropic lattice, one in which the
spatial and temporal lattice spacings are different. The magnitude of this
difference is encoded in the anisotropy parameter $\xi = a_s / a_t$. By varying
$a_s$ and $a_t$ separately we can compute derivatives with respect to
temperature and volume independently, and calculate the thermodynamic quantities
%
\begin{alignat}{99}
  \mathcal{P} &{}={}& \centermathcell{\minus\frac{a_s}{3 \beta V}} \bigg\langle &
    \centermathcell{\frac{\partial \mathcal{S}}{\partial a_s}} &\bigg\rangle, \\
  \mathcal{E} &{}={}& \centermathcell{\frac{a_t}{\beta}} \bigg\langle &
    \centermathcell{\frac{\partial \mathcal{S}}{\partial a_t}} &\bigg\rangle, \\
  \mathcal{N} &{}={}& \centermathcell{\minus{}z} \bigg\langle & 
    \centermathcell{\frac{\partial \mathcal{S}}{\partial z}} &\bigg\rangle
\end{alignat}
%
Now that they are expressed in terms of ensemble averages we can use the
previously introduced Monte Carlo methods to estimate their values.

\section{Finite density simulations} \label{sec-finite-density-lattice}

\section{The sign problem, NP hard, exponential cancellations}
\label{sec-sign-problem}

\subsection{Reweighting}
\subsection{Analytic extrapolation}
\subsection{Taylor series}
\subsection{Stochastic quantisation}
