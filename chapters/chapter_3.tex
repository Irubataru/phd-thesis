\chapter{Statistical Mechanics and Phase Transitions}

In this chapter we will give a short overview of some aspects of statistical
mechanics that are important to the present work in \secref{sec:stat-mech}.
Then the field theoretical equivalent is given in
\secref{sec:thermal-field-theory}, before we once more study the discretised
theory in \secref{sec:thermal-lattice-theory}. We tie up the chapter 
in \secrefs{sec:finite-density-lattice,sec:sign-problem} with a
thorough discussion of simulations of finite density systems, and the hurdles
that need to be overcome.

Several volumes of the standard literature has been consulted with respect to
the contents of this chapter. These include, but are not limited to,
\cite{landau2013statistical,pathria2011statistical} for introductions to
statistical mechanics, \cite{kapusta2006finite} with an introduction to Thermal
Field Theory (\emph{TFT}), and \cite{montvay1997quantum,Philipsen:2010gj} for
the lattice formulation of finite temperature theory.

\section{Statistical mechanics} \label{sec:stat-mech}

Every quantity of interest to anyone studying equilibrium thermodynamics is
encoded into the previously mentioned partition function
%
\begin{equation} \label{eq:sm-partition-function} \mathcal{Z} = \sum_i
  \mathcal{Z}_i = \sum_i e^{-\beta E_i} \,.  \end{equation}
%
The sum goes over all states of the degrees of freedom of the system with $E_i$
being the energy cost of the configuration, and $\beta = 1/T$ is the reciprocal
temperature\footnote{This should not be confused with the lattice gauge
  coupling, which was introduced in the previous chapter. $\beta$ will
  exclusively be used to indicate $1/T$ in this chapter only}. How one chooses
to define the scope of these \emph{states} depends on the physics one is
interested in obtaining, as all choices should in the end be independent,
although different physical quantities are easier to extract in some rather than
others. Three common descriptions are the \emph{microcanonical}-,
\emph{canonical}- and \emph{grand canonical} ensembles.  They differ in their
scope as the microcanonical ensemble includes systems at a fixed energy shell,
the canonical ensemble a system of fixed temperature and particle number while
the grand canonical ensemble also allows for fluctuations in particle numbers.

Both the canonical- and grand canonical ensembles have associated partition
functions, and the two can be related through a simple relation
%
\begin{equation}
  \mathcal{Z}_{\text{GC}}(z, V, T) = \sum_{N=0}^{\infty} z^N \mathcal{Z}_{\text{C}}(N, V, T),
\end{equation}
%
where $z = e^{\beta \mu}$ is the fugacity, $\mu$ being the chemical
potential. The partition function can in turn be used to calculate a multitude
of thermodynamic quantities
%
\begin{align}
  \mathcal{P} &= \frac{1}{\beta} \bigg( \frac{\partial}{\partial V} \log \mathcal{Z}
    \bigg)_{\beta,z}, \\
  \mathcal{E} &= \,\minus\, \bigg( \frac{\partial}{\partial \beta}
    \log \mathcal{Z} \bigg)_{z,V}, \\
  \mathcal{N} &= \;z\; \bigg( \frac{\partial}{\partial z}
    \log \mathcal{Z} \bigg)_{\beta,V},
\end{align}
%
pressure, average energy and average particle number respectively.

\subsection{Phases and phase transitions}

The phase of a system is linked to the characteristic behaviour of one or more
of its physical quantities, such as the magnetisation of spin glasses, the
average positions of atoms in a crystal and the molecular composition of
solutions. We refer to this defining physical physical quantity as the
\emph{order parameter} of the phase.  The transition between two or more phases
are known as \emph{phase transitions}, and due to the very nature of phases,
happens through the induction of mathematical singularities. 

The order of a phase transition is exactly described by which derivative of the
free energy with respect to the order parameter of choice diverge. First order
transitions have discontinuous first derivatives, while only higher order
derivatives diverge for second order transitions.

However, if the order parameter tends towards zero, but never quite reaches it,
one can define a pseudo-phase transition, known as a \emph{crossover}, to be the
point of the most rapid change. While this is not a real phase transition,
following its trajectory in a phase diagram will result in one, if a transition
exists, linking the two concepts.

Phase transitions is an inherently macroscopic concept, and does not know about
the microscopic details of the theory. Theories with different microscopic
details might therefore still behave alike on the macroscopic level. These
categorise into \emph{universality classes}, catalogued by the singular
behaviour of their physical quantities close to the transition. If we denote the
order parameter by $m$, and the ordering field by $h$\footnote{a characteristic
variable which define the transition, $m\to0$ as $h\to0$}, we can define some of
the critical exponents by
%
\begin{align}
  m \;&\sim (T - T_c)^{\beta}, \\
  \bigg(\frac{\partial m}{\partial h}\bigg)_{\mathrlap{T}} &\sim (T - T_c)^{-\gamma}, \\
  C_V = - \beta^2 &\bigg(\frac{\partial^2 \mathcal{F}}{\partial
    \beta^2}\bigg)_{\mathrlap{V}} \sim (T - T_c)^{-\alpha}.
\end{align}
%
Systems which have the same critical exponents fall into the same universality
class. A notable example is the $4D$ Ising model whose critical exponents are
$\alpha = 0$, $\beta = 1/2$, and $\gamma = 1$.

\subsection{Yang Lee zeros}

An alternative approach to a rigorous study of the phase transitions was
proposed by Yang and Lee \citep{Yang:1952be,Lee:1952ig}. They suggested that one
can analyse the properties of the thermodynamic functions around a transition by
studying the zeros of the grand canonical partition function in the complex
fugacity-plane.

Seeing as the partition function can be interpreted as the normalisation of a
statistical probability function, it can never be zero for real and positive
values of $z$. However, as one increases the volume of the system, the number of
zeros increases, and tends to a continuous curve in the thermodynamic limit,
$V\to\infty$. As this curve forms, zeros on the positive and negative imaginary
axis could tend towards coinciding at the real axis, signalling the onset of a
phase transition.

\section{Thermal field theory} \label{sec:thermal-field-theory}

The quantum partition function is defined similar to the classical one, but with
the energy and particle numbers promoted to operators
%
\begin{equation}
  \mathcal{Z}_{\text{GC}} = \sum_{\mathclap{[\phi_i, \pi_i]}}
  \big\langle [\phi_i, \pi_i] \big|
    z^{\hat{N}} e^{-\beta \hat{H}} \big| [\phi_i, \pi_i] \big\rangle \,.
\end{equation}
%
This sum reduces to a form similar to \meqref{eq:partition} after introducing
the second quantisation and evaluating the integrals over the conjugate momentum
fields $\pi_i$.

% TODO is the interval of t_E really half open?

Focussing on the canonical ensemble, there are two amendments which need to be
made to the Euclidean action in \meqref{eq:action-def}. First the Euclidean time
$\tau_E$ have to be integrated over the half open interval $[0, \beta)$
%
\begin{equation}
  \mathcal{S}_E[\phi_i] = \int_{\mathrlap{0}}^{\mathrlap{\beta}} \mathrm{d} \tau_E \int \mathrm{d}^3 x \,
    \mathcal{L}_E\big[\phi_i(x)\big].
\end{equation}
%
The second rectification is a bit more subtle and concerns the periodic boundary
conditions of the fields. For the fields to respect their statistical properties
(Bose- vs Fermi statistics) they need to obey the boundary conditions
%
\begin{equation} \label{eq:boundary-conditions}
  \phi(\tau_E + \beta) = \begin{cases}
    \hphantom{\minus}\phi(\tau_E), \hskip1em \text{for bosonic fields}, \\
    \minus\phi(\tau_E), \hskip1em \text{for fermionic fields}.
  \end{cases}
\end{equation}

We can now return to the grand canonical ensemble and the particle number
operator $\hat{N}$. A suitable operator can be created from the conserved
Noether charges, such as the one from \meqref{eq:noether-charge}. For every
distinct conserved charge one wishes to study the dynamics of, there should be a
separate chemical potential. Two common systems of interest are those of quark-
and isospin chemical potential. Quark chemical potential is connected to the
global U$(1)$ symmetry of the fermion fields. The associated Noether charge is
that of total fermion number
%
\begin{equation} \label{eq:quark-number-density-op}
  N_f = \int \mathrm{d}^3 x \, \bar{\psi}_f(x) \gamma^0 \psi_f(x),
\end{equation}
%
with either separate or degenerate $\mu_f$ for each of the fermion species. The
isospin charge is related to the Cartan generator of the flavour mixing SU$(2)$
symmetry of a two flavour system (the third Pauli matrix)
%
\begin{equation}
  N_I  = \int \mathrm{d}^3 x \, \bar{\psi}_i(x) \gamma^0 (\sigma^3)_{ij}
    \psi_j(x),
\end{equation}
%
which for QCD gives the up and down quarks chemical potentials that only differ
in sign. By tuning the value of $\mu_I$, one can study the effects of a $u/d$
asymmetry, while adjusting $\mu_f$ induces a particle/anti particle imbalance.

\section{Thermal fields on the lattice} \label{sec:thermal-lattice-theory}

Following the same path as in \secref{sec:lattice_intro}, discretising the
expression for the partition function $\mathcal{Z}$ is straightforward. We see
that lattice theory naturally describe systems at non zero temperature, as the
temporal extent always will be finite. The temperature is given in terms of the
number of temporal lattice sites, $N_t$,
%
\begin{equation}
  T = \frac{1}{a N_t}.
\end{equation}
%
Therefore, when analysing continuum physics it is important to take a proper
infinite volume limit for spatial coordinates, while for the Euclidean temporal
extent one sends $a\to0$ and $N_t\to\infty$ in such a way that $a N_t$ remains
finite.

At a first glance it might seem as if when integrating lattice theories we can
only vary the temperature in finite increments, seeing as $N_t$ has to be an integer.
It is however much more common to keep $N_t$ fixed while varying $a$, which one
varies by tuning the gauge coupling $g$, as we saw in \secref{sec:scale_setting}.

Other than imposing the proper boundary conditions for the variables, no other
amendments need to be made for the finite temperature simulations. These follow
directly from \meqref{eq:boundary-conditions}
%
\begin{align}
  \psi(n_t + N_t) &= \minus \psi(n_t), \\
  U_{\mu}(n_t + N_t) &= \hphantom{\minus} U_{\mu}(n_t).
\end{align}

\subsection{Thermodynamic quantities}

In \secref{sec:stat-mech} we stated the thermodynamic quantities $\mathcal{P}$,
$\mathcal{E}$ and $\mathcal{N}$ in terms of derivatives of the partition
function. However one cannot sample the partition function through Monte Carlo
simulations, and we therefore have to device a different scheme to calculate
these quantities. By interchanging the order of the derivatives and the
integrals we get 
%
\begin{equation}
  \mathcal{E} = \minus\frac{\partial}{\partial \beta} \log \mathcal{Z}
    = \frac{1}{\mathcal{Z}} \int \prod_i \mathrm{d} \phi_i \, 
    \bigg( \frac{\partial\mathcal{S}}{\partial \beta} \bigg) e^{-\mathcal{S}}
    \equiv \bigg\langle \frac{\partial\mathcal{S}}{\partial \beta} \bigg\rangle
    \,.
\end{equation}
%
However, there is still a problem in defining the derivative with respect to
$\beta$. As we wish to keep $N_t$ fixed, we have to vary $\beta$ by varying $a$,
but we only want to vary the lattice spacing in the temporal direction, not the
spatial ones. By naively varying $a$, we would not only change the temperature,
but also the volume, which is supposed to stay fixed in the definition of
$\mathcal{E}$.

To rectify this we have to introduce an anisotropic lattice, one in which the
spatial and temporal lattice spacings are different. The magnitude of this
difference is encoded in the anisotropy parameter $\xi = a_s / a_t$. By varying
$a_s$ and $a_t$ separately we can compute derivatives with respect to
temperature and volume independently, and calculate the thermodynamic quantities
%
\begin{alignat}{99}
  \mathcal{P} &{}={}& \centermathcell{\minus\frac{a_s}{3 \beta V}} \bigg\langle &
    \centermathcell{\frac{\partial \mathcal{S}}{\partial a_s}} &\bigg\rangle, \\
  \mathcal{E} &{}={}& \centermathcell{\frac{a_t}{\beta}} \bigg\langle &
    \centermathcell{\frac{\partial \mathcal{S}}{\partial a_t}} &\bigg\rangle, \\
  \mathcal{N} &{}={}& \centermathcell{\minus{}z} \bigg\langle & 
    \centermathcell{\frac{\partial \mathcal{S}}{\partial z}} &\bigg\rangle.
\end{alignat}
%
Now that they are expressed in terms of ensemble averages we can use the
previously introduced Monte Carlo methods to estimate their values.

\section{Finite density simulations} \label{sec:finite-density-lattice}

To carry out simulations of systems with non zero quark number density we saw
that we have to introduce a quark chemical potential. Unfortunately, a naive
discretisation of the quark number operator from
\meqref{eq:quark-number-density-op} leads to divergences in the thermodynamic
quantities in the continuum limit \citep{Hasenfratz:1983ba,Kogut:1983ia}. This
is due to the lack of an implicit gauge invariance of the lattice action needed
for the right continuum cancellations to take place.

The correct way to introduce quark chemical potential to the lattice action is
by weighing the appropriate gauge links with the correct single step fugacity
%
\begin{equation}
  U_0(x) \to e^{a\mu} U_0(x), \hskip1em U_0^{\dagger}(x) \to e^{-a\mu}
    U_0^{\dagger}(x).
\end{equation}
%
Adding this change to the Wilson fermion matrix from \meqref{eq:wilson_action},
we get
%
\begin{multline}
  Q_{yx} \hskip1ex=\hskip1ex \big(m + {\textstyle\frac{4 r}{a}}\big) \delta_{y,x}
  \hskip1ex-\hskip1ex \frac{1}{2a} \sum_{\mu=\pm 1}^{\pm 3} (r + \gamma^{\mu}) U_{\mu}(x) \delta_{y- \hat{\mu},x} \\
   \hskip1ex - \frac{1}{2a} \big(\, e^{a\mu} (r+\gamma^0) U_0(x) \delta_{y-\hat{0},x}
   + e^{-a\mu}(r-\gamma^0) U_0^{\dagger}(x-\hat{0}) \delta_{y+\hat{0},x} \big)\,,
\end{multline}
%
which is also the final form of the lattice action we will study in this thesis.

\subsection{Dense systems and lattice saturation}
\label{sec:saturation}

Lattice studies of systems at finite chemical potential is limited in the very
dense limit by lattice saturation. This is a mere lattice artefact arising from
the Pauli exclusion principle, which tells us that two fermion degrees of
freedom cannot occupy the same quantum state. Due to the discretised nature
of it, a lattice cannot accommodate more than $2 N_f N_c N_t N_s^3$ fermions.

\begin{figure}[t]
  \begin{center}
    \includegraphics[width=.7\textwidth]{saturation_sketch}
  \end{center} \vskip -.4cm
  \caption{Sketch of the average number of quarks per lattice site as a function of baryon
  chemical potential on the lattice, $N_f = 2, N_c = 3$. Starts in the silver blaze state, moves
  into onset, reaches half filling then saturates at $2 N_f N_c$.}
  \label{fig:saturation_sketch}
\end{figure}

This limitation is sketched in \figref{fig:saturation_sketch} where a typical
lattice curve has been plotted. At small and intermediate values of the chemical
potential, we see the \emph{Silver Blaze} property. Coined by
\cite{Cohen:2003kd}, is the phenomena that at any chemical potentials smaller
than the smallest mass of the theory, the eigenvalues of the Dirac operator are
$\mu$ independent. Slightly before $\mu = m$ er have onset, which signals the
beginning of the nuclear condensation phase change. After which the system will
eventually reach half-filling, where half of the states on the lattice are
occupied. One should generally ignore results beyond this point as lattice
artefacts become dominant. This was the study of \citep{Rindlisbacher:2015pea},
in which the authors discovered a particle-hole symmetry around half-filling,
similar to electron-hole symmetries in solid state systems. As the lattice
continues to fill up we eventually reach saturation, at which point the lattice
states are all occupied, dynamics stop, and no more quarks can be added to the
system. As we see in the sketch, this happens when every lattice site is
populated by $2 N_f N_c$ fermions.


\section{The sign problem} \label{sec:sign-problem}

An issue arises when trying to numerically evaluate the partition function
integrals through Monte Carlo methods. We saw that we use the probability
density
%
\begin{equation}
  P\big([U_{\mu}]\big) \propto \det Q [U_{\mu}] \, e^{-\mathcal{S}_g[U_{\mu}]}
\end{equation}
%
to evaluate the integrals numerically. This requires that the fermion
determinant is positive definite, something we guaranteed by the fact that the
matrix was $\gamma^5$-hermitian. Unfortunately, this property is spoiled when a
chemical potential is introduced. When digging deeper we find that this
conceptual issue is just the tip of the iceberg, and that there is a more
fundamental issue at play.

The partition function has to be a real positive quantity, and therefore we know
that if we carry out the integral properly, that all the imaginary contributions
have to cancel out in the end. It is getting these exact cancellations out of
inexact numerical methods that is the underlying complication. This is what is
referred to as the \emph{sign problem}, and the problem is not exclusive to
lattice studies of finite density fermions. Being categorised as an NP-hard
problem it is unlikely that we will ever find a generic solution to the problem,
and it needs to be tackled in an intelligent way through deep understanding of
the source of the issue itself.

The problem is most easily illustrated with a sketch. In
\figref{fig:sign-problem} we have plotted the real and imaginary part of a
symmetric function with a complex phase, $f(x) e^{i \theta(x)}$. The real part
integrates to something finite that is exponentially suppressed by by the phase,
while the imaginary part integrates to zero. However, this subtlety is hard to
reproduce numerically, just as it is hard to see with the bare eye that one of
the two figures should give zero, while the other should give something finite.
The number of operations needed to keep the errors down grow exponentially with
the system size, making continuum studies impossible.
%
\begin{figure}
  \begin{center}
    \begin{adjustbox}{max width=\textwidth}
      \includegraphics[scale=2]{sign_problem_real}
      \includegraphics[scale=2]{sign_problem_imag}
    \end{adjustbox}
  \end{center} \vskip -.5cm
  \caption{The essence of the sign problem. Real and imaginary part of a
    function with a phase, $f(x) e^{i \theta(x)}$.}
  \label{fig:sign-problem}
\end{figure}

% TODO Add citations for the various methods

There are multiple workarounds devised to allow for evaluating finite density
systems using Monte Carlo methods. Such methods include reweighting, Taylor
expansions, analytic continuation from imaginary chemical potential and  the
method of stochastic quantisation. Although none of these methods actually
tackle the essence of the sign problem, with sufficiently powerful computers 
results with reasonable accuracy can be obtained. However one should take great
care to sample correctly if one wants sensible results, as was demonstrated in
\citep{Osborn:2008eg}.

For the remainder of this chapter we will cover the basics ideas of these
methods, and refer to \citep{deForcrand:2010ys} for a more thorough review.

\subsection{Reweighting}

Reweighting is a refactoring scheme used to great effect, where one takes the
phase of the fermion determinant as part of the observation rather than the
probability weight. Factoring out the phase
%
\begin{equation}
  \det Q = |\det Q| \, e^{i\theta}
\end{equation}
%
one can rewrite the expectation value of an observable as
%
\begin{align}
  \langle O \rangle  &=
  \frac{1}{\mathcal{Z}}\int \prod_i \mathrm{d} \phi_i \, O \, | \det
    Q | \, e^{i \theta} e^{-\mathcal{S}_g} \nonumber\\
  & \hskip1em = \frac{\mathcal{Z}_{\text{pq}}}{\mathcal{Z}_{\text{pq}}}
  \frac{\int \prod_i \mathrm{d} \phi_i \, O \, | \det
    Q | \, e^{i \theta} e^{-\mathcal{S}_g}}{%
  \int \prod_i \mathrm{d} \phi_i \, | \det
    Q | \, e^{i \theta} e^{-\mathcal{S}_g}}
  = \frac{\langle O \, e^{i\theta} \rangle_{\text{pq}}}{\langle e^{i\theta} \rangle_{\text{pq}}} .
\end{align}
%
The \emph{phase quenched} ensembles, where the phase is kept constant, only have
real weights, and one can therefore utilise Monte Carlo techniques. It is
possible to introduce a positive function of the phase, $f(\theta)$, to lessen
the difference between the true ensemble and the phase quenched one
%
\begin{equation}
  \langle O \rangle
  = \frac{\langle O \, e^{i\theta} / f(\theta) \rangle_f}{%
    \langle e^{i\theta} / f(\theta) \rangle_f} .
\end{equation}
%
This was explored in \citep{deForcrand:2002pa}, and an optimal choice for $f$
was determined to be $f(\theta) = |\cos\theta|$.

The sign problem comes into play when the average phase drastically deviates
from $0$, in which case the ensemble we sample is far from the ensemble we
want to measure, drastically increasing the required statistics. The issue is
sketched in \figref{fig:reweighting-overlap} where we see the overlap between
the full and reweighted ensembles. Because the reweighted ensemble deviates
significantly from the full ensemble at high chemical potentials, most of the
sampled configurations will be discarded, often referred to as the \emph{overlap
  problem}.

\begin{figure}
  \begin{center}
    \begin{adjustbox}{max width=\textwidth}
      \includegraphics[scale=2]{reweighting_ensemble_overlap_low_mu}
      \includegraphics[scale=2]{reweighting_ensemble_overlap_high_mu}
    \end{adjustbox}
  \end{center} \vskip -.5cm
  \caption{Overlap of the full ensemble to the reweighted ensemble at low and
    high chemical potential.}
  \label{fig:reweighting-overlap}
\end{figure}

\subsection{Imaginary chemical potentials}

% TODO Introduce the Columbia plot

Another method that has seen much attention recently is simulating at purely
imaginary chemical potentials, and then analytically continue from the negative
$\mu^2$ parameter region to the positive. At purely imaginary $\mu$ the measure
becomes a real, positive quantity, which can be seen from
%
\begin{equation}
  \det Q^{\dagger} (\mu) = \det Q(-\mu^*).
\end{equation}
%
This has been used to study the critical surfaces of the Columbia plot from
which one can see a trend as one cross into the region of real chemical
potential.

On top of being a useful method of extrapolating to real chemical potentials,
one can also study the interesting phase structure in the region of imaginary
chemical potentials. Because the imaginary chemical potentials introduce a pure
phase, the partition function is periodic under centre transformations of the
group. This is known as the Roberge-Weiss symmetry, and the periodicity leads to
a phase transition of the same name \citep{Roberge:1986mm}.
This has been extensively studied in
\citep{Wu:2013bfa,Cuteri:2015qkq,Philipsen:2016hkv}.

\subsection{Taylor series}

A similar approach to analytic extrapolation is to calculate the Taylor series
of an observable and then change the order of integration and derivation,
leading to the expansion
%
\begin{equation}
  \langle O \rangle_{\mu} = \langle O \rangle_{0}
   + \bigg\langle \frac{\mathrm{d} O}{\mathrm{d} \mu} \bigg\rangle_{\mathclap{0}} \mu
   + \frac{1}{2} \bigg\langle \frac{\mathrm{d}^2 O}{\mathrm{d} \mu^2} \bigg\rangle_{\mathclap{0}} \mu^2
   + \mathcal{O}(\mu^3).
\end{equation}
%
This has been used to try and follow the trajectory of the phase transition into
the region of finite $\mu$. The convergence region is however governed by the
partition function's behaviour at imaginary chemical potential, and the
Roberge-Weiss transition is to our knowledge the bound for any extrapolation
from zero chemical potential. There is therefore a strict limit to how far this
expansion is valid, which was the study of \citep{Osborn:2008eg}. [+citation?]

\subsection{Stochastic quantisation}

The final method we will cover is the use of stochastic quantisation. This
approach is different from the others as it doesn't use Monte Carlo methods at
all, rather, one uses a different stochastic process to estimate the partition
function integral, the Langevin equations \citep{Parisi:1980ys}. One introduces
an additional continuous parameter of the fields, $\alpha$, and then evolve the
system in this new parameter following the stochastic differential equation
%
\begin{equation}
  \frac{\partial \phi_i (x, \alpha)}{\partial \alpha} = - \frac{\delta
    \mathcal{S}}{\delta \phi} + \eta(x,\alpha)
\end{equation}
%
where $\eta$ is a Gaussian white noise. The method then relies on the fact that
in the $\alpha \to \infty$ limit, expectation value of observables with respect
to the $\eta$ distribution is identical to the true expectation value
%
\begin{equation}
  \lim_{\alpha_i\to\infty} \big\langle O [\phi(x,\alpha)]_i \big\rangle_{\eta}
    = \big\langle O [\phi(x)]_i \big\rangle,
\end{equation}
%
where the noise average is defined as
%
\begin{equation}
  \big\langle O \big\rangle_{\eta} =
  \frac{\int \prod_i \mathrm{d} \eta_i \,O\, e^{-\frac{1}{4}\int \mathrm{d}^4
      x\, \eta(x)^2}}{\int \prod_i \mathrm{d} \eta_i \, e^{-\frac{1}{4}\int
      \mathrm{d}^4 x\, \eta(x)^2}}\;.
\end{equation}
%
This equality can be rigorously proven for real actions
\citep{Damgaard:1987rr,Huffel:2003hf}, however one can nevertheless apply the
method to complex actions. Although using a complex action induces imaginary
contributions to the noise average, these are expected to average out in the
limit for physical observables. This is not always the case as the complex
Langevin sometimes diverges towards unphysical solutions \citep{Ambjorn:1985iw},
the reason for which is still an open question.

The sign problem itself manifests itself in the complex flow of the fields. For
large values of the chemical potential, the field configurations will spend a
considerable amount of time in the complex phase, resulting in the need for
longer flow times before the imaginary contributions average out. This can to
some extent be remedied by the introduction of gauge cooling
\citep{Seiler:2012wz}, which are invariant transformations on the variables that
push them closer to the real field values. For QCD it has been shown that this
this places a restriction of the coupling strengths for which the method is
applicable. Recently a comparison between the reach of reweighting techniques to
those of stochastic quantisation, showing comparable results
\citep{Fodor:2015doa}. Newest developments can be found in
\citep{Aarts:2016qrv}.

\begin{figure}
  \begin{center}
    \begin{adjustbox}{max width=\textwidth}
      \includegraphics{baryon_muscan}
      \includegraphics{chiral_mfscan}
    \end{adjustbox}
  \end{center} \vskip -.5cm
  \caption{Left: Baryon number density of the Stephanov model measured by
    complex langevin compared to full and phase quenched analytical results at
    fixed $m = 0.0$.  Right: Same comparison for the chiral condensate at fixed
    $\mu=1.0$.}
  \label{fig:rmt_cl}
\end{figure}

To demonstrate a case in which we can clearly see that the Langevin flow
converging to the wrong theory, we can study a \emph{Random Matrix Theory}
(RMT), which is a simpler model that also suffers from a strong sign problem.
One such RMT is the Stephanov model \citep{Stephanov:1996ki}, whose partition
function is
%
\begin{equation}
  \mathcal{Z}_{\mathrm{S}} = \int [\mathrm{d} W] e^{-N \Sigma^2 \tr W^{\dagger} W}
    \det{}^{N_f} 
      \begin{pmatrix}
        M & i W + \mu \\
        i W^{\dagger} + \mu & M
      \end{pmatrix} \,,
\end{equation}
%
where the degrees of freedom are random matrices $W \in M_{\mathbb{C}}(N,N)$.
The mass matrix $M$, chemical potential matrix $\mu$, and flavour number $N_f$,
are introduced to mimic the parameters of QCD. This model has the advantage that
it is analytically solvable both at finite $\mu$, as well as in the phase
quenched \citep{Stephanov:1996ki,Halasz:1997he}. The results from simulating
this theory using the unimproved complex langevin algorithm is shown in
\figref{fig:rmt_cl}. In the same plots the full analytic result is plotted in
red, and the phase quenched result in green. It should be noted that the phase
quenched is only accessible in the thermodynamic limit, $N \to \infty$, however
$48$ is close enough to this limit that we expect few discrepancies from finite
volume effects. We see that the complex langevin results clearly converges to
the phase quenched theory even though the full theory is used as input. Further
analysis is yet to be made regarding improvement of this case. This includes the
aforementioned gauge cooling method, which was studied in
\citep{Nagata:2016alq,Nagata:2016vkn} for a different RMT, namely
the Osborn model, and proved to be effective with the right choice for the
required \emph{cooling norm}. Another alternative comes from the theory of
Lefschetz thimbles, where one can use the thimbles to weigh the probability
distributions at the stochastic step \citep{DiRenzo:2015foa}.

