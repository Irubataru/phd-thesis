\chapter{Summary and discussion}

In this thesis we have presented a significant improvement to an ongoing study
of an effective theory approach to cold and dense lattice QCD for heavy quarks.
This is a particularly interesting region of study as traditional first
principle approaches fail in this domain. The effective theory approaches this
attractive parameter region from a theory that describes strongly interacting
heavy quarks on a lattice. By the use of various expansion techniques we have
shown that we can systematically move closer into the region of cold and dense
continuum QCD.

We opened this thesis with an introduction to both continuum, and lattice gauge
theories in \chapref{chap2}. Emphasis was put on symmetry properties of the
systems we study, and how these influence large scale dynamics. In
\chapref{chap3} we extended vacuum gauge theories to the realm of thermodynamics
and statistical mechanics, and saw how these systems can be put on a lattice.
Application of lattice methods to dense systems was stressed, and a longer
analysis of the difficulty of, and possible remedies to, simulating finite chemical
potential physics was given.

\hyperref[chap4]{\mbox{\textsc{Chapter} \ref*{chap4}}} was dedicated to a
detailed summary of the derivation of the effective theory. The two expansion
schemes, namely the character expansion around $\beta = 0$, $g \to \infty$, and
the hopping parameter expansion $\kappa = 0$, $m_q \to \infty$, was introduced
one by one, and their effects on the effective theory discussed. Two important
resummation schemes were introduced, an exponential resummation, which improve
thermodynamic studies, as well as a logarithmic one, for further analytic
evaluation. Then the cold and dense regime were introduced in which the
existence of heavy temporal quark lines are favoured over anti-quark lines, and
the combinatorics of the computation of the expansion coefficients simplify.
Finally the full $\mathcal{O}(\kappa^8 u^5 N_t^4)$ effective theory action was
given, and a short discussion on its numerical prowess was presented.

In \chapref{chap5} the main objective of the present work was discussed,
specifically the fully analytical treatment of the effective theory. We started
by introducing the renowned linked cluster expansion for thermodynamic systems,
and then showed how one could map the lowest order effective theory,
$\mathcal{Z}_2$, onto this approach. This was unfortunately not sufficient for a
complete study of the higher order effective theory term, and thus a more
general, polymer linked cluster expansion was developed. We used this new
systematic approach to map the complete $\mathcal{O}(\kappa^8 u^5 N_t^4)$
effective theory onto this framework, and was therefore able to present a
systematic, and analytic study of the higher order contributions to the cold and
dense limit of heavy QCD.

In the subsequent section we utilised the full power of the analytic approach
and discussed three potent resummation schemes inaccessible to the numerics.
First was the chain resummation, which made use of recursive integrability of a
specific combination of terms to predict an infinite chain of quark
interactions. This specifically adds interactions at arbitrary distances to the
effective theory, improving its ability to describe the correct thermodynamics
for macroscopic systems. Another resummation on the cluster expansion lever in
the same spirit was also introduced, which helps increase the accuracy of the
expansion as compared to numerical methods. This is only a fraction of the full
LCE resummation machinery, and can therefore be expanded in the future to
include additional long range effects, normally off limits to series expansion
computations. Finally a hypothetical ladder resummation was presented, which
would work as a nearest neighbour coupling renormalisation. The results were
then re-examined in the fully resummed analytic computation. The equation of
state in the continuum was shown, which had the same scaling as the EoS for
non-relativistic weakly interacting fermions. The effective theory was
dynamically able to extract the baryonic degrees of freedom, showing the
properties of confinement from a strong coupling, heavy quark foundation. A
further analysis of the degrees of freedom predicted by this equation of state
shows us that there are still questions left to explore.

Finally a short overview of further applications of the effective theory was
presented. This include the extension to higher $N_c$ gauge groups, which is of
great interest to the supersymmetry community. Also the study of analytic Yang
Lee zeros, and its advantages and challenges in the quest to investigating
non-analyticities and phase transitions.

We have seen in this thesis that the effective theory approach, the spatial
hopping expansion, has a rich structure and that there are still undiscovered
and unexplored future research subjects, which we will discuss next.
