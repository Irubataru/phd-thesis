\chapter{Summary and discussion}

The effective effective theory approach to to the study of cold and dense QCD
manages to probe a region of the phase diagram in which traditional first
principle methods fail. The method is based on expanding a system of strongly
interacting, heavy quark around these two limits. By the use of various
expansion techniques, the effective theory systematically approach the continuum
physics of cold and dense, heavy QCD.

This thesis's main objective has been the continued improvement of the effective
theory. We started out by introducing both continuum, and lattice gauge theories
in \chapref{chap2}. We focussed on the symmetry properties of the systems we
study, and how these influence large scale dynamics. In \chapref{chap3} we
extended vacuum gauge theories to the realm of thermodynamics and statistical
mechanics, and saw how these systems can be put on a lattice.  Application of
lattice methods to dense systems was stressed, and a longer analysis of the
difficulty of, and possible remedies to, simulating finite chemical potential
physics was given.

\hyperref[chap4]{\mbox{\textsc{Chapter} \ref*{chap4}}} was dedicated to a
detailed summary of the derivation of the effective theory. The two expansion
schemes, namely the character expansion around $\beta = 0$, $g \to \infty$, and
the hopping parameter expansion $\kappa = 0$, $m_q \to \infty$, were introduced,
and their effects on the effective theory discussed. Two important resummation
schemes were introduced, an exponential resummation, which improve thermodynamic
studies, as well as a logarithmic one, for further analytic evaluation. Then the
cold and dense regime, in which the existence of heavy temporal quark lines are
favoured over anti-quark lines were introduced, and the combinatorics of the
computation of the expansion coefficients simplified. Finally, the full
$\mathcal{O}(\kappa^8 u^5 N_t^4)$ effective theory action was derived, followed
by a short discussion on its numerical prowess.

In \chapref{chap5} the main objective of the present work was discussed,
specifically the fully analytical treatment of the effective theory. We started
by introducing the renowned linked cluster expansion for thermodynamic systems,
and then showed how one can map the lowest order effective theory,
$\mathcal{Z}_2$, onto this approach. This was unfortunately not sufficient for a
complete study of the higher order effective theory term, and thus a more
general, polymer linked cluster expansion, was developed. We used this new
systematic approach to map the complete $\mathcal{O}(\kappa^8 u^5 N_t^4)$
effective theory onto this framework, and were consequently able to present an
analytic study of the higher order contributions to the cold and dense limit of
heavy QCD.

In the subsequent section we employed the full power of the analytic approach
and discussed three potent resummation schemes inaccessible to numerical
methods.  First up was the chain resummation, which makes use of recursive
integrability of a specific combination of terms to predict an infinite chain of
quark interactions. This specifically adds interactions at arbitrary distances
to the effective theory, improving its ability to describe the correct
thermodynamics for macroscopic systems. In the same spirit we introduced another
resummation on the cluster expansion level , which helps increase the accuracy
of the expansion as compared to numerical methods. This is only a fraction of
the full LCE resummation machinery, and can therefore be extended in the future
to include additional long range effects, normally off limits to series
expansion computations. Finally, a hypothetical ladder resummation which would
work as a nearest neighbour coupling renormalisation was presented. The results
were then re-examined in the fully resummed analytic computation. We showed the
equation of state in the continuum, where it has the same scaling as the EoS for
non-relativistic weakly interacting fermions. The effective theory dynamically
extracts the baryonic degrees of freedom, showing the properties of confinement
from a strong coupling, heavy quark foundation. A further analysis of the
degrees of freedom predicted by this equation of state shows us that there are
still questions left to explore.

Finally, a short overview of further applications of the effective theory was
presented. This included the extension to higher $N_c$ gauge groups, which is of
great interest to the supersymmetry community, and the evaluation of the
analytic Yang Lee zeros. The latter of which is important to the study of
non-analyticities and phase transitions.

The present work demonstrated that the effective theory approach, the spatial
hopping expansion, has a rich structure and that there are still undiscovered
and unexplored future research topics, which we will discuss next.
