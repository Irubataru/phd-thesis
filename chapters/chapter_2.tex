\chapter{Gauge Theories and Lattice Gauge Theories}

In this chapter we will formally introduce the quantum field theory describing
fermions invariant under local group transformations and introduce the Yang
Mills Lagrangian. This is covered in \secref{sec:continuum_gauge}. We follow
this up in \secref{sec:group_intro} with a brief overview of basic group theory,
introducing the concepts and tools necessary for the thesis. In
\secref{sec:symmetries} we investigate the defining symmetry properties of our
quantum field theory. After this introduction to continuum physics, in
\secref{sec:lattice_intro} we discretise space-time and put our theory on a
lattice. In the \secrefs{sec:lattice_symmetries,sec:fermion_doubling,sec:scale_setting}
we review the similarities and differences between the lattice and the
continuum, and stress where additional care has to be taken.

For the introduction, standard texts on the subject matter has been consulted.
Such as introductory volumes on quantum field theory
\citep[e.g.][]{peskin1995introduction,maggiore2004modern}, lattice gauge theory
\citep[e.g.][]{montvay1997quantum,gattringer2009quantum} and group theory
\citep[e.g.][]{georgi1999lie,fulton2013representation}.

\section{The free Dirac Lagrangian} \label{sec:continuum_gauge}

The undoubtaly most important quantity in Quantum Field Theory (\emph{QFT}) and
Statistical Mechanics (\emph{SM}) is $\mathcal{Z}$, named the generating
functional and the partition function in the two theories respectively. It is in
SM defined as
%
\begin{equation}
  \mathcal{Z}_E = \int \prod_i \mathrm{d} \phi_i e^{-\mathcal{S}_E[\phi_i]},
\end{equation}
%
where we have represented the field parameters by $\phi_i$. There is an analogue
between the two theories and one can convert between them by Wick rotating the
time coordinate $x^0_E = -i x^0_M$. Whenever the possibility for confusion
exists we use an $M$ to symbolise Minkowski space, the world of QFT, and an $E$
to symbolise Euclidean space, the world of SM.

The action, $\mathcal{S}$, is in turn defined by the Lagrangian density
%
\begin{equation}
  \mathcal{S}[\phi_i] = \int \mathrm{d}^4 x\, \mathcal{L}[\phi_i(x)],
\end{equation}
%
and our theories are therefore uniquely defined through $\mathcal{L}$ and the
properties of the fields $\phi_i$.

The most basic starting point for us will be the Dirac Lagrangian, which reads
%
\begin{equation}
  \mathcal{L} = \bar{\psi}_i \big(i\gamma^{\mu} \partial_{\mu} - m_i\big)
  \psi_i. \label{eq:ldirac}
\end{equation}
%
where we have introduced the fermionic fields $\psi_i$. To get the right
statistics, these fields have to obey the anti-commutation relations, placing
them in the group known as Grassmann numbers
%
\begin{align}
  \big\{ \psi_i, \psi_j \big\} &\equiv \psi_i \psi_j + \psi_j \psi_i = 0,\\
  \big\{ \bar{\psi}_i, \bar{\psi}_j \big\} &\equiv \bar{\psi}_i \bar{\psi}_j +
    \bar{\psi}_j \bar{\psi}_i = 0,\\
  \big\{ \bar{\psi}_i, \psi_j \big\} &\equiv \bar{\psi}_i \psi_j + \psi_j
    \bar{\psi}_i = \delta_{ij}.
\end{align}

This Lagrangian describe the dynamics of a set of free, massive fermions, which
is not very exciting. There are a multitude of ways to add interactions while
still maintaining the desired symmetry properties (which we will discuss later).
One is to add a self interaction term, e.g. a four point term, $g
\big(\bar{\psi}\psi\big)^2$, which results in the NJL model
\citep{Nambu:1961tp}. Another option is to introduce more fields and
interactions. Finally one can localise the symmetries of the fields and induce
new fields and couplings. For this procedure we need a short overview of group
theory which we will tackle next.

\section{A group theory perspective} \label{sec:group_intro}

A group $G$ is a set and a rule for assigning to every (ordered) pair in the set
a new element also in the set
%
\begin{equation}
  \text{For every} \hskip1ex f, g \in G, \hskip1ex\text{then}\hskip1ex
    h = fg \in G
\end{equation}
%
this assignment is called the group product. The group product is associative,
and in every group a unit element exists ($e$). Every element also has an inverse so
that
%
\begin{equation}
  \forall\hskip.5ex f \in G, \hskip1ex\exists\hskip.5ex f^{-1}
    \hskip1ex:\hskip1ex f^{-1} f = f f^{-1} = e
\end{equation}
%
A representation of a group $G$ on a complex vector space $V$ is a homomorphism
%
\begin{equation}
  D_{R}:\hskip1em G \to \text{GL}(V)
\end{equation}
%
of the group $G$ on the group of automorphisms of $V$. Therefore representations
have to maintain the group structure, meaning
%
\begin{align}
  D_R(e) &= \mathbb{1}_R,\\
  D_R(f) D_R(g) &= D_R(fg). \label{eq:group_prod}
\end{align}
%
In physics we are normally content with a coordinate basis, in which the
representations map to $n$ by $n$ invertible matrices, i.e. $V := \mathbb{C}$,
$\text{GL}(V) := M_{\mathbb{C}}(n,n)$.

\subsection{Lie groups}

If the elements of a group $G$ depend smoothly on some set of parameters,
$g(\alpha_i) \in G$, we call that group a Lie group. Lie groups make up the
theory of continuous transformations, and are therefore of special interest to
physics. The continuous nature of the Lie groups create a sense of closeness
where two elements that are close to each other can be represented through
differentials. For example
%
\begin{equation}
  D_R(g(\delta\alpha_i)) = \mathbb{1}_R + i\, \delta \alpha_i T_{Ri} +
    \mathcal{O}(\delta \alpha^2).
\end{equation}
%
One commonly refers to the $T_{Ri}$ as the generators of the group $G$. Applying
the smoothness condition to the group product, \meqref{eq:group_prod},
we find that the commutator of the generators follow a special relation
%
\begin{equation}
  \big[ T_{Ri}, T_{Rj} \big] = i\, f_{ijk} T_{Rk}.
\end{equation}
%
The structure constants, $f_{ijk}$, are in fact independent of the
representation $R$. They are tremendously important as they summarise the entire
group multiplication law. They are also referred to as the \emph{algebra} of the
group, $\mathfrak{g}$. The generators and the group algebra will come into play
when we later introduce continuous symmetries of our quantum theory of fields.

\subsection{Character and character expansion}

One defines the character of a representation as
%
\begin{equation}
  \chi_R(g) = \tr D_R(g).
\end{equation}
%
The characters appear frequently in field theory due to the cyclic properties of
the trace and its ability to facilitate the creation of invariant objects. On
top of this the characters span a basis of functions that share this cyclic
properties, i.e. any function $\rho(f g f^{-1}) = \rho(g)$. Hence
%
\begin{equation}
  \rho(g) = \sum_r \rho_r \chi_r(g),
\end{equation}
%
with
\begin{equation}
  \rho_r = \int_{\mathrlap{g \in G}\hskip1.5ex} \mathrm{d}g \, \chi_r^*(g) \rho(g).
\end{equation}
%
It is normal to factor out both the trivial representation ($D_0(g) = 1$), as
well as the dimension of the representation, and write
%
\begin{equation}
  \rho(g) = \rho_0 \Big(1 + \sum_{r \neq 0} d_r \bar{\rho}_r \chi_r(g)\Big),
\end{equation}
%
which we will refer to as the character expansion. We will make use of this in
\secref{sec:char_exp}.

\subsection{Group integrals}

Finally we need to define the integrals over the continuous groups. We introduce
the Haar integration measure, which has the following properties
%
\begin{equation}
  \int \mathrm{d}g \, \rho(g) = \int \mathrm{d}g \, \rho(fg) 
    = \int \mathrm{d}g \, \rho(gf) \hskip2ex \forall \hskip1ex f \in G
\end{equation}
%
and the normalisation
%
\begin{equation}
  \int \mathrm{d}g = 1.
\end{equation}
%
During the derivation of the effective theory, we will encounter polynomial
integrals on the form
%
\begin{equation}
  I = \int \mathrm{d} g \, g^n (g^{-1})^m
\end{equation}
%
or more commonly, with a representation
%
\begin{equation}
  I_{i_1,j_1,...,i_n,j_n}^{k_1,l_1,...,k_m,l_m} = \int \mathrm{d} g \,
    D_R(g)_{i_1j_1} \cdots D_R(g)_{i_nj_n} D_R(g^{-1})_{k_1l_1} \cdots D_R(g^{-1})_{k_ml_m}.
\end{equation}
%
There are multiple ways of tackling these integrals depending on the Lie group
one is interested in. For SU($N$) one can for instance decompose the integral
into generalised Euler angles \citep[as described in][]{Tilma:2004kp}, construct
generating functionals for the integrals $I_{i_1,j_1,...,i_n,j_n}^{k_1,l_1,...,k_m,l_m}$
\citep[details in][]{Creutz:1978ub}, or one can use tensor product
decompositions to write the integrals in terms of Young projectors
\citep{Myers:2014dia}. In this work a combination of the first two approaches
has been used, and extensive details are provided in
\apxrefs{sec:haar_measure,sec:character_integrals,sec:sun_integrals}.

\section{Symmetries of the Lagrangian} \label{sec:symmetries}

With the necessary formalism in place, we can discuss the symmetries of the
Lagrangian we introduced in \meqref{eq:ldirac}. We will first look at the
symmetries that come by construction before delving into possible ways of
extending the theory to be both globally and locally symmetric.

\subsection{Symmetries by construction}

Every quantum field theory is constructed to be symmetric under the full
Poincar\'{e} group, namely the group of space-time translations plus the Lorentz
transformations. 

To see the other symmetries of the Dirac Lagrangian we have to decompose the
fields into its spinorial components. Every element of $\psi_i$ in
\meqref{eq:ldirac} is a $4$ component spinor, and the $\gamma$'s $4 \times 4$
matrices. In the standard representation they take for form
%
\begin{equation}
  \psi = \frac{1}{\sqrt{2}} \begin{pmatrix} \psi_R + \psi_L \\ \psi_R - \psi_L \end{pmatrix}, \hskip1cm
    \gamma^0 = \begin{pmatrix} 1 & 0 \\ 0 & -1 \end{pmatrix}, \hskip1ex
    \gamma^i = \begin{pmatrix} 0 & \sigma^i \\ -\sigma^i & 0 \end{pmatrix}.
\end{equation}
%
The fields $\psi_{R,L}$ are two component fields that transform under two
distinct representations of the Lorentz group, $(\frac{1}{2}, 0)$ and $(0,
\frac{1}{2})$ respectively. With this basis the Dirac Lagrangian reads
%
\begin{equation}
  \mathcal{L} = i \bar{\psi}_R \gamma^{\mu}\partial_{\mu} \psi_R
    + i \bar{\psi}_L \gamma^{\mu}\partial_{\mu} \psi_L
    - m (\bar{\psi}_R\psi_L + \bar{\psi}_L\psi_R).
\end{equation}
%
It is apparent that this Lagrangian is symmetric under the U($1$) $\otimes$ U($1$)
transformation
%
\begin{equation}
  \psi_{R,L} \to e^{i \alpha_{R,L}} \psi_{R,L}
\end{equation}
%
if and only if the mass term is set to zero. Invariance of the independent
transformations of the left- and right handed fields is called \emph{chiral}
symmetry, and signal the existence of helicity eigenstates. The mass term breaks
the symmetry into a single U($1$) symmetry, where $\alpha_R = \alpha_L$.

The global symmetry group of $\psi$ can be trivially extended to admit any Lie
group by making the field live in the vector space the representation acts on 
%
\begin{equation} \label{eq:global_trafo}
  \psi \to D_R(g) \psi, \hskip1em \bar{\psi} \to \bar{\psi} D_R^{\dagger}(g),
  \hskip1em g \in G.
\end{equation}
%
This symmetry has an associated conserved Noether current for every generator of
the group $G$
%
\begin{equation}
  j^{\mu}_i = \bar{\psi} \gamma^{\mu} T_i \psi, \hskip1ex\text{where}\hskip1ex
    \partial_{\mu}j^{\mu}_i = 0,
\end{equation}
%
which gives the associated charge
%
\begin{equation}
  Q_i = \int \mathrm{d}^3 x\, \bar{\psi} \gamma^0 T_i \psi.
\end{equation}

\subsection{Gauge symmetries}

We just saw that the Lagrangian can be trivially extended to be invariant under
global symmetry transformations. However, we also want to promote them to local
symmetries. This is motivated post facto by QED, with exhibits a local U($1$)
symmetry.

The idea was first explored by Weyl, who tried introducing local scale
transformations to the metric
%
\begin{equation}
  g_{\mu\nu}(x) \to e^{\sigma(x)}g_{\mu\nu}(x)
\end{equation}
%
which prompted the need for an additional vector field to gauge how much one
needs to adjust scales to compare two separate points. Weyl then showed that
these vector fields need to follow Maxwell's equations \citep{Weyl:1918ib}.
Although this ultimately is not how the universe works, it encourages the ideas
that there is a fundamental relation between local symmetries and the geometry
of space, also the name stuck. Instead of adjusting local scales we will apply
group transformations to a vector space that lives on top of space-time.

We define a local group element by a Lie group whose parameters are space-time
dependent $g(\alpha_i(x))$. We will use the short hand notation $U(x) =
D_R\big(g(\alpha_i(x))\big)$ for representations of such group elements. We
suppress the representation $R$ wherever it is not needed. The transformations
of the fields at a single coordinate follow trivially from
\meqref{eq:global_trafo}
%
\begin{equation} \label{eq:local_trafo}
  \psi(x) \to U(x) \psi(x), \hskip1em \bar{\psi}(x) \to \bar{\psi}(x)
  U^{\dagger}(x).
\end{equation}
%
There is however no way to make derivative terms invariant under these
symmetries. This becomes obvious when writing the derivative in terms of limits
%
\begin{equation}
  \eta^{\mu} \partial_{\mu} \psi(x) = \lim_{\delta \to 0} \frac{1}{\delta} \big(
    \psi(x + \delta\eta) - \psi(x) \big)
\end{equation}
%
where $\eta^{\mu}$ is some unit vector. We have to take the difference between
fields at two different points in space-time, which transform independently
under \meqref{eq:local_trafo}. It is completely nonsensical trying to calculate
this difference at all, and we therefore instead introduce a two-point parallel
transporter, $\Lambda(y,x)$, and define the \emph{covariant derivative} as
%
\begin{equation}
  \eta^{\mu} D_{\mu} \psi(x) = \lim_{\delta \to 0} \frac{1}{\delta} \big(
    \psi(x + \delta\eta) - \Lambda(x+\delta\eta,x)\psi(x) \big)
\end{equation}
%
where this scalar function transform as
%
\begin{equation}
  \Lambda(y,x) \to U(y) \Lambda(y,x) U^{\dagger}(x).
\end{equation}
%
This in turn make the total covariant derivative transform as
%
\begin{equation}
  D_{\mu}\psi(x) \to U(x) D_{\mu} \psi(x),
\end{equation}
%
ensuring that terms of the form $\bar{\psi} D_{\mu} \psi$ remain invariant under
local symmetry transformations. From its transformation properties, it is
possible to write $\Lambda$ in terms of the generators of the group. This is on
infinitesimal form
%
\begin{equation}
  \Lambda(x + \delta\eta,x) = 1 + i g \delta \eta^{\mu} A^i_{\mu}(x) T_i +
    \mathcal{O}(\delta^2)
\end{equation}
%
where the $A^i_{\mu}$ have to transform as
%
\begin{equation}
  A^i_{\mu} \to A^i_{\mu} + \partial_{\mu} \alpha^i - g f^{ijk} \alpha^j A^k_{\mu}
    + \mathcal{O}(\alpha^2)
\end{equation}
%
completely independent of the representation. The local symmetric Dirac
Lagrangian therefore reads
%
\begin{equation}
  \mathcal{L} = \bar{\psi}\big(i \gamma^{\mu} D_{\mu} - m\big) \psi.
\end{equation}
%
Unfortunately, without a kinetic term for the newly introduced fields
$A^i_{\mu}$, the theory would permit violently oscillating fields with no cost
in energy. This would make our theory non-renormalisiable, which we want to
avoid.  We therefore need to find a kinetic term for the fields that admits all
of our restrictions. To construct such a term we again turn to the field
transporters $\Lambda$. It is easy to see that the geometric plaquette could
create such a term
%
\begin{align}
  U_{\mu\nu}(x) =& \Lambda(x, x + \hat{\nu} \delta)
    \Lambda(x + \hat{\nu} \delta, x + \hat{\mu} \delta + \hat{\nu} \delta) \nonumber\\
    &\hskip1em\times\Lambda(x + \hat{\mu} \delta + \hat{\nu}\delta, x + \hat{\mu} \delta)
    \Lambda(x + \hat{\mu}\delta, x)
\end{align}
%
which transforms as
%
\begin{equation}
  U_{\mu\nu}(x) \to U(x) U_{\mu\nu}(x) U^{\dagger}(x)
\end{equation}
%
and therefore $\tr U_{\mu\nu}(x)$ is invariant under local group
transformations. Expanding $U_{\mu\nu}$ in the fields $A^i_{\mu}$, we get
%
\begin{equation}
  U_{\mu\nu} = 1 + i g \delta^2 T_i \big( \partial_{\mu} A^i_{\nu} - \partial_{\nu}
  A^i_{\mu} + g f^{ijk} A^j_{\mu} A^k_{\nu} \big) + \mathcal{O}(\delta^3)
\end{equation}
%
from which we extract the term
%
\begin{equation}
  F_{\mu\nu}^i = \partial_{\mu} A^i_{\nu} - \partial_{\nu}
    A^i_{\mu} + g f^{ijk} A^j_{\mu} A^k_{\nu}
\end{equation}
%
which is both invariant under the gauge transformations as well as being the
kinetic term we were looking for. The full gauge extended Lagrangian therefore
reads
\begin{equation}
  \mathcal{L} = \bar{\psi}\big(i \gamma^{\mu} D_{\mu} - m\big)\psi
    - \frac{1}{4} F^i_{\mu\nu} F^{i\hskip.2ex\mu\nu}
\end{equation}





\section{Lattice discretisation} \label{sec:lattice_intro}

\section{Symmetries on the lattice} \label{sec:lattice_symmetries}

\section{Fermion doubling and chiral symmetry} \label{sec:fermion_doubling}

\section{Scale setting and the continuum limit} \label{sec:scale_setting}

