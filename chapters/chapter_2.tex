\chapter{Gauge Theories and Lattice Gauge Theories}

In this chapter we will formally introduce the quantum field theory describing
fermions invariant under local group transformations and introduce the Yang
Mills Lagrangian. This is covered in \secref{sec:continuum_gauge}. We follow
this up in \secref{sec:group_intro} with a brief overview of basic group theory,
introducing the concepts and tools necessary for the thesis. In
\secref{sec:symmetries} we investigate the defining symmetry properties of our
quantum field theory. After this introduction to continuum physics, in
\secref{sec:lattice_intro} we discretise space-time and put our theory on a
lattice. In the \secrefs{sec:lattice_symmetries,sec:fermion_doubling,sec:scale_setting}
we review the similarities and differences between the lattice and the
continuum, and stress where additional care has to be taken.

For the introduction, standard texts on the subject matter has been consulted.
Such as introductory volumes on quantum field theory
\citep[e.g.][]{peskin1995introduction,maggiore2004modern}, lattice gauge theory
\citep[e.g.][]{montvay1997quantum,gattringer2009quantum} and group theory
\citep[e.g.][]{georgi1999lie,fulton2013representation}.

\section{Continuum gauge theories with fermions} \label{sec:continuum_gauge}

The undoubtaly most important quantity in Quantum Field Theory (\emph{QFT}) and
Statistical Mechanics (\emph{SM}) is $\mathcal{Z}$, named the generating
functional and the partition function in the two theories respectively. It is in
SM defined as
%
\begin{equation}
  \mathcal{Z}_E = \int \prod_i \mathrm{d} \phi_i e^{-\mathcal{S}_E[\phi_i]},
\end{equation}
%
where we have represented the field parameters by $\phi_i$. There is an analogue
between the two theories and one can convert between them by Wick rotating the
time coordinate $x^0_E = -i x^0_M$. Whenever the possibility for confusion
exists we use an $M$ to symbolise Minkowski space, the world of QFT, and an $E$
to symbolise Euclidean space, the world of SM.

The action, $\mathcal{S}$, is in turn defined by the Lagrangian density
%
\begin{equation}
  \mathcal{S}[\phi_i] = \int \mathrm{d}^4 x\, \mathcal{L}[\phi_i(x)],
\end{equation}
%
and our theories are therefore uniquely defined through $\mathcal{L}$ and the
properties of the fields $\phi_i$.

The most basic starting point for us will be the Dirac Lagrangian, which reads
%
\begin{equation}
  \mathcal{L} = \bar{\psi}_i \big(i\gamma^{\mu} \partial_{\mu} - m_i\big)
  \psi_i.
\end{equation}
%
where we have introduced the fermionic fields $\psi_i$. To get the right
statistics, these fields have to obey the anti-commutation relations, placing
them in the group known as Grassmann numbers
%
\begin{align}
  \big\{ \psi_i, \psi_j \big\} &\equiv \psi_i \psi_j + \psi_j \psi_i = 0,\\
  \big\{ \bar{\psi}_i, \bar{\psi}_j \big\} &\equiv \bar{\psi}_i \bar{\psi}_j +
    \bar{\psi}_j \bar{\psi}_i = 0,\\
  \big\{ \bar{\psi}_i, \psi_j \big\} &\equiv \bar{\psi}_i \psi_j + \psi_j
    \bar{\psi}_i = \delta_{ij}.
\end{align}


\section{Important ideas from group theories} \label{sec:group_intro}
\section{Symmetry groups and continuum symmetries} \label{sec:symmetries}
\section{Lattice discretisation} \label{sec:lattice_intro}
\section{Symmetries on the lattice} \label{sec:lattice_symmetries}
\section{Fermion doubling and chiral symmetry} \label{sec:fermion_doubling}
\section{Scale setting and the continuum limit} \label{sec:scale_setting}
