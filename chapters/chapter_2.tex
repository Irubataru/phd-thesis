\chapter{Gauge Theories and Lattice Gauge Theories}

In this chapter we will formally introduce the quantum field theory describing
fermions invariant under local group transformations and introduce the Yang
Mills Lagrangian. This is covered in \secref{sec:continuum_gauge}. We follow
this up in \secref{sec:group_intro} with a brief overview of basic group theory,
introducing the concepts and tools necessary for the thesis. In
\secref{sec:symmetries} we investigate the defining symmetry properties of our
quantum field theory. After this introduction to continuum physics, in
\secref{sec:lattice_intro} we discretise space-time and put our theory on a
lattice. In the \secrefs{sec:lattice_symmetries,sec:fermion_doubling,sec:scale_setting}
we review the similarities and differences between the lattice and the
continuum, and stress where additional care has to be taken.

For the introduction, standard texts on the subject matter has been consulted.
Such as introductory volumes on quantum field theory
\citep[e.g.][]{peskin1995introduction,maggiore2004modern}, lattice gauge theory
\citep[e.g.][]{montvay1997quantum,gattringer2009quantum} and group theory
\citep[e.g.][]{georgi1999lie,fulton2013representation}.

\section{Continuum gauge theories with fermions} \label{sec:continuum_gauge}

The undoubtaly most important quantity in Quantum Field Theory (\emph{QFT}) and
Statistical Mechanics (\emph{SM}) is $\mathcal{Z}$, named the generating
functional and the partition function in the two theories respectively. It is in
SM defined as
%
\begin{equation}
  \mathcal{Z}_E = \int \prod_i \mathrm{d} \phi_i e^{-\mathcal{S}_E[\phi_i]},
\end{equation}
%
where we have represented the field parameters by $\phi_i$. There is an analogue
between the two theories and one can convert between them by Wick rotating the
time coordinate $x^0_E = -i x^0_M$. Whenever the possibility for confusion
exists we use an $M$ to symbolise Minkowski space, the world of QFT, and an $E$
to symbolise Euclidean space, the world of SM.

The action, $\mathcal{S}$, is in turn defined by the Lagrangian density
%
\begin{equation}
  \mathcal{S}[\phi_i] = \int \mathrm{d}^4 x\, \mathcal{L}[\phi_i(x)],
\end{equation}
%
and our theories are therefore uniquely defined through $\mathcal{L}$ and the
properties of the fields $\phi_i$.

The most basic starting point for us will be the Dirac Lagrangian, which reads
%
\begin{equation}
  \mathcal{L} = \bar{\psi}_i \big(i\gamma^{\mu} \partial_{\mu} - m_i\big)
  \psi_i.
\end{equation}
%
where we have introduced the fermionic fields $\psi_i$. To get the right
statistics, these fields have to obey the anti-commutation relations, placing
them in the group known as Grassmann numbers
%
\begin{align}
  \big\{ \psi_i, \psi_j \big\} &\equiv \psi_i \psi_j + \psi_j \psi_i = 0,\\
  \big\{ \bar{\psi}_i, \bar{\psi}_j \big\} &\equiv \bar{\psi}_i \bar{\psi}_j +
    \bar{\psi}_j \bar{\psi}_i = 0,\\
  \big\{ \bar{\psi}_i, \psi_j \big\} &\equiv \bar{\psi}_i \psi_j + \psi_j
    \bar{\psi}_i = \delta_{ij}.
\end{align}

This Lagrangian describe the dynamics of a set of free, massive fermions, which
is not very exciting. There are a multitude of ways to add interactions while
still maintaining the desired symmetry properties (which we will discuss later).
One is to add a self interaction term, e.g. a four point term, $g
\big(\bar{\psi}\psi\big)^2$, which results in the NJL model
\citep{Nambu:1961tp}. Another option is to introduce more fields and
interactions. Finally one can localise the symmetries of the fields and induce
new fields and couplings. For this procedure we need a short overview of group
theory which we will tackle next.

\section{Important ideas from group theories} \label{sec:group_intro}

A group $G$ is a set and a rule for assigning to every (ordered) pair in the set
a new element also in the set
%
\begin{equation}
  \text{For every} \hskip1ex f, g \in G, \hskip1ex\text{then}\hskip1ex
    h = fg \in G
\end{equation}
%
this assignment is called the group product. The group product is associative,
and in every group a unit element exists ($e$). Every element also has an inverse so
that
%
\begin{equation}
  \forall\hskip.5ex f \in G, \hskip1ex\exists\hskip.5ex f^{-1}
    \hskip1ex:\hskip1ex f^{-1} f = f f^{-1} = e
\end{equation}
%
A representation of a group $G$ on a complex vector space $V$ is a homomorphism
%
\begin{equation}
  D_{R}:\hskip1em G \to \text{GL}(V)
\end{equation}
%
of the group $G$ on the group of automorphisms of $V$. Therefore representation
has to maintain the group structure so that
%
\begin{align}
  D_R(f) D_R(g) &= D_R(fg), \label{eq:group_prod}\\
  D_R(e) &= \mathbb{1}_R.
\end{align}
%
In physics we are normally content with a coordinate basis, in which the
representations map to $n$ by $n$ invertible matrices, i.e. $V := \mathbb{C}$,
$\text{GL}(V) := M_{\mathbb{C}}(n,n)$.

\subsection{Lie groups}

If the elements of a group $G$ depend smoothly on some set of parameters,
$g(\alpha_i) \in G$, we call that group a Lie group. Lie groups make up the
theory of continuous transformations, and are therefore of special interest to
physics. The continuous nature of the Lie groups create a sense of closeness
where two elements that are close to each other can be represented through
differentials. For example
%
\begin{equation}
  D_R(g(\delta\alpha_i)) = \mathbb{1}_R + i\, \delta \alpha_i T_{Ri} +
    \mathcal{O}(\delta \alpha^2).
\end{equation}
%
One commonly refers to the $T_{Ri}$ as the generators of the group $G$. Applying
the smoothness condition to the group product, \meqref{eq:group_prod},
we find that the commutator of the generators follow a special relation
%
\begin{equation}
  \big[ T_i, T_j \big] = i\, f_{ijk} T_k.
\end{equation}
%
The structure constants, $f_{ijk}$, are in fact independent of the
representation $R$. They are tremendously important as they summarise the entire
group multiplication law. They are also referred to as the \emph{algebra} of the
group, $\mathfrak{g}$. The generators and the group algebra will come into play
when we later introduce continuous symmetries of our quantum theory of fields.

\subsection{Character and character expansion}

One defines the character of a representation as
%
\begin{equation}
  \chi_R(g) = \tr D_R(g).
\end{equation}
%
The characters appear frequently in field theory due to the cyclic properties of
the trace and its ability to facilitate the creation of invariant objects. On
top of this the characters span a basis of functions that share this cyclic
properties, i.e. any function $\rho(f g f^{-1}) = \rho(g)$. Hence
%
\begin{equation}
  \rho(g) = \sum_r \rho_r \chi_r(g),
\end{equation}
%
with
\begin{equation}
  \rho_r = \int_{\mathrlap{g \in G}\hskip1.5ex} \mathrm{d}g \, \chi_r^*(g) \rho(g).
\end{equation}
%
It is normal to factor out both the trivial representation ($D_0(g) = 1$), as
well as the dimension of the representation, and write
%
\begin{equation}
  \rho(g) = \rho_0 \Big(1 + \sum_{r \neq 0} d_r \bar{\rho}_r \chi_r(g)\Big),
\end{equation}
%
which we will refer to as the character expansion. We will make use of this in
\secref{sec:char_exp}.

\subsection{Group integrals}

\section{Symmetry groups and continuum symmetries} \label{sec:symmetries}
\section{Lattice discretisation} \label{sec:lattice_intro}
\section{Symmetries on the lattice} \label{sec:lattice_symmetries}
\section{Fermion doubling and chiral symmetry} \label{sec:fermion_doubling}
\section{Scale setting and the continuum limit} \label{sec:scale_setting}
