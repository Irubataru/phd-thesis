\chapter{The Effective Theory}

Having presented the challenges and difficulties in simulating strongly
interacting fermions, especially in the dense regime, we will introduce an
effective theory that tackles some of these problems, while reproducing the full
theory in a certain parameter region. We will see that although simulations of
the effective theory still suffer from the side effects of the sign problem
(complex actions), the essence of the sign problem is weak enough that
reweighting can be readily applied.

The work in this thesis builds on previous work with the effective theory while
pushing the derivation further and introducing analytic tools, which we will
cover in the next chapter.

In this chapter we will first introduce the effective theory before we introduce
two expansion schemes that facilitate the computation of the theory. These are
namely the \emph{character expansion}, mentioned in \secref{sec-group_intro},
the second is the \emph{hopping parameter expansion} for heavy fermions. We
round off the chapter with a discussion on the numerical evaluation of the
effective theory.

\section{The effective theory - \texorpdfstring{\itshape Introduction}{Introduction}}

The essence of the derivation of the effective theory is to integrate out some
of the degrees of freedom analytically. This will both lessen the burden for the
numerical evaluation, having fewer degrees of freedom left to vary, as well as
lessen the sign problem. The sign problem being milder due to the fact that
many, or as we will see, most, of the fluctuations cancel exactly, as they
should. In this we will integrate the spatial gauge links of the partition
function
%
\begin{equation}
  \mathcal{Z} = \int \prod_{x, \mu} \mathrm{d} U_{\mu}(x) \, \det Q \, [U_{\mu}] \,
    e^{-\mathcal{S}_g[U_{\mu}]}
    \equiv \int \prod_{x} \mathrm{d} U_0(x) \,
    e^{-\mathcal{S}_{\text{eff}}[U_0]},
\end{equation}
%
which defines the effective action to be
%
\begin{equation} \label{eq:eff_action_def}
  \mathcal{S}_{\text{eff}} = - \log \int \prod_{x, i} \mathrm{d} U_i(x) \, \det
    Q \, [U_{\mu}] \, e^{-\mathcal{S}_g [U_{\mu}]}.
\end{equation}
%
The integrals over the spatial gauge links $U_i(x)$ is unfortunately not
something we can evaluate analytically without the aid of approximations. We
will therefore introduce two expansion schemes and work towards deriving the
effective theory such that it reproduce the exact expansion coefficients of the
full theory in the end.

\section{The character expansion} \label{sec-char_exp}

The first expansion we will apply is the character expansion introduced in 
\secref{sec-group_intro}. In the form of an exact equality, it is not of much
help. Nevertheless, from the character expansion of the single plaquette gauge
contribution
\footnote{We will from this point forward assume that the gauge group is SU$(N_c)$
  and that the fermions transform under the fundamental representation unless
  stated otherwise.}
%
\begin{equation}
  e^{-\beta (1 - \frac{1}{N_c} \text{Re} \tr U_p)} = u_0(\beta) \bigg(1 +
  \sum_{r \neq 0} d_r\, u_r(\beta)\, \chi_r(U_p) \bigg),
\end{equation}
%
we see that the character expansion coefficients are dependent on the lattice
gauge coupling $\beta$. It can be easily seen that the higher dimensional
representations come with a higher power of this coupling. A natural ordering
therefore arises if one expands around the infinite coupling limit, $g\to\infty$,
$\beta\to0$. This expansion scheme is aptly named the \emph{strong coupling
  expansion}, and has been the focus of numerous studies for the past decades,
also having picked up in recent years by groups studying conformal field
theories. Introductions to the field can be found in \cite{Drouffe:1983fv} and
\cite{montvay1997quantum}.

The lowest order character expansion coefficient, namely that of the fundamental
representation, has for SU$(3)$ been calculated to high orders
%
\begin{align}
  u_f(\beta) &= \frac{1}{N_c} \frac{\int \mathrm{d} g\, \tr g\, e^{-\frac{\beta}{2 N_c}
    ( \tr g + \tr g^{\dagger})}}{\int \mathrm{d} g\: e^{-\frac{\beta}{2 N_c}
    ( \tr g + \tr g^{\dagger})}} \nonumber\\
  &\hskip1em= \frac{
    x + \frac{1}{2} x^2 + x^3 + \frac{5}{8} x^4 + \frac{13}{24} x^5 + \mathcal{O}(x^6)%
  }{%
    1 + x^2 + \frac{1}{3} x^3 + \frac{1}{2} x^4 + \frac{1}{4} x^5 + \mathcal{O}(x^6)%
  }, \hskip2ex x=\frac{\beta}{2 N_c}.
\end{align}
%
To leading order $u_f(\beta) \approx \frac{\beta}{2 N_c^2}$, and we
therefore use $u_f$ as our expansion parameter rather than $\beta$. The
character expansion only permits a single plaquette from any representation to
be placed at every position, making order counting easier than a standard Taylor
expansion of the gauge action.

\section{Pure gauge effective theory} \label{sec-pure-gauge-theory}

With the character expansion at hand we can evaluate the pure gauge
contributions to the effective action. Ignoring the quark contribution the
effective action is
%
\begin{equation}
  e^{-\mathcal{S}_{\text{eff}}} = \int \big[ \mathrm{d} U \big]_i \, \prod_p
    \bigg(1 + \sum_{r \neq 0} d_r u_r(\beta) \chi_r (U_p) \bigg).
\end{equation}
%
where we have introduced the shorthand integration measure $\big[ \mathrm{d} U
\big]_i = \prod_{x,i} \mathrm{d} U_{i}(x)$.  Expanding the product over the
plaquettes gives a sum of terms which is of the form
%
\begin{equation}
  d_{r_1} u_{r_1}(\beta) \chi_{r_1}(U_{p_1}) \, 
  d_{r_2} u_{r_2}(\beta) \chi_{r_2}(U_{p_2}) \, \cdots
\end{equation}
%
where if one or more of the plaquettes in a term has a link that fall on
$(x,\mu)$, gives the integral
%
\begin{equation}
  \int \mathrm{d} U_{\mu}(x) \, \chi_{r_1} \big(U_{s_1} U_{\mu}(x) \big) \,
    \chi_{r_2} \big(U_{s_2} U_{\mu}(x) \big) \, \cdots,
\end{equation}
%
where $U_{s_i}$ is the remaining \emph{staple} after the link $U_{\mu}(x)$ has
been factored out of the plaquette. One approach to solving these integrals is
to carry out the Kronecker product of the representation matrices and
decompose them to their irreducible representations using the Clebsch-Gordan
coefficients. We see that only products whose Clebsch-Gordan series contains the
trivial representation vanish due to the identity
%
\begin{equation}
  \int \mathrm{d} g \, \chi_r (g) = \delta_{r,0}.
\end{equation}
%
On top of restricting the valid plaquette combinations sharing a link, it also
restricts the graphs created from combining plaquettes to ones that have no
boundaries. On an infinite lattice the lowest order contribution would therefore
come from combining six fundamental plaquettes into a cube.

For finite lattices the periodic boundary can be utilised to create closed
surfaces. In fact, only graphs periodic in the temporal direction give
contribution to the finite temperature as the non-periodic ones can be
normalised out. Since we only integrate spatial links the contributing graphs
need only have closed surfaces in the spatial directions. The lowest order
contribution to the effective action comes from a strip of plaquettes spanning
the temporal direction as shown in \figref{fig-plaquette-strip}. Since only two
links meet at all the spatial sites we only need the integral
%
\begin{equation}
  \int \mathrm{d} U \: \chi_r(V U) \: \chi_s(W U^{-1})
    = \delta_{r,s} \frac{1}{d_r} \chi_r (V W),
\end{equation}
%
which can be represented graphically as
%
\begin{equation} \label{eq-gauge-link-integral-graphic}
  \int \mathrm{d} U \;%
  \begin{tikzpicture}[baseline={(base)}]
    \coordinate (base) at (0,.35);
    \draw[link line] (0,0) rectangle (2,1);
    \draw[link] (0,0) -- (0,1)
      node[midway,left=1mm,inner sep=0pt,ColourBase,scale=0.8] {$V$};
    \draw[link] (2,1) -- (2,0)
      node[midway,right=1mm,inner sep=0pt,ColourBase,scale=0.8] {$W$};
    \draw[link double] (1,0) -- (1,1)
      node[midway,right=1mm,inner sep=0pt,ColourBase,scale=0.8] {$U$};
  \end{tikzpicture}
  \,=\, \frac{1}{d_r} \;
  \begin{tikzpicture}[baseline={(base)}]
    \coordinate (base) at (0,.35);
    \draw[link line] (0,0) rectangle (2,1);
    \draw[link] (0,0) -- (0,1)
      node[midway,left=1mm,inner sep=0pt,ColourBase,scale=0.8] {$V$};
    \draw[link] (2,1) -- (2,0)
      node[midway,right=1mm,inner sep=0pt,ColourBase,scale=0.8] {$W$};
  \end{tikzpicture} \,.
\end{equation}

\begin{figure}
  {\centering
    \includegraphics[width=.75\textwidth]{pure_gauge_strip}\par}
  \caption{Lowest order pure gauge contribution to the effective action}
  \label{fig-plaquette-strip}
\end{figure}

Integrating out the spatial links of the strip of plaquettes leaves two
disconnected loops at the neighbouring spatial lattice sites
%
\begin{equation}
  e^{-\mathcal{S}_{\text{eff}}} = 1 + \sum_{\langle \vec{x}, \vec{y} \rangle} u_f^{N_t}
  \big( L_{\vec{x}} L^*_{\vec{y}} + L^*_{\vec{x}} L_{\vec{y}} \big) + \mathcal{O}(u_f^{N_t+4})
\end{equation}
%
where $L$ is the so-called \emph{Polyakov loop}
%
\begin{equation}
  L_{\vec{x}} = \tr \prod_{\mathclap{t=0}}^{\mathclap{N_t-1}} U_0(\vec{x},t) .
\end{equation}
%
We see that the explicit time dependence of the links have disappeared as the
only degrees of freedom left are full windings. The integral over the effective
action therefore simplify to
%
\begin{equation}
  \mathcal{Z}_{\text{eff}} = \int \big[\mathrm{d} U\big]_0 \,
    e^{-\mathcal{S}_{\text{eff}}[L]} 
  = \int \prod_{\vec{x}} \mathrm{d} L_{\vec{x}} \, \sqrt{\det U_0} \,
    e^{-\mathcal{S}_{\text{eff}}[L]} ,
\end{equation}
%
where $\sqrt{\det U_0}$ is the reduced Haar measure of the group, the calculation of
which will be covered in \apxref{sec-haar_measure}. As one can see the effective
theory is a three dimensional theory of Polyakov loop interactions. At first
order we have a nearest neightbour spin system with an effective coupling
$u_f^{N_t}$.

At higher orders in $\beta$ new effects are introducted through interactions
between loops at higher order representations, next to nearest neighbour
interactions as well as corrections to the nearerst neighbour coupling between
fundamental Polyakov loops. The effects of higher order representations in
Polyakov loop effective theories was studied in \citep{Wozar:2007tz}. The
corrections to the fundamental nearest neighbour coupling was calculated to 
$\mathcal{O}(u_f^{N_t + 10})$ in \citep{Langelage:2010yr} while the effects of
long range interactions was examined in \citep{Bergner:2015rza}.

We will leave the topic of pure gauge effective theories for now as the work in
this thesis is mostly concerned with the cold and dense regime, in which pure
gauge corrections are negligible.

\section{The hopping parameter expansion}

Even for lattice simulations at zero chemical potential, evaluating the fermion
determinant is by far the most expensive operation. For heavy quarks it takes a
close to block diagonal form, while for light quarks the dynamics delocalise,
and no such simplifications appear. It is therefore clear that the analysis of
heavy quarks is of reduced complexity, and an expansion around this limit can be
used to derive an effective theory for heavy quarks. By rescaling the fields, we
see that the quark matrix can be refactored to be
%
\begin{equation}
  Q_{yx} = \delta_{yx} - \kappa H_{yx}, \hskip1em \kappa = \frac{1}{2(4 + am)}
\end{equation}
%
where we have introduced the \emph{hopping parameter} $\kappa$ and the
\emph{hopping matrix} $H$. The hopping matrix for Wilson fermions is
%
\begin{equation}
  H_{yx} = (1 \pm \gamma^0) e^{\pm a\mu} U_{\pm 0}(x) \delta_{y \mp \hat{0},x}
    + \sum_{\mu = \pm 1}^{\pm 3} (1 + \gamma^{\mu}) U_{\mu}(x) \delta_{y-\hat{\mu},x},
\end{equation}
%
where we have chosen $r=1$. We then expand the fermion propagator in powers of
$\kappa$ resulting in
%
\begin{equation}
  Q^{-1}_{yx} = \sum_{n=0}^{\infty} \kappa^n (H^n)_{yx} .
\end{equation}
%
Since every factor of $H$ comes with a $\delta_{y+\hat{\mu},x}$, they symbolise
a single discrete hop on the lattice. And the full fermion propagator is
therefore the sum of all fermion lines starting at $x$ and ending at $y$. Due to
the accompanying spin factor, and the fact that $(1 - \gamma^{\mu}) (1 +
\gamma^{\mu}) = 0$, it is restricted to lines with no backtracking.  If the
series is truncated, it is approximated by lines with a specific upper bound for
its length. The fermion matrix can likewise be rewritten using the trace-log
identity
%
\begin{equation}
  \det Q = \exp \big(\tr \log (1 - \kappa H) \big) = \exp \bigg( -\sum_{n=1}^{\infty} \frac{1}{n}
  \kappa^n \tr H^n \bigg).
\end{equation}
%
The trace over $H^n$ gives all closed fermion loops of length $n$ with no
backtracking. In lieu of the hopping expansion we see that the fermion
propagator is the sum of all fermion lines while the determinant is the
exponential of all fermion loops.

\section{Pure fermion effective theory} \label{sec-pure-fermion-eff-theory}

The first step towards deriving an effective three dimensional theory for heavy
quarks and strong coupling is to separate the temporal and spatial hops
%
\begin{equation}
  H_{yx} = T_{yx} + \sum_{i=1}^3 S_{i,yx},
\end{equation}
%
where the temporal and spatial hopping matrices are divided into positive and
negative components: $T = T^+ + T^-$, $S_i = S_i^+ + S_i^-$, and
%
\begin{align}
  T^{\pm}_{yx} &= (1 \pm \gamma^{0}) e^{\pm a\mu} U_{\pm 0}(x)\,
    \delta_{\vec{y},\vec{x}} \, \delta_{t_y, t_x\pm1},\\
  S_{i,yx}^{\pm} &= (1\pm \gamma^i) U_{\pm i}(x) \,
    \delta_{\vec{y},\vec{x}\pm\hat{i}} \,\delta_{t_y,t_x}.
\end{align}
%
The fermion determinant can then be refactored into static and kinematic factors
by factoring out the temporal hopping matrix
%
\begin{equation}
  \det (Q) = \det (1 - \kappa T - \kappa S)
   = \det\underbrace{(1 - \kappa T)}_{\qstat} \,
   \det \underbrace{\Big(1 - \frac{\kappa S}{1 - \kappa
       T}\Big)}_{\qkin}.
\end{equation}

\subsection{Static determinant}

For the derivation of the effective theory we need the full static propagator
and static determinant. Since every hop in the temporal direction come with a
fugacity factor, the true temporal hopping expansion parameter is $e^{\pm a\mu}
\kappa$, which is not a small parameter for sufficiently dense systems. 

We can simplify the static determinant through the trace-log identity
%
\begin{equation}
  \det \qstat \equiv \det (1 - \kappa T) = \exp \bigg(- \sum_{n=1}^{\infty}
    \frac{1}{n} \kappa^n \tr (T^+ + T^-)^n \bigg).
\end{equation}
%
Due to the no backtracking criterion we get no mixed $T^+ T^-$ terms, and the
static determinant factorises into fermion and anti-fermion static determinants
%
\begin{equation}
  \det (1 - \kappa T) = \det (1 - \kappa T^+)\, \det (1 - \kappa T^-) .
\end{equation}
%
The trace in the determinant restricts us to closed loops, which for the static
hopping matrix results in only full windings in the time direction. A term that
winds the lattice $n$ times in the positive direction has the mathematical form
%
\begin{align}
  &(-1)^n \kappa^{n N_t} (1+\gamma^0)^{n N_t} e^{n N_t a \mu}
    \sum_{i=0}^{N_t-1} \prod_{t_i=0}^{n N_t-1} U_0(\vec{x},t_i) \nonumber\\
  &\hskip3cm= \frac{1}{2} N_t (-1)^n (2 e^{a\mu} \kappa)^{n N_t} (1+\gamma^0) W^n(\vec{x}),
\end{align}
%
where $W(\vec{x})$ is the untraced Polyakov loop, the minus sign originate from
fermion anti-periodicity , and we have used the fact that $(1\pm\gamma^{\mu})^2
= 2 (1\pm\gamma^{\mu})$. The positive static determinant therefore simplifies to
%
\begin{equation}
  \exp\bigg(-\frac{1}{2} \tr(1+\gamma^0) \sum_{n=1}^{\infty} \frac{1}{n} (-h_1)^{n}
  \tr W^n (\vec{x}) \bigg) = \prod_{\vec{x}} \det(1 + h_1 W(\vec{x}))^2,
\end{equation}
%
in which $h_1(\mu) = (2e^{a\mu}\kappa)^{N_t} = z\, e^{N_t \log(2\kappa)}$
($=\bar{h}_1(-\mu)$) is the static loop (anti loop) weight. Since $W$ is simply
a product of $U_0$ matrices, it has to belong to the same symmetry group. We can
therefore use trace decomposition of the determinant together with the
Cayley-Hamilton theorem to express it in terms of the traces of $W$, namely the
Polyakov loops. We state the result for SU$(3)$
%
\begin{equation}
  \det(1 + h_1 W) = 1 + h_1 L + h_1^2 L^* + h_1^3
\end{equation}
%
and refer to \apxref{sec-evaluating-fermion-determinants} for the more generic
approach. The full static determinant is therefore
%
\begin{equation}
  \det(1 - \kappa T) = \prod_{\vec{x}} (1 + h_1 L_{\vec{x}} + h_1^2 L^*_{\vec{x}} + h_1^3)^2
    (1 + \bar{h}_1 L^*_{\vec{x}} +  \bar{h}_1^2 L_{\vec{x}} + \bar{h}_1^3)^2.
\end{equation}

\subsection{Static propagator}

The static propagator can be calculated in several ways. Either one can once
more apply the Cayley-Hamilton theorem to calculate the matrix inverse, or one
can expand in $\kappa$ and then resum the resulting expression to all orders.
Since the latter approach is limited in convergence, we will choose a third
method, a straightforward calculation of the matrix inverse. Once more, due to
the fact that backtracking is disallowed, the propagator separates into two
pieces
%
\begin{equation} \label{eq:static_prop_separation}
  \qstat^{-1} \equiv \frac{1}{1 - \kappa T} = \frac{1}{1 - \kappa T^+} + \frac{1}{1 - \kappa
    T^-} -1,
\end{equation}
%
and we are therefore content calculating one of these. The matrix in temporal
indices has a simple pseudo upper triangular shape, except for one term from the
periodic boundary condition
%
\begin{equation}
  (1-\kappa T^+)_{t_y t_x} = 
  \begin{pmatrix}
    1 & \minus\eta U_0 (1) & 0 & 0 & \cdots & 0\\
    0 & 1 & \minus\eta U_0 (2) & 0 & \cdots & 0\\
    0 & 0 & 1 & \minus\eta U_0 (3) & \cdots & 0\\
    \vdots & \vdots & \vdots & \vdots & \ddots & \vdots\\
    \eta U_0 (N_t) & 0 & 0 & 0 & \cdots & 1\\
  \end{pmatrix},
\end{equation}
%
where $\eta = (1+\gamma^0)\kappa e^{a\mu}$ is the time independent factor in
$T^+$. This matrix can be easily inverted by standard row reduction, giving
%
\begin{equation}
  \scalemath{0.8}{%
  \frac{1}{1\!+\!\prod_{t=1}^{N_t}\!\eta U_0(t)}
  \begin{pmatrix}
    1 & \eta U_0(1) & \eta^2 U_0(1) U_0(2) & \cdots & \prod_{t=1}^{N_t-1}\!\eta U_0(t)\\
    \minus\eta^{N_t-1} U^{\dagger}_0(2) & 1 & \eta U_0(2) & \cdots & \prod_{t=2}^{N_t-1}\!\eta U_0(t)\\
    \minus\eta^{N_t-2} U^{\dagger}_0(2) U^{\dagger}_0(3) & \minus\eta^{N_t-1}
      U^{\dagger}_0(3) & 1 & \cdots & \prod_{t=3}^{N_t-1}\!\eta U_0(t)\\
    \vdots  & \vdots  & \vdots  & \ddots & \vdots \\
    \minus\eta\prod_{t=2}^{N_t}\!U^{\dagger}_0(t) &
    \minus\eta^2\prod_{t=3}^{N_t}\!U^{\dagger}_0(t) &
    \minus\eta^3\prod_{t=4}^{N_t}\!U^{\dagger}_0(t) & \cdots & 1
  \end{pmatrix}}
\end{equation}
%
which on component form simplifies to
%
\begin{equation}
  (1 - \kappa T^{\pm})^{-1}_{t_yt_x} = \delta_{t_y,t_x} + \frac{1 \pm \gamma^0}{2} K^{\pm}_{t_yt_x},
\end{equation}
%
\vspace*{-2em}
%
\begin{alignat}{4}
  K^+_{t_y t_x} &= \frac{h_1 W}{1 + h_1 W} \delta_{t_y t_x} &&+
  (2 e^{a\mu}\kappa)^{t_y-t_x}
  &\frac{U_0(t_x \!\to\! t_y)}{1 + h_1 W}
  ( \theta_{t_y, t_x} - h_1 W\, \theta_{t_x, t_y}),\\
  K^-_{t_y t_x} &= \frac{\bar{h}_1 W^{\dagger}}{1 + \bar{h}_1 W^{\dagger}} \delta_{t_y t_x} &&+
  (2 e^{-a\mu}\kappa)^{t_x-t_y}
  &\frac{U_0(t_x \!\to\! t_y)}{1 + \bar{h}_1 W^{\dagger}}
  ( \theta_{t_x, t_y} - \bar{h}_1 W^{\dagger} \theta_{t_y, t_x}).
  \label{eq:kminus_def}
\end{alignat}
%
We have introduced the gauge transporter $U_0(t_x \!\to\! t_y)$, which is
%
\begin{equation}
  U_0(t_x \!\to\! t_y) =  \left\{ \renewcommand{\arraystretch}{1.75}
  \begin{array}{@{}l@{\quad}l@{}}
    \prod_{t = t_x}^{t_y-1} U_0(t) & \text{if} \hspace{.75em} t_x < t_y,\\
    \prod_{t = t_y}^{t_x-1} U_0^{\dagger}(t) & \text{if} \hspace{.75em} t_x > t_y.
  \end{array}\right.\kern-\nulldelimiterspace
\end{equation}
%
The full static propagator therefore reads
%
\begin{equation} \label{eq-qstat-full}
  (\qstat^{-1})_{t_y,t_x} = \delta_{t_y,t_x}
   + \frac{1 + \gamma^0}{2} K^+_{t_y,t_x}
   + \frac{1 - \gamma^0}{2} K^-_{t_y,t_x}
\end{equation}

\subsection{Spatial hopping expansion}

We now have all the necessary ingredients to start a systematic expansion of the
kinetic quark determinant. First, we introduce the fundamental building blocks
of the spatial hopping expansion
%
\begin{alignat}{99}
  P &= \>\sum_{i=1}^3 P_i &&= \frac{1}{1-\kappa T} \sum_{i=1}^3 \kappa S_i^+, \\
  M &= \>\sum_{i=1}^3 M_i &&= \frac{1}{1-\kappa T} \sum_{i=1}^3 \kappa S_i^-.
\end{alignat}
%
The $P$ and $M$ symbolise a single lattice hop in positive or negative spatial
directions, combined with arbitrary movement in the temporal direction,
including all windings. The kinetic determinant, the object of our expansion, is
simply
%
\begin{equation}
  \det \qkin = \det ( 1 - P - M ) = \exp \bigg(- \sum_{n=1}^{\infty}
  \frac{1}{n} \tr (P + M)^n \bigg),
\end{equation}
%
and since $P$ and $M$ both come with a single power of $\kappa$, we can expand
around these two being $0$. The determinant is described by all closed fermion
lines, and thus we need an equal number of positive and negative hops. The
next to leading order contribution is therefore
%
\begin{equation}
  \det \qkin = \exp \Big( - \tr(PM) - \tr(PPMM) -
    {\textstyle\frac{1}{2}}\tr(PMPM) + \mathcal{O}(\kappa^6)\Big).
\end{equation}
%
For every $P$ and $M$ we get a multitude of different combinations of terms
depending on whether we have fermions coupled to fermions, or fermions to
anti-fermions. The four combinations of the lowest order $\tr(PM)$ term is shown in
\figref{fig-lowest-order-fermionic}.

\begin{figure}
  {\centering
    \includegraphics[width=.75\textwidth]{pure_fermion_lowest}\par}
  \caption{The four contributions from the lowest order spatial hopping
    expansion.\hskip1em The principal path is indicated in \ColHlIText{} and the
    additional windings in \ColBaseText{}}
  \label{fig-lowest-order-fermionic}
\end{figure}

To be able to calculate the spatial gauge integrals, it is necessary to expand
the exponential. The lowest non trivial order contribution to the partition
function is therefore
%
\begin{equation} \label{eq:single_hop_partition}
  \mathcal{Z}_2 = \int \big[ \mathrm{d} U \big]_{\mu} \det \qstat \, e^{-\tr(PM)}
    = \int \big[ \mathrm{d} U \big]_{\mu} \det \qstat \, (1 - \tr(PM) + \mathcal{O}(\kappa^4)).
\end{equation}
%
The only nontrivial integral is the one over the $PM$ factor. Focussing on the
spatial links only, the integral to be solved is
%
\begin{multline}
  I\big[PM\big] = \int \big[ \mathrm{d} U \big]_i \tr(PM) \\
   = \kappa^2 \int \big[ \mathrm{d} U \big]_i \scalemath{0.95}{\sum_{\vec{x}, j}} \tr_{sct} \hskip-.5ex
   \big( \qstat^{-1}(\vec{x})(1+\gamma_j) U_j(\vec{x},t_1) 
   \qstat^{-1}(\vec{x}+\hat{j})(1-\gamma_j)U^{\dagger}_j(\vec{x},t_2) \big).
   \label{eq-second-order-spatial-integral}
\end{multline}
%
Using one of the simplest gauge integral selection rules
\meqref{eq-integral-selection-rule}, we see that this integral is only non zero
if the two links overlap, namely if $t_1 = t_2$. The implications of which
manifest itself in the temporal indices only. We therefore divide the evaluation
of the trace into its three remaining indices: spin, colour and temporal.

The spin indices are unrelated to the group integral and can be evaluated
immediately. Inserting the expression for $\qstat^{-1}$, \meqref{eq-qstat-full},
we get
%
\begin{multline} \textstyle
  \tr_s \Big( ( 1
   + \frac{1 + \gamma^0}{2} K^+_{\vec{x}}
   + \frac{1 - \gamma^0}{2} K^-_{\vec{x}} ) (1+\gamma_j)
  ( 1 + \frac{1 + \gamma^0}{2} K^+_{\vec{x}+\hat{j}}
   + \frac{1 - \gamma^0}{2} K^-_{\vec{x}+\hat{j}} ) (1-\gamma_j) \Big) \\[5pt]
   = 2 (K^+_{\vec{x}} - K^-_{\vec{x}})
    (K^+_{\vec{x}+\hat{j}} - K^-_{\vec{x}+\hat{j}}).
\end{multline}
%
Inserted back into \meqref{eq-second-order-spatial-integral} we have reduced its
complexity
%
\begin{multline}
  I\big[PM\big]
   = 2\kappa^2 \int \big[ \mathrm{d} U \big]_i \scalemath{0.95}{\sum_{\vec{x}, j}} \tr_{ct} \hskip-.5ex
   \big( (K^+_{\vec{x}} - K^-_{\vec{x}}) U_j(\vec{x},t_1) 
   (K^+_{\vec{x}+\hat{j}} - K^-_{\vec{x}+\hat{j}})U^{\dagger}_j(\vec{x},t_2) \big), \\
  = 2\kappa^2 \tr_t \scalemath{0.95}{\sum_{\vec{x}, j}}
    (K^+_{\vec{x}} - K^-_{\vec{x}})_{ab} (K^+_{\vec{x}+\hat{j}} - K^-_{\vec{x}+\hat{j}})_{cd}
    \>\delta_{t_1,t_2} \int \mathrm{d} U \, U_{bc} U^{\dagger}_{da}.
\end{multline}
%
In the second line we reintroduced the colour indices, carried out the
unoccupied link integrals and renamed the spatial links to $U$. Making use of
the group integral \meqref{eq-uub-integral}, we get
%
\begin{align}
  I\big[ PM \big] &= \frac{2\kappa^2}{N_c} \tr_t \scalemath{0.95}{\sum_{\vec{x}, j}}
    (K^+_{\vec{x}} - K^-_{\vec{x}})_{ab} (K^+_{\vec{x}+\hat{j}} - K^-_{\vec{x}+\hat{j}})_{cd}
    \>\delta_{t_1,t_2} \delta_{ab} \delta_{cd}, \nonumber\\
  &=\frac{2\kappa^2}{N_c} \tr_t \scalemath{0.95}{\sum_{\vec{x}, j}}
   \tr_c (K^+_{\vec{x}} - K^-_{\vec{x}}) \tr_c (K^+_{\vec{x}+\hat{j}} - K^-_{\vec{x}+\hat{j}})
    \>\delta_{t_1,t_2}.
\end{align}
%
The final step missing is evaluating the temporal trace, which is easily done by
summing over the delta and picking out only the diagonal pieces of $K^{\pm}$
%
\begin{multline}
  I\big[ PM \big] = \frac{2\kappa^2}{N_c} \sum_{t_1, t_2} \sum_{\vec{x, j}}
   \tr_c (K^+_{\vec{x}} - K^-_{\vec{x}})_{t_1 t_2} \tr_c (K^+_{\vec{x}+\hat{j}} - K^-_{\vec{x}+\hat{j}})_{t_2 t_1}
    \>\delta_{t_1,t_2}, \\
  = \frac{\kappa^2 N_t}{N_c} \sum_{\langle \vec{x}, \vec{y} \rangle}
    \tr_c \bigg(\frac{h_1 W_{\vec{x}}}{1 + h_1 W_{\vec{x}}}
    - \frac{\bar{h}_1 W_{\vec{x}}^{\dagger}}{1 + \bar{h}_1 W_{\vec{x}}^{\dagger}}\bigg) \\
  \times \tr_c \bigg(\frac{h_1 W_{\vec{y}}}{1 + h_1 W_{\vec{y}}}
    - \frac{\bar{h}_1 W_{\vec{y}}^{\dagger}}{1 + \bar{h}_1 W_{\vec{y}}^{\dagger}}\bigg).
\end{multline}
%
The final sum over the time-slice resulted in a factor $N_t$. Analysing this
expression we see that due to the fact that there is no colour mixing between
sites, the lowest order contribution to the spatial hopping expansion of the
kinetic determinant is simply a nearest neighbour interaction between Polyakov
loop dependent objects. We will see later that is is useful to introduce the
short hand
%
\begin{equation}
  W_{nm}(\vec{x}) = \tr_c \frac{(h_1 W_{\vec{x}})^m}{(1 + h_1 W_{\vec{x}})^n}, 
  \hskip1ex\text{and}\hskip1ex
  \xoverline{W}_{nm}(\vec{x}) = \tr_c
    \frac{(\bar{h}_1 W^{\dagger}_{\vec{x}})^m}{(1 + \bar{h}_1 W^{\dagger}_{\vec{x}})^n}
\end{equation}
%
With this the lowest order spatial hopping contribution to the effective action
can be written as
%
\begin{equation} \label{eq:lowest_order_ea_unsummed}
  e^{-\mathcal{S}_{\text{eff}}} = 1 - \frac{\kappa^2 N_t}{N_c}
  \sum_{\langle \vec{x}, \vec{y} \rangle} 
    \big( W_{11}(\vec{x}) - \xoverline{W}_{11}(\vec{x}) \big)
    \big( W_{11}(\vec{y}) - \xoverline{W}_{11}(\vec{y}) \big) + \mathcal{O}(u, \kappa^4)
\end{equation}
%
At the next to leading order in $\kappa$ we get new terms both to the nearest
neighbour contribution as well as non local terms spanning further on the
lattice. Examples of these contributions are sketched in
\figref{fig-next-order-fermionic}.

\begin{figure}
  {\centering
    \includegraphics[width=.75\textwidth]{pure_fermion_next_order_examples}\par}
  \caption{Examples of next to leading order contributions to the kinematic
    determinant. Left: single fermion hopping twice between nearest neighbouring
    sites. Middle: two separate fermions hopping between the same nearest
    neighbouring sites. Right: single fermion visiting a next to nearest
    neighbour.}
  \label{fig-next-order-fermionic}
\end{figure}

\section{Mixed contributions and gauge corrections}

So far we have only considered either pure gluonic contributions or pure
fermionic contributions, no mix between the two. In \secref{sec-pure-gauge-theory}
we considered an expansion in $\beta$, ignoring corrections from $\kappa$, while
in \secref{sec-pure-fermion-eff-theory} we carried out an expansion in $\kappa$
only. In this section we will see how these two expansions affect each other,
and how the effects can mostly be absorbed into shifts in the effective coupling
constants.

\subsection{Fermionic corrections}

The simplest fermionic correction one can imagine is to replace any gauge
plaquette by a fermionic loop, given that they have the same group structure
%
\begin{equation}
  \begin{tikzpicture}[baseline=0.4121cm]
    \draw[plaquette] (0,0) -- (1,0) -- (1,1) -- (0,1) -- cycle;
    \draw[-{Stealth[round]}] (1.75,0.5) -- (2.75,0.5);
    \begin{scope}[xshift=3.5cm]
      \draw[plaquette] (0,0) -- (1,0) -- (1,1) -- (0,1) -- cycle;
      \node at (1.75,0.5) {$+$};
      \draw[link,xshift=2.5cm] (0,0) -- (1,0) -- (1,1) -- (0,1) -- cycle;
    \end{scope}
  \end{tikzpicture}\;.
\end{equation}
%
We get a contribution from every fermionic flavour, and this results in a shift
in $\beta$
%
\begin{equation}
  \beta_R \to \beta_R +  16 d_R {\textstyle\sum_f}\kappa_f^4.
\end{equation}
%
The next order contribution comes from replacing a pair of plaquettes by six
fermion hops
%
\begin{equation}
  \begin{tikzpicture}[baseline=0.4121cm]
    \draw[plaquette] (0,0) -- (1cm - \deltaskip,0)  -- +(0,1) -- (0,1) -- cycle;
    \draw[plaquette,xshift=1cm] (\deltaskip,0) -- (1,0)  -- +(0,1) -- (\deltaskip,1) -- cycle;
    \draw[-{Stealth[round]}] (2.75,0.5) -- (3.75,0.5);
    \begin{scope}[xshift=4.5cm]
      \draw[plaquette] (0,0) -- (1cm - \deltaskip,0)  -- +(0,1) -- (0,1) -- cycle;
      \draw[plaquette,xshift=1cm] (\deltaskip,0) -- (1,0)  -- +(0,1) -- (\deltaskip,1) -- cycle;
      \node at (2.75,0.5) {$+$};
      \draw[link,xshift=3.5cm] (0,0) -- (1,0)  -- (2,0) -- (2,1) -- (1,1) -- (0,1) -- cycle;
    \end{scope}
  \end{tikzpicture}\;,
\end{equation}
%
which only give the same result after link integration due to
\meqref{eq-gauge-link-integral-graphic}.  Higher order corrections can be
constructed in a similar manner. We will see that this means that most of the
finite $\beta$ corrections to the effective theory can also be implemented in
terms of $\kappa$ corrections by simply replacing the plaquettes by a
sufficiently long fermion loop with the same geometric border.

\subsection{Gauge corrections}

Since we have shown that many of the gluonic corrections to our theory can be
reproduced by fermionic loops we turn our attention to finding these. We will
focus on corrections to the pure fermionic effective theory, ignoring the pure
gauge theory as it is subdominant in the cold and dense regime.

The first corrections to consider are those to the static determinant. They
consist of making detours in the spatial directions, filling the surface with
plaquettes
%
\begin{multline} \setlength{\medmuskip}{2mu} \setlength{\thickmuskip}{5mu}
  \label{eq-static-loop-gauge-corr}
  \qstat^{N_t}e^{-\mathcal{S}_g} = 
  \begin{tikzpicture}[baseline=.25em, scale=0.75]
    \clip (0.25,-0.25) rectangle (3.25,1.25);
    \draw[link] (0,0) -- ++(1,0) -- ++(1,0) -- ++(1,0) -- ++(1,0) -- ++(1,0);
  \end{tikzpicture}
  +
  \begin{tikzpicture}[baseline=.25em,scale=0.75]
    \clip (0.25,-0.25) rectangle (3.25,1.25);
    \draw[link] (0,0) -- ++(1,0) -- ++(0,1) -- ++(1,0) -- ++(0,-1) -- ++(1,0) -- ++(1,0) -- ++(1,0);
    \draw[plaquette,xshift=1cm] (2\deltaskip,0) -- ++(1cm - 4\deltaskip,0) --
      ++(0,1cm - 2\deltaskip) -- ++(4\deltaskip - 1cm,0) -- cycle;
  \end{tikzpicture}
  +
  \begin{tikzpicture}[baseline=.25em,scale=0.75]
    \clip (0.25,-0.25) rectangle (3.25,1.25);
    \draw[link] (0,0) -- ++(1,0) -- ++(0,1) -- ++(1,0) -- ++(1,0) -- ++(0,-1) -- ++(1,0) -- ++(1,0);
    \draw[plaquette,xshift=1cm] (2\deltaskip,0) -- ++(1cm - 3\deltaskip,0) --
      ++(0,1cm - 2\deltaskip) -- ++(3\deltaskip - 1cm,0) -- cycle;
    \draw[plaquette,xshift=2cm] (1\deltaskip,0) -- ++(1cm - 3\deltaskip,0) --
      ++(0,1cm - 2\deltaskip) -- ++(3\deltaskip - 1cm,0) -- cycle;
  \end{tikzpicture}
  + \mathcal{O}(\kappa^4 \beta,\kappa^2 \beta^3).
\end{multline}
%
These types of diagrams also result in Polyakov loops after spatial gauge
integration%
\footnote{As long as their extended boundary is not in contact with other links
  or Polyakov loops at the point of integration},
and can therefore be absorbed into a redefinition of the invariant parameters of
the static determinant, namely the looping weight $h_1(z,\kappa)$. The shift in
$h_1$ has been calculated to higher orders in the expansion parameters,
$\kappa, u$ \citep{Fromm:2011qi,Christensen:2013xea}
%
\begin{equation} \label{eq:h1_corrections}
  h_1(z,\kappa,\beta) = h_1(z,\kappa) \exp \bigg( 6 N_t \kappa^2 u \bigg(
    \frac{1-u^{N_t-1}}{1-u} + 4 u^4 - 12\kappa^2 + 9\kappa^2 u +
    \mathcal{O}(\kappa^4, u^4)\bigg) \bigg).
\end{equation}

Analogously, the nearest neighbour coupling strength has a similar correction
scheme where we shift spatial hops and fill it with plaquettes. The diagrams are
shown in \figref{fig-nearest-neighbour-corrections} and they give the following
corrections to nearest neighbour interactions \citep{Langelage:2014vpa}
%
\begin{multline}
  I\big[PM\big] \to
    \frac{\kappa^2 N_t}{N_c} \Big(1 + 2 \frac{u-u^{N_t}}{1-u} + 8u^5 + 16\kappa^2 u^4\Big)\\
  \times \sum_{\langle \vec{x}, \vec{y} \rangle} 
    \big( W_{11}(\vec{x}) - \xoverline{W}_{11}(\vec{x}) \big)
    \big( W_{11}(\vec{y}) - \xoverline{W}_{11}(\vec{y}) \big).
\end{multline}
%
It is therefore useful to introduce the nearest neighbour coupling constant
%
\begin{equation}
  h_2(\kappa,\beta) = \frac{\kappa^2 N_t}{N_c} \Big(1 + 2 \frac{u-u^{N_t}}{1-u} + 8u^5
    +16\kappa^2 u^4 + \mathcal{O}(\kappa^4 u^3)\Big),
\end{equation}
%
which will frequently appear in later calculations.

\begin{figure}
  {\centering
    \includegraphics[width=.75\textwidth]{nearest_neighbour_corrections}\par}
  \caption{Diagrams contributing to the corrections to the nearest neighbour
    coupling constant of the effective theory. From left to right
    $\mathcal{O}(1)$, $\mathcal{O}(u)$, $\mathcal{O}(u^5)$ and
    $\mathcal{O}(\kappa^2 u^4)$ respectively.}
  \label{fig-nearest-neighbour-corrections}
\end{figure}

\section{Resummations}

One of the more powerful tools available to improve convergence is the process
of resumming an infinite series of terms into a closed analytic expression. This
has already been utilised in the expression for $h_1$, \meqref{eq:h1_corrections}.
We will however go though the exponensiation of the effective action in more
detail as it is an integral part of the linked cluster expansion which will be
introduced in \chapref{chap5}.

Expanding the single hop partition function,
\meqref{eq:single_hop_partition}, to all powers gives
%
\begin{equation}
  \mathcal{Z}_2 = \int \big[ \mathrm{d} U \big]_{\mu} \det \qstat \,
    \sum_{n=0}^{\infty} \frac{(-1)^n}{n!} \big( \tr_{xsc}(PM) \big)^n
\end{equation}
%
where every power carries an independent sum over the spatial degrees of
freedom. These traces will contain both terms in which the $PM$ matrices have
overlapping spatial links as well as terms where they are separate. In the
latter case the integrals themselves separates, and to lowest order we have
%
\begin{align}
  \mathcal{Z}_2 
  = \int \big[ \mathrm{d} U \big]_{0} \det \qstat \,
    \sum_{n=0}^{\infty} \frac{(-1)^n}{n!} \bigg( \int \big[\mathrm{d} U\big]_i\tr_{xsc}(PM) \bigg)^n
    +\mathcal{O}(k^4)&, \nonumber\\
  = \int \big[ \mathrm{d} U \big]_{0} \det \qstat \,
    \exp \bigg( -\int \big[\mathrm{d} U\big]_i \tr_{xsc}(PM) \bigg) +
    \mathcal{O}(\kappa^4)&.
\end{align}
%
We see that since the effective action originate from an exponential it
naturally also resums to one, with corrections due to overlapping terms can be
taken into account order by order in a systematic way. This resummation gives a
more satisfactory expression for $S_{\text{eff}}$ which we gave to first order
in \meqref{eq:lowest_order_ea_unsummed}
%
\begin{equation} \label{eq:lowest_order_ea_unsummed}
  S_{\text{eff}} = \frac{\kappa^2 N_t}{N_c}
  \sum_{\langle \vec{x}, \vec{y} \rangle} 
    \big( W_{11}(\vec{x}) - \xoverline{W}_{11}(\vec{x}) \big)
    \big( W_{11}(\vec{y}) - \xoverline{W}_{11}(\vec{y}) \big) + \mathcal{O}(u, \kappa^4).
\end{equation}
%
Since the effective action is given by the logarithm of the partial partition
function, it is more advantageous to expand this quantity directly. To
facilitate this we introduce the method of moments and cumulants.

\subsection{Method of moments and cumulants}

The method of moments and cumulants is an elegant mathematical formalism which
can be utilised to extract the correct infinite volume limit for thermodynamic
physics \citep{Rushbrooke:1980zb,Munster:1980iv}, and will be an integral
part in the linked cluster expansion of \chapref{chap5}.

The \emph{moment}, $\moment$, is a symmetric function operating on symbols
where the moment product
%
\begin{equation}
  \moment_1 \otimes \moment_2 =  \moment_3
\end{equation}
%
is defined by
%
\begin{equation}
  \moment[\alpha, \dots, \beta]_3 = \sum_{p_2} \moment[\alpha, \dots, \delta]_1\,
    \moment[\gamma, \dots, \epsilon]_2
\end{equation}
%
where the sum is over all partitions of the symbols $\alpha, \dots, \beta$ into
two sets. The \emph{cumulant}, $\cumulant$, of the moment $\moment$ is defined
through the $\otimes$ exponential
%
\begin{equation}
  \exp_{\otimes} \cumulant = 1 + \sum_{n=1}^{\infty} \frac{1}{n!} \cumulant^{\otimes n}
    \equiv 1 + \moment,
\end{equation}
%
The moments and the cumulants can then be defined in terms of the other using
partition sums
%
\begin{align}
  \moment[\alpha_1, \dots, \alpha_n] &= \sum_{k=1}^{n} \sum_{p_k} \cumulant[\alpha_1, \dots, \alpha_m]_1
    \dots \cumulant[\alpha_i, \dots, \alpha_j]_k \\
  \cumulant[\alpha_1, \dots, \alpha_n] &= \sum_{k=1}^{n} (-1)^{k-1} (k-1)! \sum_{p_k} 
  \moment[\alpha_1, \dots, \alpha_m]_1
    \dots \moment[\alpha_i, \dots, \alpha_j]_k
\end{align}
%
We define the generating functional $f_{\moment}$ through
indexed variables, $x_{\alpha}$ for $\alpha$ in the set of symbols
%
\begin{equation}
  f_{\moment}(\{x_\alpha\}) = \sum_{n=1}^{\infty} \sum_{\alpha_1,\dots,\alpha_n}
    \frac{1}{n!} \moment[\alpha_1, \dots, \alpha_n] x_{\alpha_1} \cdots
    x_{\alpha_n}
\end{equation}
%
with an analogous definition for $f_{\cumulant}(\{x_\alpha\})$. The main theorem
of the method of moments and cumulants then tell us
%
\begin{equation} \label{eq:moment_cumulant_theorem}
  \exp f_{\cumulant}(\{x_{\alpha}\}) = 1 + f_{\moment}(\{x_{\alpha}\})
\end{equation}
%
which can be easily proven through induction.

Next we want to apply this method to the so far computed effective theory so
that we can generalise the exponentiation procedure. We want to compute the
effective action which is defined by \meqref{eq:eff_action_def}. Let us for now
consider the strong coupling limit of this expression
%
\begin{equation}
  e^{-S_{\text{eff}}} = \int \big[ \mathrm{d} U \big]_i \,
    \exp \bigg(- \sum_{n=1}^{\infty} \frac{1}{n} \tr (P + M)^n \bigg).
\end{equation}
%
We define the general polymer variables $X_i$ to represent a combination of $\tr
(P + M)^n$ factors with a connected set of overlapping links and a given spatial
extent on the lattice. The function $I(X_i)$ gives the value after integration
over the spatial links of the polymer $X_i$. We introduce a cluster moment such
that
%
\begin{equation}
  \moment[X_1, \dots, X_n] =
  \begin{cases}
    1, & \text{if every $X_i$, $X_j$ is disconnected} \\
    0, & \text{otherwise}
  \end{cases}
\end{equation}
%
with which we can easily express the effective action
%
\begin{equation}
  e^{-S_{\text{eff}}} = 1 + \sum_{n=1}^{\infty} \sum_{X_1, \dots, X_n} \frac{1}{n!}\,
    \moment[X_1, \dots, X_n] I(X_1) \cdots I(X_n)
\end{equation}
%
because we know that the integrals factorise if the polymers share no spatial
links. We can then compute the logarithm of the above expression using
\meqref{eq:moment_cumulant_theorem}
%
\begin{equation}
  S_{\text{eff}} = -\sum_{n=1}^{\infty} \sum_{X_1,\dots,X_n} \frac{1}{n!}\,
    \cumulant[X_1,\dots,X_n]\, I(X_1) \cdots I(X_n).
\end{equation}
%
The crucial observation is that due to the alternating sign in the formula for
the cumulant in terms of the moments, the cumulants posses the opposite property
of the moments
%
\begin{equation}
  \cumulant[X_1,\dots,X_n] \neq 0 \iff X_1 \cup \dots \cup X_n \text{ is connected}.
\end{equation}
%
We can therefore conclude that the effective action properly exponentiates if
one considers connected polymers only, and their combinatorial prefactors are
given by the cumulants. It should be noted that this is only true in the
infinite volume limit, but corrections can easily be carried out in an order by
order basis. Later in the analytical chapter we will work in the actual limit
where the exponentiation is exact.

\subsection{Logarithmic resummation}

One final resummation scheme will be discussed in the following section. It is
based on a resummation of the exponentiated action into a logarithm as was
carried out for the pure gauge action in \citep{Langelage:2010yr}. Although it
does not provide as much numerical benefit as the pure gauge resummation does,
we will see that it will prove to be a useful tool when we turn to analytic
evaluation.

We will focus on the lowest order order term in the effective action before
integration, the previously mentioned $\mathcal{Z}_2$. We want to study the
expression one obtains when all the nearest neighbour interactions lie on the
same spatial link
%
\begin{equation}
  S_{\text{eff}} = - \sum_{n=1}^{\infty} \sum_X \frac{1}{n!} \cumulant[X, ..., X] \, I(X)^n + \mathcal{O}(\kappa^4)
\end{equation}
%
where $X = \tr_{xsc}(PM)$. In the earlier calculations we saw that $\sum_X = N_t
\sum_{\langle x,y \rangle}$ and the single link cumulant is
$\cumulant[X, \dots, X] = (-1)^{n-1} (n-1)!$. The $\mathcal{Z}_2$ effective
action therefore gives
%
\begin{align}
  S_{\text{eff}} &= - N_t \sum_{\langle x,y \rangle} \sum_{n=1}^{\infty}
    \frac{(-1)^{n-1}}{n} I(X)^n + \mathcal{O}(\kappa^4), \nonumber\\
    &= \hphantom{-} N_t \sum_{\langle x,y \rangle} \log \big( - I(X) \big) + \mathcal{O}(\kappa^4)
\end{align}
%
and the full $\mathcal{Z}_2$ is therefore
%
\begin{equation}
  \mathcal{Z}_2 \approx \int_{U_0} \det \qstat \prod_{\langle \vec{x},\vec{y} \rangle}
    \Big(1 - \frac{\kappa^2}{N_c} 
    \big( W_{11}(\vec{x}) - \xoverline{W}_{11}(\vec{x}) \big)
    \big( W_{11}(\vec{y}) - \xoverline{W}_{11}(\vec{y}) \big) \Big)^{N_t},
\end{equation}
%
with higher order corrections at $\mathcal{O}(\kappa^4)$. This formulation is
particularly useful for studying e.g. the finite volume dependence of the Yang
Lee zeros.

\section{The cold and dense}

Although we have built the foundations for the calculation of the effective 3D
theory, we have still not presented any results beyond leading order in the
hopping expansion beyond the discussions on resummation. In
\citep{Langelage:2014vpa} the effective theory was computed to
$\mathcal{O}(\kappa^4)$, and a detailed computation of the appearing terms as
well as gauge corrections can be found in \citep{Neuman:2015zb}. In the same
publications an observation was made that the effective action greatly simplify
in the cold and dense limit, an area of QCD which is of great interest. The two
limits aid our computations in two ways. In the dense limit the thermodynamics
is dominated by quarks, not anti-quarks, and we can therefore neglect any such
terms involving anti-quarks. Mathematically this would be an expansion to
zeroth order in $\bar{h}_1$. The simplifications coming from the cold limit, the
limit in which $N_t \to \infty$ are more subtle and is easiest to see with an
example.

\begin{figure}
  {\centering
    \includegraphics[width=.75\textwidth]{cold_pmpm_hops}\par}
  \caption{The different spatial and temporal occupation of the $\tr(PMPM)$ term
    appearing at the NLO of the hopping parameter expansion. The last figure in
    which all spatial hops occupy the same spatial link vanish in the cold
    limit.}
  \label{fig-pmpm-hop-variants}
\end{figure}

Consider the term $\tr(PMPM)$ appearing at NLO of the hopping parameter
expansion. The term has four spatial hops which has to pair up for the gauge
integral to give something non-zero. There are three fundamental scenarios
which fulfil this criterion, these are depicted in \figref{fig-pmpm-hop-variants}.
In the leftmost graph the two pair of hops occupy different spatial positions
and can therefore never overlap. The temporal sum in which case gives a factor
$N_t^2$ as the two hops can independently choose a temporal slice. When the two
hops occupy the same nearest neighbour pair we have two situations. There are
$N_t(N_t-1)$ copies in which they occupy different time slices and $N_t$ copies
in which they overlap. Since
%
\begin{equation}
  \int \mathrm{d} U\, U U^{\dagger} U U^{\dagger} \neq
  \bigg(\int \mathrm{d} U\, U U^{\dagger} \bigg)^2
\end{equation}
%
they naturally give different results, all of which is accounted for by the
method of moments and cumulants. However, the integrals and spin traces are
independent of the number of temporal lattice sites, and there therefore has to
exist an $N_t$ for which
%
\begin{equation}
  N_t(N_t-1) \bigg( \int \mathrm{d} U \, U U^{\dagger} \bigg)^2 \gg
  N_t \int \mathrm{d} U \, U U^{\dagger} U U^{\dagger}
\end{equation}
%
and the more complicated overlapping diagrams can be neglected.

\subsection{Combinatorics}

Before we present the N\textsuperscript{3}LO result for the effective action in
the cold and dense limit we review some of the combinatorics that went into the
computation.

Because of the selection criterion for non-vanishing gauge integrals
\meqref{eq-integral-selection-rule} we see that one must restrict the coordinate
sums in the matrix multiplications of the $P$ and $M$ matrices in such a way
that spatial links overlap and give contributing results. We therefore define a
contraction
%
\begin{subequations}
  \begin{align}
    \begin{tikzpicture}[inline math]
      \node (one) {$P_{xy}$};
      \node (two) [right=0pt of one]{$M_{zw}$};
      \path[contraction] ([xshift=5pt]one.north west) edge [skip loop=6pt] ([xshift=7pt]two.north west);
    \end{tikzpicture} &= P_{xy} M_{zw} \,\delta_{\vec{y}\vec{z}} \delta_{\vec{x}\vec{w}} \,\delta_{t_yt_w} \,,
    \label{eq:contraction_def_pm}\\
    \begin{tikzpicture}[inline math]
      \node (one) {$P_{xy}$};
      \node (two) [right=0pt of one]{$P_{zw}$};
      \path[contraction] ([xshift=5pt]one.north west) edge [skip loop=6pt] ([xshift=5pt]two.north west);
    \end{tikzpicture} &= P_{xy} P_{zw} \,\delta_{\vec{x}\vec{z}} \delta_{\vec{y}\vec{w}} \,\delta_{t_yt_w} \,,
    \label{eq:contraction_def_pp}
  \end{align}
\end{subequations}
%
which can trivially be extended to the contraction of more elements. We see that
the contraction fixes a single temporal index as well as fully fixes the
spatial position and orientation of every matrix but one. One can also contract
more than one matrix as can be seen in \tabref{tab-pmpmpm-contractions} showing
all non-vanishing contractions of the NNLO term $\tr(PMPMPM)$. Due to the fact
that every contraction fixes a temporal index in all matrices involved we see
that the degrees of freedom naturally decrease. The various contractions of
\tabref{tab-pmpmpm-contractions} can be categorised into three distinct groups
%
\begin{subequations}
  \tikzset{
    every node/.style = {anchor=south, text depth=0.ex, inner sep=1pt}
  }
  \begin{align}
    \sum_{\mathclap{t_1,t_2,t_3}} \begin{tikzpicture}[baseline = .5ex]
      \node (a) {$P$};
      \node (b) [right=0pt of a]{$M$};
      \node (c) [right=0pt of b]{$P$};
      \node (d) [right=0pt of c]{$M$};
      \node (e) [right=0pt of d]{$P$};
      \node (f) [right=0pt of e]{$M$};
      \path[contraction] (a.north) edge [skip loop=6pt] (b.north);
      \path[contraction] (c.north) edge [skip loop=6pt] (d.north);
      \path[contraction] (e.north) edge [skip loop=6pt] (f.north);
    \end{tikzpicture} \propto N_{t}^3\;,\\
    \sum_{t_1,t_2}\begin{tikzpicture}[baseline = .5ex]
      \node (a) {$P$};
      \node (b) [right=0pt of a]{$M$};
      \node (c) [right=0pt of b]{$P$};
      \node (d) [right=0pt of c]{$M$};
      \node (e) [right=0pt of d]{$P$};
      \node (f) [right=0pt of e]{$M$};
      \path[contraction] (a.north) edge [skip loop=6pt] (b.north);
      \path[contraction] (c.north) edge [skip loop=6pt] (f.north);
      \path[contraction] (d.north) edge [skip loop=6pt] (e.north);
    \end{tikzpicture} \propto N_{t}^2\;,\\
    \sum_{t_1,t_2}\begin{tikzpicture}[baseline = .5ex]
      \node (a) {$P$};
      \node (b) [right=0pt of a]{$M$};
      \node (c) [right=0pt of b]{$P$};
      \node (d) [right=0pt of c]{$M$};
      \node (e) [right=0pt of d]{$P$};
      \node (f) [right=0pt of e]{$M$};
      \path[contraction] (a.north) edge [skip loop=4pt] (c.north);
      \path[contraction] (c.north) edge [skip loop=4pt] (e.north);
      \path[contraction] (b.north) edge [skip loop=8pt] (d.north);
      \path[contraction] (d.north) edge [skip loop=8pt] (f.north);
    \end{tikzpicture} \propto N_{t}^2\;,\\
    \sum_{t_1}\begin{tikzpicture}[baseline = .5ex]
      \node (a) {$P$};
      \node (b) [right=0pt of a]{$M$};
      \node (c) [right=0pt of b]{$P$};
      \node (d) [right=0pt of c]{$M$};
      \node (e) [right=0pt of d]{$P$};
      \node (f) [right=0pt of e]{$M$};
      \path[contraction] (a.north) edge [skip loop=6pt] (d.north);
      \path[contraction] (b.north) edge [skip loop=6pt] (e.north);
      \path[contraction] (c.north) edge [skip loop=6pt] (f.north);
    \end{tikzpicture} \propto N_{t}\;.
  \end{align}
\end{subequations}
%
Generally the power of $N_t$ is the same as the number of independent
contractions. This is only true when $N_t$ is large as the number of free
temporal slices must be large enough for the contractions to be separate. Since
we are interested in the $N_t \gg 1$ range, we will disregard contributions from
contractions not consisting of a single $PM$ pair.

\begin{table}
  \begin{center}
    \includegraphics{pmpmpm_contractions}
  \end{center}
  \caption{All contractions contributing to the $\mathcal{O}(\kappa^6)$ term
    $\tr(PMPMPM)$.}
  \label{tab-pmpmpm-contractions}
\end{table}

At this stage in the computation we will disperse with the notion that hops in
positive and negative spatial direction are distinguishable. In contrast to the
temporal hops, which get boosted by baryon chemical potential, there is no
asymmetry between positive and negative spatial hops. We therefore switch to a
notation which focus on the dominant pairings
%
{
\begin{equation}
  \tikzset{
    every node/.style = {anchor=south, text depth=0.35ex, inner sep=1pt}
  }
  \tr \begin{tikzpicture}[baseline={([yshift=-2pt]one.center)}]
    \node (one) {$X$};
    \node (two) [right=0pt of one] {$i$};
    \node (three) [right=0pt of two] {$Y$};
    \node (four) [right=0pt of three] {$i$};
  \end{tikzpicture} \;=\; 
  \tr \begin{tikzpicture}[baseline={([yshift=-2pt]one.center)}]
      \node (one) {$X$};
      \node (two) [right=0pt of one] {$P$};
      \node (three) [right=0pt of two] {$Y$};
      \node (four) [right=0pt of three] {$M$};
      \path[contraction] (two.north) edge [skip loop=6pt] (four.north);
    \end{tikzpicture} \;+\;
  \tr \begin{tikzpicture}[baseline={([yshift=-2pt]one.center)}]
      \node (one) {$X$};
      \node (two) [right=0pt of one] {$M$};
      \node (three) [right=0pt of two] {$Y$};
      \node (four) [right=0pt of three] {$P$};
      \path[contraction] (two.north) edge [skip loop=6pt] (four.north);
    \end{tikzpicture},
\end{equation}
}
%
in which $X$ and $Y$ symbolise the rest of the term, and is disallowed from
having spatial hops overlapping with the $i$\textsuperscript{th} pairing. Every
contracted $PM$ pair is labelled by an arbitrary symbol $i$, and is invariant
under relabelling. Of the terms in \tabref{tab-pmpmpm-contractions}, the six
terms which consists of pairings only can in this notation be reduced to three
%
\begin{equation} \label{eq:pmpmpm_pairing}
  \tr \tikz[baseline=-3pt] \matrix [matrix of math nodes,inner sep=1.5pt,ampersand replacement=\&]
    {1 \& 1 \& 2 \& 2 \& 3 \& 3 \\};, \hskip.5cm
  \tr \tikz[baseline=-3pt] \matrix [matrix of math nodes,inner sep=1.5pt,ampersand replacement=\&]
    {1 \& 2 \& 3 \& 3 \& 2 \& 1 \\};, \hskip.5cm
  \tr \tikz[baseline=-3pt] \matrix [matrix of math nodes,inner sep=1.5pt,ampersand replacement=\&]
    {1 \& 2 \& 3 \& 1 \& 2 \& 3 \\};.
\end{equation}

To be a bit more explicit we will quickly go through the notation at NLO. In the
old notation the kinetic determinant would read
%
\begin{equation}
  \det \qkin = 1 - \tr(PM) + \frac{1}{2} \big(\tr(PM)\big)^2 - \tr(PPMM) -
    \frac{1}{2} \tr(PMPM) + \mathcal{O}(\kappa^6)
\end{equation}
%
which would give
%
\begin{multline}
  \det \qkin = 
    1 - \frac{1}{2} \tr(1\,1) + \frac{1}{8} \tr(1\,1) \tr(2\,2) \\
    + \frac{1}{4}\tr (1\,2) \tr (1\,2) - \frac{1}{2} \tr(1\,1\,2\,2) +
    \mathcal{O}(\kappa^6, N_t^{-1}).
\end{multline}
%
This latter expression of course only contains non-zero terms while this is yet
to be done with the $P$ and $M$ notation. When counting terms one has to be
careful not to over count. E.g. the final term can be expanded into four terms
in the $PM$ notation
%
\begin{equation}
  \tr(PMPM), \hskip.5em \tr(PMMP), \hskip.5em \tr(MPPM), \hskip.5em \tr(MPMP),
\end{equation}
%
however one should note that the third term is identical to the second under
cyclic permutations and relabelling, and should therefore be discarded. The
separate traces are however distinct, and the invariant relabelling must be
considered on a trace by trace basis. The combinatorial prefactors $1/g$ can be
computed with the following formula
%
\begin{equation}
  \frac{1}{g}=
    \frac{%
      \tikz \node [anchor=north east,scale=0.75,align=center] {\# of unique cyclic \\ permutations of the traces};
    }{%
      \tikz \node {$n_2!n_4!\cdots{}n_N!\, 2^{n_2} 4^{n_4} \cdots N^{n_N}$};
    } \;.
\end{equation}
%
The numerator is the number of cyclic permutations within the traces that
remain different under relabelling, e.g. $\tr(1\,1\,2\,2)$ has two distinct
permutations, the one already mentioned and $\tr(1\,2\,2\,1)$. The $n_m$ in the
numerator is the number of trace factors with $m$ matrices. E.g.
$\tr(1\,2)\tr(1\,2)$ has $n_2 = 2, n_4 = 0, \dots, n_N = 0$, while
$\tr(1\,1\,2\,2)$ has $n_2 = 0, n_4 = 1, \dots, n_N = 0$.

\subsection{Dirac indices}

In this subsection we will study the spin structure of the terms dominating the
cold limit, and subsequently discover that it is in fact trivial in this limit.
In the limit of high baryon chemical potential, we see from
\meqrefs[-]{eq:static_prop_separation,eq:kminus_def} that the static propagator
simplifies to
%
\begin{equation}
  Q^{-1}_{\text{stat},t_y t_x} \approx (1-\kappa T^+)^{-1}_{t_y t_x}
    =\, \delta_{t_y t_x} + \frac{1+\gamma_0}{2} K^+_{t_y t_x} \,,
\end{equation}
%
in which the matrix $K$ has no spin dependence. We see from the definition of a
contraction \meqrefs[, ]{eq:contraction_def_pm,eq:contraction_def_pp} that
the $\delta_{t_y t_x}$ in $\qstat^{-1}$ would impose additional constraints on
the temporal index and will therefore produce terms which are subleading when
$N_t$ is large. The only exception is a contraction
%
\begin{equation}
  \begin{tikzpicture}[inline math]
    \node (one) {$P_{xy}$};
    \node (two) [right=0pt of one]{$M_{yz}$};
    \path[contraction] ([xshift=5pt]one.north west) edge [skip loop=6pt] ([xshift=7pt]two.north west);
  \end{tikzpicture} = P_{xy} M_{yz} \,\delta_{\vec{x}\vec{z}} \,\delta_{t_yt_z}
\end{equation}
%
in which the $\delta_{t_y t_x}$ condition of the $\qstat^{-1}$ matrix of the $M$ 
factor is already fulfilled. However this would apply no temporal movement and
is therefore disallowed because the hopping expansion of Wilson fermions does
not allow backtracking, $(1+\gamma_{\mu})(1-\gamma_{\mu}) = 0$. We can therefore
to leading order use the following expression for the static propagator
%
\begin{equation}
  Q^{-1}_{\text{stat},t_y t_x} 
    \begin{tikzpicture}[baseline={([yshift=-1.5pt]equal.center)}]
      \node (equal) {$=$};
      \node[above=0pt of equal] [scale=0.65,align=center] {leading\\[0pt]order};
    \end{tikzpicture}\,
    \frac{1+\gamma_0}{2} K^+_{t_y t_x} \,.
\end{equation}
%
Since the $K^{\pm}$ matrix has no spin structure, we can easily identify the full Dirac
trace of any contributing term as
%
\begin{equation}
  \tr_{xsc}(X_{xsc}) = \tr_s\big( (1+\gamma_0)
  (1\pm\gamma_{i})(1+\gamma_0)(1\pm\gamma_j) \cdots\big) \tr_{xc}(X'_{xc}).
\end{equation}
%
where every $P$ would contribute with a factor $(1+\gamma_0)(1+\gamma_i)$ and
every $M$ a factor $(1+\gamma_0)(1-\gamma_j)$. To shorten the notation we
introduce the alias $g_{\mu} = (1+\gamma_{\mu})$ and $\bar{g}_{\mu} =
(1-\gamma_{\mu})$. In this notation the spin trace of a polymer $X$ gives
%
\begin{equation}
  \tr_s (X'_s) = \tr_s(g_0 g_i g_0 \bar{g}_j \cdots)
\end{equation}
%
We can expand the first product
%
\begin{multline}
  \tr_s((1 + \gamma_0 \pm \gamma_i \pm \gamma_0 \gamma_i)g_0  \cdots) \\
    = \tr_s(g_0 \cdots) + \tr_s(\gamma_0g_0 \cdots)
    \pm \tr_s(\gamma_i g_0 \cdots) \pm \tr_s(\gamma_0\gamma_ig_0 \cdots).
\end{multline}
%
The Dirac matrices $\gamma_0$ and $\gamma_i$ anti-commute. Using this fact
together with $\gamma_0 g_0 = \gamma_0(1+\gamma_0) = (\gamma_0+1) = g_0$, we see
that the above expression simplifies
%
\begin{equation}
  \tr_s(g_0 \cdots) + \tr_s( g_0 \cdots )
    \pm \tr_s(\gamma_i g_0 \cdots ) \mp \tr_s(\gamma_i g_0 \cdots )
    = 2 \tr_s( g_0 \cdots ) .
\end{equation}
%
We therefore replace every $g_0 g_i$ and every $g_0 \bar{g}_j$ pair with a
factor $2$ until there is only one pair left. The trace of the final pair is
simply the dimension of the system, in this case $4$. The spin trace of the
large $N_t$ limit can thus be considered trivial
%
\begin{equation}
  \tr_s(X'_s) = \tr ( \underbrace{g_0 g_i g_0 \bar{g}_j \cdots}_{n \text{ pairs}})
   = 2^{n-1} \tr(g_0 g_i) = 2^{n+1}.
\end{equation}

\subsection{A comment on gauge corrections}

Before we finally state the N\textsuperscript{3}LO effective action for the cold
and dense we need to examine a final set of apparently low order graphs which
have yet to be discussed. They are the result of new geometries only possible to
construct with the insertion of plaquettes. The first of these appear at
$\mathcal{O}(\kappa^4 u)$ and is depicted to the far left of
\figref{fig-nontrivial-gauge}. One combination which gives this contribution is
%
\begin{equation}
  \kappa^4 u\, \int_{\mathrlap{U}\:} \chi(U) \tr_{xsc} (P_i P_j M_i M_j), \hskip.5cm \text{ for } i \neq j.
\end{equation}
%
This however restricts the four quark hops to occupy the same temporal slice and
therefore only gives a single factor of $N_t$ while the dominating
$\mathcal{O}(\kappa^4)$ terms give contributions proportional to $N_t^2$. It is
possible to shift the quark hops on the slices by inserting plaquettes as can be
seen in the two other graphs in \figref{fig-nontrivial-gauge}, and would give a
correction to the formula above
%
\begin{equation}
  \kappa^4 u \bigg(1 + 2 \frac{u - u^{N_t}}{1-u} \bigg)^4
    \int_{\mathrlap{U}\:} \chi(U) \tr_{xsc} (P_i P_j M_i M_j),
\end{equation}
%
which of course would be highly suppressed as $u$ is also a small parameter.
This specific geometry will indeed appear at $\mathcal{O}(\kappa^8)$ where the
plaquette can be replaced by a quark loop including windings which can shift the
links arbitrarily in the temporal coordinate.

\begin{figure}
  {\centering
    \includegraphics[width=.95\textwidth]{non_trivial_gauge_corrections}\par}
  \caption{The three first non trivial gauge corrections to
    $\mathcal{O}(\kappa^4)$}
  \label{fig-nontrivial-gauge}
\end{figure}

\subsection{The \texorpdfstring{$\mathcal{O}(\kappa^8)$}{O(k8)} effective action}


\section{Numerical evaluation}
