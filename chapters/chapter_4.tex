\chapter{The Effective Theory}

Having presented the challenges and difficulties in simulating strongly
interacting fermions, especially in the dense regime, we will introduce an
effective theory that tackles some of these problems, while reproducing the full
theory in a certain parameter region. We will see that although simulations of
the effective theory still suffer from the side effects of the sign problem
(complex actions), the essence of the sign problem is weak enough that
reweighting can be readily applied.

The work in this thesis builds on previous work with the effective theory while
pushing the derivation further and introducing analytic tools, which we will
cover in the next session.

In this chapter we will first introduce the effective theory before we introduce
two expansion schemes that facilitate the computation of the theory. These are
namely the \emph{character expansion}, mentioned in \secref{sec-group_intro},
the second is the \emph{hopping parameter expansion} for heavy fermions. We
round of the chapter with a discussion on the numerical evaluation of the
effective theory.

\section{The effective theory - \texorpdfstring{\it Introduction}{Introduction}}

The essence of the derivation of the effective theory is to integrate out some
of the degrees of freedom analytically. This will both lessen the burden for the
numerical evaluation, having fewer degrees of freedom left to vary, as well as
lessen the sign problem. The sign problem being milder due to the fact that
many, or as we will see, most, of the fluctuations cancel exactly, as they
should. In this we will integrate the spatial gauge links of the partition
function
%
\begin{equation}
  \mathcal{Z} = \int \prod_{x, \mu} \mathrm{d} U_{\mu}(x) \, \det Q \, [U_{\mu}] \,
    e^{-\mathcal{S}_g[U_{\mu}]}
    \equiv \int \prod_{x} \mathrm{d} U_0(x) \,
    e^{-\mathcal{S}_{\text{eff}}[U_0]},
\end{equation}
%
which defines the effective action to be
%
\begin{equation}
  \mathcal{S}_{\text{eff}} = - \log \int \prod_{x, i} \mathrm{d} U_i(x) \, \det
    Q \, [U_{\mu}] \, e^{-\mathcal{S}_g [U_{\mu}]}.
\end{equation}
%
The integrals over the spatial gauge links $U_i(x)$ is unfortunately not
something we can evaluate analytically without the aid of approximations. We
will therefore introduce two expansion schemes and work towards deriving the
effective theory such that reproduce the exact expansion coefficients of the
full theory in the end.

\section{The character expansion} \label{sec-char_exp}

The first expansion we will apply is the character expansion introduced in 
\secref{sec-group_intro}. In the form of an exact equality, it is not of much
help. Nevertheless, from the character expansion of the single plaquette gauge
contribution
\footnote{We will from this point forward assume that the gauge group is SU$(N_c)$
  and that the fermions transform under the fundamental representation unless
  stated otherwise.}
%
\begin{equation}
  e^{-\beta (1 - \frac{1}{N_c} \text{Re} \tr U_p)} = u_0(\beta) \bigg(1 +
  \sum_{r \neq 0} d_r\, u_r(\beta)\, \chi_r(U_p) \bigg),
\end{equation}
%
we see that the character expansion coefficients are dependent on the lattice
gauge coupling $\beta$. It can be easily seen that the higher dimensional
representations come with a higher power of this coupling. A natural ordering
therefore arises if one expands around the infinite coupling limit, $g\to\infty$,
$\beta\to0$. This expansion scheme is aptly named the \emph{strong coupling
  expansion}, and has been the focus of numerous studies for the past decades,
also having picked up in recent years by groups studying conformal field
theories. Introductions to the field can be found in \cite{Drouffe:1983fv} and
\cite{montvay1997quantum}.

The lowest order character expansion coefficient, namely that of the fundamental
representation, has for SU$(3)$ been calculated to high orders
%
\begin{align}
  u_f(\beta) &= \frac{1}{N_c} \frac{\int \mathrm{d} g\, \tr g\, e^{-\frac{\beta}{2 N_c}
    ( \tr g + \tr g^{\dagger})}}{\int \mathrm{d} g\: e^{-\frac{\beta}{2 N_c}
    ( \tr g + \tr g^{\dagger})}} \nonumber\\
  &\hskip1em= \frac{
    x + \frac{1}{2} x^2 + x^3 + \frac{5}{8} x^4 + \frac{13}{24} x^5 + \mathcal{O}(x^6)%
  }{%
    1 + x^2 + \frac{1}{3} x^3 + \frac{1}{2} x^4 + \frac{1}{4} x^5 + \mathcal{O}(x^6)%
  }, \hskip2ex x=\frac{\beta}{2 N_c}.
\end{align}
%
To leading order $u_f(\beta) \approx \frac{\beta}{2 N_c^2}$, and we
therefore use $u_f$ as our expansion parameter rather than $\beta$. The
character expansion only permits a single plaquette from any representation to
be placed at every position, making order counting easier than a standard Taylor
expansion of the gauge action.

\section{Pure gauge effective theory}
\section{The hopping parameter expansion}
\section{The effective theory - \texorpdfstring{\it Calculated}{Calculated}}
\section{Numerical evaluation}
