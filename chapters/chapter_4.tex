\chapter{The Effective Theory}

Having presented the challenges and difficulties in simulating strongly
interacting fermions, especially in the dense regime, we will introduce an
effective theory that tackles some of these problems, while reproducing the full
theory in a certain parameter region. We will see that although simulations of
the effective theory still suffer from the side effects of the sign problem
(complex actions), the essence of the sign problem is weak enough that
reweighting can be readily applied.

The work in this thesis builds on previous work with the effective theory while
pushing the derivation further and introducing analytic tools, which we will
cover in the next session.

In this chapter we will first introduce the effective theory before we introduce
two expansion schemes that facilitate the computation of the theory. These are
namely the \emph{character expansion}, mentioned in \secref{sec-group_intro},
the second is the \emph{hopping parameter expansion} for heavy fermions. We
round of the chapter with a discussion on the numerical evaluation of the
effective theory.

\section{The effective theory - \texorpdfstring{\itshape Introduction}{Introduction}}

The essence of the derivation of the effective theory is to integrate out some
of the degrees of freedom analytically. This will both lessen the burden for the
numerical evaluation, having fewer degrees of freedom left to vary, as well as
lessen the sign problem. The sign problem being milder due to the fact that
many, or as we will see, most, of the fluctuations cancel exactly, as they
should. In this we will integrate the spatial gauge links of the partition
function
%
\begin{equation}
  \mathcal{Z} = \int \prod_{x, \mu} \mathrm{d} U_{\mu}(x) \, \det Q \, [U_{\mu}] \,
    e^{-\mathcal{S}_g[U_{\mu}]}
    \equiv \int \prod_{x} \mathrm{d} U_0(x) \,
    e^{-\mathcal{S}_{\text{eff}}[U_0]},
\end{equation}
%
which defines the effective action to be
%
\begin{equation}
  \mathcal{S}_{\text{eff}} = - \log \int \prod_{x, i} \mathrm{d} U_i(x) \, \det
    Q \, [U_{\mu}] \, e^{-\mathcal{S}_g [U_{\mu}]}.
\end{equation}
%
The integrals over the spatial gauge links $U_i(x)$ is unfortunately not
something we can evaluate analytically without the aid of approximations. We
will therefore introduce two expansion schemes and work towards deriving the
effective theory such that reproduce the exact expansion coefficients of the
full theory in the end.

\section{The character expansion} \label{sec-char_exp}

The first expansion we will apply is the character expansion introduced in 
\secref{sec-group_intro}. In the form of an exact equality, it is not of much
help. Nevertheless, from the character expansion of the single plaquette gauge
contribution
\footnote{We will from this point forward assume that the gauge group is SU$(N_c)$
  and that the fermions transform under the fundamental representation unless
  stated otherwise.}
%
\begin{equation}
  e^{-\beta (1 - \frac{1}{N_c} \text{Re} \tr U_p)} = u_0(\beta) \bigg(1 +
  \sum_{r \neq 0} d_r\, u_r(\beta)\, \chi_r(U_p) \bigg),
\end{equation}
%
we see that the character expansion coefficients are dependent on the lattice
gauge coupling $\beta$. It can be easily seen that the higher dimensional
representations come with a higher power of this coupling. A natural ordering
therefore arises if one expands around the infinite coupling limit, $g\to\infty$,
$\beta\to0$. This expansion scheme is aptly named the \emph{strong coupling
  expansion}, and has been the focus of numerous studies for the past decades,
also having picked up in recent years by groups studying conformal field
theories. Introductions to the field can be found in \cite{Drouffe:1983fv} and
\cite{montvay1997quantum}.

The lowest order character expansion coefficient, namely that of the fundamental
representation, has for SU$(3)$ been calculated to high orders
%
\begin{align}
  u_f(\beta) &= \frac{1}{N_c} \frac{\int \mathrm{d} g\, \tr g\, e^{-\frac{\beta}{2 N_c}
    ( \tr g + \tr g^{\dagger})}}{\int \mathrm{d} g\: e^{-\frac{\beta}{2 N_c}
    ( \tr g + \tr g^{\dagger})}} \nonumber\\
  &\hskip1em= \frac{
    x + \frac{1}{2} x^2 + x^3 + \frac{5}{8} x^4 + \frac{13}{24} x^5 + \mathcal{O}(x^6)%
  }{%
    1 + x^2 + \frac{1}{3} x^3 + \frac{1}{2} x^4 + \frac{1}{4} x^5 + \mathcal{O}(x^6)%
  }, \hskip2ex x=\frac{\beta}{2 N_c}.
\end{align}
%
To leading order $u_f(\beta) \approx \frac{\beta}{2 N_c^2}$, and we
therefore use $u_f$ as our expansion parameter rather than $\beta$. The
character expansion only permits a single plaquette from any representation to
be placed at every position, making order counting easier than a standard Taylor
expansion of the gauge action.

\section{Pure gauge effective theory}

With the character expansion at hand we can evaluate the pure gauge
contributions to the effective action. Ignoring the quark contribution the
effective action is
%
\begin{equation}
  e^{-\mathcal{S}_{\text{eff}}} = \int \prod_{x, i} \mathrm{d} U_i(x) \, \prod_p
    \bigg(1 + \sum_{r \neq 0} d_r u_r(\beta) \chi_r (U_p) \bigg).
\end{equation}
%
Expanding the product over the plaquettes gives a sum of terms which is of the
form
%
\begin{equation}
  d_{r_1} u_{r_1}(\beta) \chi_{r_1}(U_{p_1}) \, 
  d_{r_2} u_{r_2}(\beta) \chi_{r_2}(U_{p_2}) \, \cdots
\end{equation}
%
where if one or more of the plaquettes in a term has a link that fall on
$(x,\mu)$, gives the integral
%
\begin{equation}
  \int \mathrm{d} U_{\mu}(x) \, \chi_{r_1} \big(U_{s_1} U_{\mu}(x) \big) \,
    \chi_{r_2} \big(U_{s_2} U_{\mu}(x) \big) \, \cdots,
\end{equation}
%
where $U_{s_i}$ is the remaining \emph{staple} after the link $U_{\mu}(x)$ has
been factored out of the plaquette. One approach to solving these integrals is
to write carry out the Kronecker product of the representation matrices and
decompose them to their irreducible representations using the Clebsch-Gordan
coefficients. We see that only products whose Clebsch-Gordan series contains the
trivial representation vanish due to the identity
%
\begin{equation}
  \int \mathrm{d} g \, \chi_r (g) = \delta_{r,0}.
\end{equation}
%
On top of restricting the valid plaquette combinations sharing a link, it also
restricts the graphs created from combining plaquettes to ones that have no
boundaries. On an infinite lattice the lowest order contribution would therefore
come from combining six fundamental plaquettes into a cube.

For finite lattices the periodic boundary can be utilised to create closed
surfaces. In fact, only graphs periodic in the temporal direction give
contribution to the finite temperature as the non-periodic ones can be
normalised out. Since we only integrate spatial links the contributing graphs
need only have closed surfaces in the spatial directions. The lowest order
contribution to the effective action comes from a strip of plaquettes spanning
the temporal direction as shown in \figref{fig-plaquette-strip}. Since only two
links meet at all the spatial sites we only need the integral
%
\begin{equation}
  \int \mathrm{d} U \: \chi_r(V U) \: \chi_s(W U^{-1})
    = \delta_{r,s} \frac{1}{d_r} \chi_r (V W),
\end{equation}
%
which can be represented graphically as
%
\begin{equation}
  \int \mathrm{d} U \;%
  \begin{tikzpicture}[baseline={(base)}]
    \coordinate (base) at (0,.35);
    \draw[link line] (0,0) rectangle (2,1);
    \draw[link] (0,0) -- (0,1)
      node[midway,left=1mm,inner sep=0pt,ColourBase,scale=0.8] {$V$};
    \draw[link] (2,1) -- (2,0)
      node[midway,right=1mm,inner sep=0pt,ColourBase,scale=0.8] {$W$};
    \draw[link double] (1,0) -- (1,1)
      node[midway,right=1mm,inner sep=0pt,ColourBase,scale=0.8] {$U$};
  \end{tikzpicture}
  \,=\, \frac{1}{d_r} \;
  \begin{tikzpicture}[baseline={(base)}]
    \coordinate (base) at (0,.35);
    \draw[link line] (0,0) rectangle (2,1);
    \draw[link] (0,0) -- (0,1)
      node[midway,left=1mm,inner sep=0pt,ColourBase,scale=0.8] {$V$};
    \draw[link] (2,1) -- (2,0)
      node[midway,right=1mm,inner sep=0pt,ColourBase,scale=0.8] {$W$};
  \end{tikzpicture} \,.
\end{equation}

\begin{figure}
  {\centering
    \includegraphics[width=.75\textwidth]{pure_gauge_strip}\par}
  \caption{Lowest order pure gauge contribution to the effective action}
  \label{fig-plaquette-strip}
\end{figure}

Integrating out the spatial links of the strip of plaquettes leaves two
disconnected loops at the neighbouring spatial lattice sites
%
\begin{equation}
  e^{-\mathcal{S}_{\text{eff}}} = 1 + \sum_{\langle \vec{x}, \vec{y} \rangle} u_f^{N_t}
  \big( L(\vec{x}) L^*(\vec{y}) + L^*(\vec{x}) L(\vec{y}) \big) + \mathcal{O}(u_f^{N_t+1})
\end{equation}
%
where $L$ is the so-called \emph{Polyakov loop}
%
\begin{equation}
  L(\vec{x}) = \tr \prod_{\mathclap{t=0}}^{\mathclap{N_t-1}} U_0(\vec{x},t) .
\end{equation}
%
We see that the explicit time dependence of the links have disappeared as the
only degrees of freedom left are full windings. The integral over the effective
action therefore simplify to
%
\begin{equation}
  \mathcal{Z}_{\text{eff}} = \int \prod_{\vec{x},t} \mathrm{d} U_0(\vec{x},t) \,
    e^{-\mathcal{S}_{\text{eff}}[L]} 
  = \int \prod_{\vec{x}} \mathrm{d} L(\vec{x}) \, \sqrt{\det U_0} \,
    e^{-\mathcal{S}_{\text{eff}}[L]} ,
\end{equation}
%
where $\sqrt{\det U_0}$ is the Haar measure of the group, the calculation of
which will be covered in \apxref{sec-haar_measure}. As one can see the effective
theory is a three dimensional theory of Polyakov loop interactions. At first
order we have a nearest neightbour spin system with an effective coupling
$u_f^{N_t}$.

At higher orders in $\beta$ new effects are introducted through interactions
between loops at higher order representations, next to nearest neighbour
interactions as well as corrections to the nearerst neighbour coupling between
fundamental Polyakov loops. The effects of higher order representations in
Polyakov loop effective theories was studied in \citep{Wozar:2007tz}. The
corrections to the fundamental nearest neighbour coupling was calculated to 
$\mathcal{O}(u_f^{N_t + 10})$ in \citep{Langelage:2010yr} while the effects of
long range interactions was examined in \citep{Bergner:2015rza}.

We will leave the topic of pure gauge effective theories for now as the work in
this thesis is mostly concerned with the cold and dense regime, in which pure
gauge corrections are negligible.

\section{The hopping parameter expansion}
\section{The effective theory - \texorpdfstring{\itshape Calculated}{Calculated}}
\section{Numerical evaluation}
