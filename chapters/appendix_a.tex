\chapter{Analytical Tools For SU(\texorpdfstring{$N$}{N})}

The analytic computations contained herein revolve around various SU$(N)$
dependent quantities, and we therefore need basis for computing these. In this
appendix we will look at how to compute all the necessary components such as the
Haar measure in \secref{sec-haar_measure}, the character integrals in
\secref{sec-character_integrals}, integrals over group polynomials in
\secref{sec-sun_integrals} and finally the fermionic functions in
\secrefs{sec-evaluating-fermion-determinants,sec-evaluating-polyakov-coupling-terms}.

\section{Computing the Haar measure} \label{sec-haar_measure}

We want the invariant measure of a group element expressed in terms of its
spectral decomposition. Any matrix representation of an U$(N)$ matrix has
unitary eigenvalues $\lambda_i = e^{i\theta_i}$ only, where $\sum_i \theta_i =
0 \bmod 2\pi$ for the elements of SU$(N)$. We write a matrix $U$ that is a representation
of U$(N)$ in its decomposed form
%
\begin{equation}
  U = W \Lambda W^{\dagger},
\end{equation}
%
where $\Lambda = \diag (e^{i\theta_1}, ..., e^{i\theta_n})$, and $W W^{\dagger}
= \mathbb{1}_R$.  A change in $U$ is therefore
%
\begin{equation}
  \mathrm{d}U = W \mathrm{d} \Lambda W^{\dagger} +  W \big[ W^{\dagger} \mathrm{d}
  W, \Lambda \big] W^{\dagger}.
\end{equation}
%
Using the standard unitary matrix metric, $||M||^2 = \tr(M^{\dagger} M)$, we get
the size of an infinitesimal square to be
%
\begin{equation} \label{eq-infinitesimal-group-square}
  \mathrm{d} s^2 = \tr (\mathrm{d} U^{\dagger} \mathrm{d} U)
  = \sum_i \mathrm{d} \theta_i^2 + 2 \sum_{i > j}
    \big|e^{i \theta_i} - e^{i\theta_j}\big|^2
    \big| (W^{\dagger} \mathrm{d} W)_{ij} \big|^2
\end{equation}
%
Since one identifies the set of coordinates to the metric through
$\mathrm{d} s^2 = g_{\mu\nu} \mathrm{d} \xi^{\mu} \mathrm{d} \xi^{\nu}$,
it is natural to construct the measure as
%
\begin{equation} \label{eq-natural-group-measure}
  \mathrm{d} U(x) = \sqrt{\det g(x)} \prod_{\mu} \mathrm{d}\xi^{\mu}.
\end{equation}
%
To reconcile these two formulae we use the Baker-Campbell-Hausdorff formula to
express $W^{\dagger} \mathrm{d} W$ in terms of the generators of U$(d_R)$
%
\begin{equation}
  (W^{\dagger} \mathrm{d} W)_{ij} = i (T_a)_{ij} f_{a b}(w) \mathrm{d} w_b
  \equiv Q_{ijk}(w) \mathrm{d} w_k
\end{equation}
%
with $f$ being some function of the parameters of the generator decomposition.
The general coordinates will therefore be a combination of the eigenvalue angles
$\theta_i$ and the parameters $w_i$. The determinant of the metric factorise
into parts dependent on these coordinates, and we get
%
\begin{equation}
  \det g(x) \sim \big(\det Q(w)\big)^2 \prod_{i>j} \big|e^{i\theta_i} -
  e^{i\theta_j}\big|^4.
\end{equation}
%
Since we are mostly concerned with the integrals over the eigenvalues only, one
can integrate out the $w_i$, and get the invariant measure
%
\begin{equation}
  \mathrm{d} U = \underbrace{\prod_{i>j} \big| e^{i\theta_i} - e^{i\theta_j} \big|^2}_{H(U)} \prod_i
  \mathrm{d} \theta_i,
\end{equation}
%
where we have identified the Haar measure, $H(U)$. Although this relation is
only true for U$(N)$, it can be trivially extended to SU$(N)$ through the
previously mentioned restriction, $\sum_i \theta_i = 0 \bmod 2\pi$.

To facilitate the calculation of the Haar measure, we introduce the Vandermonde
determinant
%
\begin{equation}
  \prod_{i>j} (z_i - z_j) = \det \mathcal{M} =
  \begin{vmatrix}
    1 & z_1 & \cdots & z_1^{N-1} \\
    1 & z_2 & \cdots & z_2^{N-1} \\
    \vdots & \vdots & \ddots & \vdots \\
    1 & z_N & \cdots & z_N^{N-1} 
  \end{vmatrix}.
\end{equation}
%
This relation one can use to rewrite the Haar measure
%
\begin{equation}
  H(U) = \prod_{i>j} \big| z_i - z_j \big|^2
   = \det \mathcal{M}^{\dagger} \mathcal{M} = \scalemath{0.8}{%
  \begin{vmatrix}
    N & \sum_i z_i & \sum_i z_i^2 & \cdots & \sum_i z_i^{N-1} \\
    \sum_i z_i^{\dagger} & N & \sum_i z_i & \cdots & \sum_i z_i^{N-2} \\
    \vdots & \vdots & \vdots & \ddots & \vdots \\
    \sum_i z_i^{\dagger N-1} & \sum_i z_i^{\dagger N-2} & \sum_i z_i^{\dagger N-3} & \cdots & N
  \end{vmatrix}}.
\end{equation}
%
For the fundamental representation of SU$(N)$, one has $N - 1$ independent
eigenvalues. We know that
%
\begin{equation}
  \sum_i z_i^m = \tr U^m \equiv \chi_m,
\end{equation}
%
correspond to the $m$'th powered character. One can therefore employ the
Caylay-Hamilton theorem to determine how these angles depend on each other.
This equation reads
%
\begin{equation} \label{eq-caylay-hamilton}
  M^N + c_{N-1} M^{N-1} + \cdots + c_1 M + (-1)^N \det M\: \mathbb{1}_R = 0,
\end{equation}
%
where the coefficients are
%
\begin{equation}
  c_i = \sum_{\{k_n\}} \prod_{l=1}^N \frac{(-1)^{k_l+1}}{l^{k_l} k_l!}
  \tr(M^l)^{k_l},
\end{equation}
%
in which the sum goes over all integer partitions $\{k_n\}$ so that
%
\begin{equation}
  \sum_{l=1}^N l k_l  = N - i.
\end{equation}
%
Since we know that $\det U = 1$ and $U^{\dagger} U = \mathbb{1}$ for all
matrices in SU$(N)$, \meqref{eq-caylay-hamilton} can be multiplied by powers of
$U^{\dagger}$ and traced to find relations to reduce the number of unknowns in
the Haar measure.

As an example, for SU$(3)$, $\chi_2 = \chi^2 - 2 \chi^*$, giving
%
\begin{equation} \label{eq-su3-haar-measure}
  H(U) =
  \begin{vmatrix}
    3 & \chi & \chi^2 - 2 \chi^* \\
    \chi^* & 3 & \chi \\
    \chi^{*2} - 2 \chi & \chi^* & 3
  \end{vmatrix} = 27 - 18\, |\chi|^2  + 8 \mathrm{Re}\, \chi^3 - |\chi|^4.
\end{equation}

\section{\texorpdfstring{$\chi_r \chi_s$}{Ln Lm} integrals} \label{sec-character_integrals}

One type of integrals we will encounter often are integrals of the form
%
\begin{equation}
  I_{nm} = \int \mathrm{d} U \, \chi(U)^n \chi(U^{\dagger})^m.
\end{equation}
%
These can be most easily calculated through their trigonometric decomposition,
using the fact that
%
\begin{equation}
  \left. \renewcommand{\arraystretch}{1.75}
  \begin{array}{c}
    \chi(U) = {\textstyle\sum_{\alpha=1}^N} e^{i \theta_\alpha} \\
    \chi(U^{\dagger}) = {\textstyle\sum_{\alpha=1}^N} e^{- i \theta_\alpha}
  \end{array}\right\}\kern-\nulldelimiterspace
  \hskip1em {\textstyle\sum_{\alpha=1}^N} \theta_{\alpha} = 0 \bmod 2\pi
\end{equation}
%
together with
%
\begin{equation} \label{eq-oscillating-integral}
  \int_{\mathrlap{-\pi}}^{\pi} \mathrm{d}\alpha \, e^{in \alpha} = \delta(n),
  \text{ for } n \in \mathbb{Z}
\end{equation}
%
We thus get
%
\begin{equation}
  I_{nm} = \int \big[ \mathrm{d} \theta \big]_i\, H(U)
  \delta\big({\textstyle\sum}\theta_i = 0\big)
  \sum_{\{n_i\}} \sum_{\{m_i\}} \prod_{k=1}^n
  \prod_{l=1}^m \frac{k l}{n_k! m_l!} e^{i n_k \theta_k - i m_l \theta_l}
\end{equation}
%
with the sum over integer partitions so that
%
\begin{equation}
  \sum_{i=1}^N (n,m)_i = (n,m).
\end{equation}
%
Because of \meqref{eq-oscillating-integral}, only the combinations where the sum
over the angles in the exponent equals zero will survive the integral, and
therefore calculating the integral reduces to counting all these combinations.

\subsection{Integrals over characters of SU\texorpdfstring{$(3)$}{(3)}}

Since this is not a problem that is solvable in general, we will specialise on
the group SU$(3)$. In this case we only have two free angles, let's call them
$\theta$ and $\phi$. The integral is thus
%
\begin{align}
  &I_{nm} = \int \mathrm{d} \theta \mathrm{d} \phi \, H(\theta,\phi)
  \big(e^{i\theta} + e^{i\phi} + e^{-i(\theta + \phi)}\big)^n
  \big(e^{-i\theta} + e^{-i\phi} + e^{i(\theta + \phi)}\big)^m \nonumber\\
  &=\int \mathrm{d} \theta \mathrm{d} \phi \, H(\theta,\phi)
  \sum_{\{l_i\}} \sum_{\{k_i\}} \frac{n!}{l_1! l_2! l_3!} \frac{m!}{k_1! k_2!
    k_3!} e^{i \theta (l_1 - l_3 - k_1 + k_3) + i \phi (l_2 - l_3 - k_2 + k_3)}.
\end{align}
%
If we ignore the Haar measure and define a new integral, $J_{nm} = \int
[\mathrm{d} \theta]_i\, \chi^n \chi^{*m}$, we see by the arguments at the previous
section that this integral is
%
\begin{equation}
  J_{nm} = \sum_{\{l_i\}} \sum_{\{k_i\}}
    \frac{n!}{l_1! l_2! l_3!} \frac{m!}{k_1! k_2!k_3!}\,
    \delta\big(l_1 - l_3 - k_1 + k_3\big)\, \delta\big(l_2 - l_3 - k_2 + k_3\big).
\end{equation}
%
We carry out the explicit sum over four of the variables, and are left
with
%
\begin{equation}
  J_{nm} = (2\pi)^2 \sum_{l = 0}^n \sum_{k = 0}^m
    \frac{n!}{l!\, (k + \frac{n-m}{3})!\,(\frac{2n+m}{3} - l - k)!}
    \frac{m!}{k!\, (l + \frac{m-n}{3})!\,(\frac{2m+n}{3} - l - k)!}
\end{equation}
%
where the remaining variables $l$ and $k$ are restricted by the requirement that
the factorials are all non negative. Since the Haar measure itself is expressed
in terms of the characters, using \meqref{eq-su3-haar-measure} we see that the
full integral is
%
\begin{equation}
  I_{n,m} = 27\, J_{n,m} - 18\, J_{n+1,m+1} + 4\, J_{n+3,m} + 4\, J_{n,m+3} - J_{n+2,m+2}.
\end{equation}
%
From the equation for $J_{nm}$, is is easy to see that it is symmetric in its
indices, $J_{nm} = J_{mn}$ , and by extension, so is $I_{nm}$. Since all the
quantities of interest are normalised, we introduce the normalised integral,
$\tilde{I}_{nm} = I_{nm} / I_{00}$, the value of some of which are listed in 
\tabref{tab-character-integrals}. Analysing the sequences one notices that they
have deep combinatorial connections \citep{OEIS}. E.g. the first row is the sequence of
permutations of $S_n$ with longest increasing subsequence length $<=3$
(A005802). The first column on the other hand corresponds to the 3 dimensional
Catalan numbers (A005789). A more thorough review of the combinatorics of
SU$(3)$ can be found in \citep{Unger:2014oga}.

\begin{table}
  \begin{center}
    \begin{tabular}{*8c} \toprule
      {$\tilde{I}_{nm}$} & $\mathbb{1}$ & $\chi\chi^*$ & $(\chi\chi^*)\mathrlap{^2}$ & $(\chi\chi^*)\mathrlap{^3}$ 
        & $(\chi\chi^*)\mathrlap{^4}$ & $(\chi\chi^*)\mathrlap{^5}$ & $(\chi\chi^*)\mathrlap{^6}$ \\\midrule
      $\mathbb{1}$ & 1 & 1 & 2 & 6 & 23 & 103 & 513 \\
      $\chi^3$ & 1 & 3 & 11 & 47 & 225 & 1\,173 & 6529 \\
      $\chi^6$ & 5 & 21 & 98 & 498 & 2\,709 & 15\,565 & 93\,500 \\
      $\chi^9$ & 42 & 210 & 1\,122 & 6\,336 & 37\,466 & 230\,230 & 1\,461\,330 \\
      $\chi^{12}$ & 462 & 2\,574 & 15\,015 & 91\,091 & 571\,428  & 3\,688\,932 & 24\,410\,334 \\
      $\chi^{15}$ & 6\,006 & 36\,036 & 223\,652 & 1\,429\,428 & 9\,372\,168 & 62\,833\,836 & 429\,568\,036 \\\bottomrule
    \end{tabular}
  \end{center}
  \caption{Some integrals over the characters of SU$(3)$ where the integrand is made
    up of the product of the topmost row with the leftmost column. All other
    integrals are zero due to the selection rule of
    \protect\meqref{eq-integral-selection-rule}.}
  \label{tab-character-integrals}
\end{table}

\subsection{Characters of plaquettes}

Another type of integrals that appear often are those of characters over
plaquettes sharing a common line, like the ones in
\secref{sec-pure-gauge-theory}. These are integrals of the form
%
\begin{equation}
  F_{nm}^r(V,W) = \int \mathrm{d} U \, \chi_r(VU)^n \chi_r(WU^{\dagger})^m.
\end{equation}
%
In \citep{Bars:1979xb} the authors introduced a recursive formula to evaluate
the diagonal integrals on this form in the fundamental representation of U$(N)$,
namely the integrals $F_{nn}^f(V,V^{\dagger})$. The procedure goes as follows. First one
introduces a new integral
%
\begin{equation}
  G_n(\{V_i\}) = \int \mathrm{d} U \> \chi_f(V_1 U) \cdots \chi_f(V_n U) \,
    \chi_f(V_1^{\dagger} U^{\dagger}) \cdots \chi_f(V_n^{\dagger} U^{\dagger}),
\end{equation}
%
and verifies that by introducing the special form $(A_n)_{kl} = \delta_{ik}
\delta_{lq}$, $(A_n^{\dagger})_{kl} = \delta_{qk} \delta_{lj}$ for some integers
$i,j,q$, the final factors of $G$ simplify when summing over $k$
%
\begin{equation}
  {\textstyle\sum_k} G_n(\{V_i\}) = \delta_{ij} G_{n-1}(\{V_i\}').
\end{equation}
%
Due to the properties of the integrand of $G$, it has to be U$(N)$ $\times$
U$(N)$ invariant, and we can therefore decompose any $G$ into its invariant
combinations. Examples of the two lowest orders give
%
\begin{align}
  G_1(A_1) &= \int \mathrm{d} U \> \chi_f(A_1 U) \, \chi_f(A_1^{\dagger} U^{\dagger})
   = C_1 \> \chi_f(A_1 A_1^{\dagger}), \\
  G_2(A_1,A_2) &= \int \mathrm{d} U \> \chi_f(A_1 U) \chi_f(A_2 U) 
  \chi_f(A_1^{\dagger} U^{\dagger}) \chi_f(A_2^{\dagger} U^{\dagger})  \nonumber \\
  &= C_1 \big( \chi_f(A_1 A_1^{\dagger}) \chi_f(A_2 A_2^{\dagger}) + \chi_f(A_1
  A_2^{\dagger}) \chi_f(A_2 A_1^{\dagger})\big) \nonumber \\
  &\hskip2em+ C_2 \big(\chi_f(A_1 A_1^{\dagger} A_2 A_2^{\dagger})
  + \chi_f(A_1 A_2^{\dagger} A_2 A_1^{\dagger}) \big).
\end{align}
%
One then inserts the special form for $A_n$ on the left and right hand side of
the equations and systematically determine the unknown parameters $C_i$.

\section{\texorpdfstring{$g^n (g^{-1})^m$}{Un Udm} integrals} \label{sec-sun_integrals}

In this section we will present a method of computing integrals over
matrix representations of the SU$(N)$ members. Namely integrals of the form
\meqref{eq-group-integrals-representation}
%
\begin{equation}
  I_{i_1,j_1,...,i_n,j_n}^{k_1,l_1,...,k_m,l_m} = \int \mathrm{d} U \,
    U_{i_1j_1} \cdots U_{i_nj_n} U^{\dagger}_{k_1l_1} \cdots U^{\dagger}_{k_ml_m}.
\end{equation}
%
Although there are multiple ways to approach this, we will closely follow the
paper of \citep{Creutz:1977yy,Creutz:1978ub}. First we introduce the generating
functional
%
\begin{equation}
  \mathcal{W}(J,K) = \int \mathrm{d} U \, \exp \Big( \tr \big(J U + K U^{\dagger}\big) \Big)
\end{equation}
%
noting that we can use it to reexpress $I$
%
\begin{equation}
  I_{i_1,j_1,...,i_n,j_n}^{k_1,l_1,...,k_m,l_m} = \bigg(
    \frac{\delta}{\delta J_{i_1j_1}} \cdots \frac{\delta}{\delta J_{i_nj_n}}
    \frac{\delta}{\delta K_{k_1l_1}} \cdots \frac{\delta}{\delta K_{k_ml_m}}
    \bigg) \mathcal{W}(J,K) \bigg|_{J=K=0}.
\end{equation}
%
We then use the cofactor expansion of $U^{\dagger}$ to remove the dependence on
$K$
%
\begin{align}
  U^{\dagger}_{ij} &= \frac{1}{\det U} \big( \cof U^T \big)_{ij} \nonumber\\
  &= \frac{1}{(N-1)!} \epsilon_{j, i_1, \dots, i_{N-1}} \epsilon_{i, j_1, \dots, j_{N-1}}
    U_{i_1j_1} \cdots U_{i_{N-1}j_{N-1}}.
\end{align}
%
where we have introduced the completely anti-symmetric tensors $\epsilon$.
With this we can factor out the dependence of $K$ on the generating function
%
\begin{equation}
  \mathcal{W}(J,K) = \exp\bigg( \tr \Big( K \cof \frac{\delta}{\delta J} \Big) \bigg)
    \underbrace{\,\int \mathrm{d} U \, \exp \big( \tr (JU) \big)\,}_{\mathcal{W}(J)}.
\end{equation}
%
Using a determinant expansion of $\mathcal{W}(J)$, which is allowed due to the fact that
$\mathcal{W}(VUW) = \mathcal{W}(U)$ for arbitrary matrices $W, V \in
\text{SU}(3)$ \citep{Creutz:1977yy}, we finally obtain
%
\begin{equation}
  \mathcal{W}(J) = \sum_{i=0}^{\infty} \frac{2! \cdots (N-1)!}{i! \cdots
    (i+N-1)!} \big(\det J\big)^i.
\end{equation}
%
Although this expression is a little unwieldy, we can see an immediate selection
rule for the integrals. Using the fact that the determinant of an $N \times N$ matrix is a
polynomial of $N$'th power of its elements
%
\begin{equation}
  \det M = \frac{1}{N!} \epsilon_{i_1,\dots,i_N} \epsilon_{j_1, \dots, j_N}
  M_{i_1j_1} \cdots M_{i_Nj_N},
\end{equation}
%
we see that only factors of $N$'th power derivatives of $\mathcal{W}$ will
survive when we set $J=0$. Since the cofactor is a $N-1$'th power polynomial we
get the selection rule
%
\begin{equation} \label{eq-integral-selection-rule}
  \int \mathrm{d} U \, U^n U^{\dagger m} \neq 0  \hskip1em\iff\hskip1em n + (N-1)m = 0 \pmod N.
\end{equation}
%
Finally we will compute the most commonly used integral in this thesis
%
\begin{align}
  I_{ij}^{kl} &= \int \mathrm{d} U \, U_{ij} U^{\dagger}_{kl} \nonumber\\
  &= \frac{1}{(N-1)!} \epsilon_{l,i_1,\dots,i_{N-1}} \epsilon_{k, j_1, \dots, j_{N-1}}
    \int \mathrm{d} U \, U_{ij} U_{i_1j_1} \cdots U_{i_{N-1}j_{N-1}} \nonumber\\
  &= \frac{1}{(N-1)!} \epsilon_{l,i_1,\dots,i_{N-1}} \epsilon_{k, j_1, \dots, j_{N-1}}
    \bigg( \frac{\delta}{\delta J_{ij}} \frac{\delta}{\delta J_{i_1j_1}} \cdots
    \frac{\delta}{\delta J_{i_{N-1}j_{N-1}}} \bigg) \mathcal{W}(J) \bigg|_{J=0}
    \nonumber \\
  &= \frac{1}{N!(N-1)!} \epsilon_{l,i_1,\dots,i_{N-1}} \epsilon_{k, j_1, \dots, j_{N-1}}
  \epsilon_{i,i_1,\dots,i_{N-1}} \epsilon_{j,j_1,\dots,j_{N-1}} \nonumber \\
  &= \frac{1}{N!(N-1)!} (N-1)!\,\delta_{il}\,(N-1)!\,\delta_{jk} = \frac{1}{N}
  \delta_{il} \delta_{jk} .
\end{align}

\section{Static determinant} \label{sec-evaluating-fermion-determinants}
\section{\texorpdfstring{$W_{nm}$}{Wnm} terms}
\label{sec-evaluating-polyakov-coupling-terms}
