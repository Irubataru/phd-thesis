\chapter{Analytical Tools For SU(\texorpdfstring{$N$}{N})}

The analytic computations contained herein revolve around various SU$(N)$
dependent quantities, and we therefore need basis for computing these. In this
appendix we will look at how to compute all the necessary components such as the
Haar measure in \secref{sec-haar_measure}, the character integrals in
\secref{sec-character_integrals}, integrals over group polynomials in
\secref{sec-sun_integrals} and finally the fermionic functions in
\secrefs{sec-evaluating-fermion-determinants,sec-evaluating-polyakov-coupling-terms}.

\section{Computing the Haar measure} \label{sec-haar_measure}

We want the invariant measure of a group element expressed in terms of its
spectral decomposition. Any matrix representation of an U$(N)$ matrix has
unitary eigenvalues $\lambda_i = e^{i\theta_i}$ only, where $\sum_i \theta_i =
0 \bmod 2\pi$ for the elements of SU$(N)$. We write a matrix $U$ that is a representation
of U$(N)$ in its decomposed form
%
\begin{equation}
  U = W \Lambda W^{\dagger},
\end{equation}
%
where $\Lambda = \diag (e^{i\theta_1}, ..., e^{i\theta_n})$, and $W W^{\dagger}
= \mathbb{1}_R$.  A change in $U$ is therefore
%
\begin{equation}
  \mathrm{d}U = W \mathrm{d} \Lambda W^{\dagger} +  W \big[ W^{\dagger} \mathrm{d}
  W, \Lambda \big] W^{\dagger}.
\end{equation}
%
Using the standard unitary matrix metric, $||M||^2 = \tr(M^{\dagger} M)$, we get
the size of an infinitesimal square to be
%
\begin{equation} \label{eq-infinitesimal-group-square}
  \mathrm{d} s^2 = \tr (\mathrm{d} U^{\dagger} \mathrm{d} U)
  = \sum_i \mathrm{d} \theta_i^2 + 2 \sum_{i > j}
    \big|e^{i \theta_i} - e^{i\theta_j}\big|^2
    \big| (W^{\dagger} \mathrm{d} W)_{ij} \big|^2
\end{equation}
%
Since one identifies the set of coordinates to the metric through
$\mathrm{d} s^2 = g_{\mu\nu} \mathrm{d} \xi^{\mu} \mathrm{d} \xi^{\nu}$,
it is natural to construct the measure as
%
\begin{equation} \label{eq-natural-group-measure}
  \mathrm{d} U(x) = \sqrt{\det g(x)} \prod_{\mu} \mathrm{d}\xi^{\mu}.
\end{equation}
%
To reconcile these two formulae we use the Baker-Campbell-Hausdorff formula to
express $W^{\dagger} \mathrm{d} W$ in terms of the generators of U$(d_R)$
%
\begin{equation}
  (W^{\dagger} \mathrm{d} W)_{ij} = i (T_a)_{ij} f_{a b}(w) \mathrm{d} w_b
  \equiv Q_{ijk}(w) \mathrm{d} w_k
\end{equation}
%
with $f$ being some function of the parameters of the generator decomposition.
The general coordinates will therefore be a combination of the eigenvalue angles
$\theta_i$ and the parameters $w_i$. The determinant of the metric factorise
into parts dependent on these coordinates, and we get
%
\begin{equation}
  \det g(x) \sim \big(\det Q(w)\big)^2 \prod_{i>j} \big|e^{i\theta_i} -
  e^{i\theta_j}\big|^4.
\end{equation}
%
Since we are mostly concerned with the integrals over the eigenvalues only, one
can integrate out the $w_i$, and get the invariant measure
%
\begin{equation}
  \mathrm{d} U = \underbrace{\prod_{i>j} \big| e^{i\theta_i} - e^{i\theta_j} \big|^2}_{H(U)} \prod_i
  \mathrm{d} \theta_i,
\end{equation}
%
where we have identified the Haar measure, $H(U)$. Although this relation is
only true for U$(N)$, it can be trivially extended to SU$(N)$ through the
previously mentioned restriction, $\sum_i \theta_i = 0 \bmod 2\pi$.

To facilitate the calculation of the Haar measure, we introduce the Vandermonde
determinant
%
\begin{equation}
  \prod_{i>j} (z_i - z_j) = \det \mathcal{M} =
  \begin{vmatrix}
    1 & z_1 & \cdots & z_1^{N-1} \\
    1 & z_2 & \cdots & z_2^{N-1} \\
    \vdots & \vdots & \ddots & \vdots \\
    1 & z_N & \cdots & z_N^{N-1} 
  \end{vmatrix}.
\end{equation}
%
This relation one can use to rewrite the Haar measure
%
\begin{equation}
  H(U) = \prod_{i>j} \big| z_i - z_j \big|^2
   = \det \mathcal{M}^{\dagger} \mathcal{M} = \scalemath{0.8}{%
  \begin{vmatrix}
    N & \sum_i z_i & \sum_i z_i^2 & \cdots & \sum_i z_i^{N-1} \\
    \sum_i z_i^{\dagger} & N & \sum_i z_i & \cdots & \sum_i z_i^{N-2} \\
    \vdots & \vdots & \vdots & \ddots & \vdots \\
    \sum_i z_i^{\dagger N-1} & \sum_i z_i^{\dagger N-2} & \sum_i z_i^{\dagger N-3} & \cdots & N
  \end{vmatrix}}.
\end{equation}
%
For the fundamental representation of SU$(N)$, one has $N - 1$ independent
eigenvalues. We know that
%
\begin{equation}
  \sum_i z_i^m = \tr U^m \equiv \chi_m,
\end{equation}
%
correspond to the $m$'th powered character. One can therefore employ the
Caylay-Hamilton theorem to determine how these angles depend on each other.
This equation reads
%
\begin{equation} \label{eq-caylay-hamilton}
  M^N + c_{N-1} M^{N-1} + \cdots + c_1 M + (-1)^N \det M\: \mathbb{1}_R = 0,
\end{equation}
%
where the coefficients are
%
\begin{equation}
  c_i = \sum_{\{k_n\}} \prod_{l=1}^N \frac{(-1)^{k_l+1}}{l^{k_l} k_l!}
  \tr(M^l)^{k_l},
\end{equation}
%
in which the sum goes over all integer partitions $\{k_n\}$ so that
%
\begin{equation}
  \sum_{l=1}^N l k_l  = N - i.
\end{equation}
%
Since we know that $\det U = 1$ and $U^{\dagger} U = \mathbb{1}$ for all
matrices in SU$(N)$, \meqref{eq-caylay-hamilton} can be multiplied by powers of
$U^{\dagger}$ and traced to find relations to reduce the number of unknowns in
the Haar measure.

As an example, for SU$(3)$, $\chi_2 = \chi^2 - 2 \chi^*$, giving
%
\begin{equation}
  H(U) =
  \begin{vmatrix}
    3 & \chi & \chi^2 - 2 \chi^* \\
    \chi^* & 3 & \chi \\
    \chi^{*2} - 2 \chi & \chi^* & 3
  \end{vmatrix} = 27 - 18 \chi \chi^* + 8 \mathrm{Re} \chi^3 - (\chi \chi^*)^2.
\end{equation}

\section{\texorpdfstring{$\chi_r \chi_s$}{Ln Lm} integrals} \label{sec-character_integrals}

\section{\texorpdfstring{$g^n (g^{-1})^m$}{Un Udm} integrals} \label{sec-sun_integrals}
\section{Fermion determinant} \label{sec-evaluating-fermion-determinants}
\section{\texorpdfstring{$W_{nm}$}{Wnm} terms}
\label{sec-evaluating-polyakov-coupling-terms}
