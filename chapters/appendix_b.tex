\chapter{Additional analytic results}

In this chapter we will summarise all additional analytic results which are
either to lengthy, or do not fit into the main text.

\section{Non graphical \texorpdfstring{$\mathcal{O}(\kappa^8)$}{O(k8)} effective action}
\label{apx:full_k8_action}

The effective action to $\mathcal{O}(\kappa^8 N_t^{-1})$

\begin{flalign}
  S_{\mathrm{eff}} = S_0 + S_2 + S_4 + S_6 + S_8 + \mathcal{O}\big(\kappa^{10}, \frac{1}{N_{\tau}}\big) &&
\end{flalign}
%
\begin{flalign}
  S_0 = -\sum_x \log \det \qstat(x), &&
\end{flalign}
%
\begin{flalign} \label{eq:expansionBegin}
   S_2 = h_2 N_f \sum_x \sum_i W_{1,1}(x) W_{1,1}(x+i), &&
\end{flalign}
%
\begin{flalign}
  S_4 = 
    &-h_2^2 N_f \sum_x \sum_{i,j} W_{2,1}(x) W_{1,1}(x+i) W_{1,1}(x+j)& \nonumber\\
    &-h_2^2 N_f^2 \sum_x \sum_{i} W_{2,1}(x) W_{2,1}(x+i),&
\end{flalign}
%
\begin{flalign}
  S_6 =
    +\frac{1}{3}h_2^3 N_f &\sum_x \sum_{i,j,k} \big[ W_{3,1}(x) - W_{3,2}(x) \big] W_{1,1}(x+i) \label{eq:w3w1w1w1}
     W_{1,1}(x+j) W_{1,1}(x+k)& \nonumber\\
    +h_2^3 N_f &\sum_x \sum_{i,j,k} W_{2,1}(x) W_{2,1}(x+i) W_{1,1}(x+i+j) W_{1,1}(x+k)& \nonumber\\
    +2h_2^3 N_f^2 &\sum_x \sum_{i,j} \big[ W_{3,1}(x) - W_{3,2}(x) \big] W_{2,1}(x+i) W_{1,1}(x+j)& \nonumber\\
    +\frac{1}{6}h_2^3 N_f &\sum_x \sum_i \big[ W_{3,1}(x) W_{3,1}(x+i) + W_{3,2}(x) W_{3,2}(x+i) \big]& \nonumber\\
    -\frac{4}{3}h_2^3 N_f^3 &\sum_x \sum_i W_{3,1}(x) W_{3,2}(x+i),&
\end{flalign}
%
{
\allowdisplaybreaks
\begin{flalign}
  S_8 =
  +\frac{1}{12} h_2^4 N_f &\smash[b]{\sum_x} \smash[b]{\sum_{i,j,k,l}}
    \big[ W_{4,1}(x) - 4 W_{4,2}(x) + W_{4,3}(x) \big] W_{1,1}(x+i) W_{1,1}(x+j)& \nonumber\\
  & \hspace{4em} \times W_{1,1}(x+k) W_{1,1}(x+l)& \nonumber\\
  +h_2^4 N_f &\smash[b]{\sum_x} \smash[b]{\sum_{i,j,k,l}}
    \big[ W_{3,1}(x) - W_{3,2}(x)  \big] W_{2,1}(x+i) W_{1,1}(x+i+j)& \nonumber\\
  & \hspace{4em} \times W_{1,1}(x+k) W_{1,1}(x+l)& \nonumber\\
  +h_2^4 N_f &\smash[b]{\sum_x} \smash[b]{\sum_{i,j,k,l}} W_{2,1}(x) W_{2,1}(x+i) W_{2,1}(x+j)& \nonumber\\
  & \hspace{4em} \times W_{1,1}(x+i+k) W_{1,1}(x+j+l)& \nonumber\\
  +h_2^4 N_f^2 &\smash[b]{\sum_x} \smash[b]{\sum_{i,j,k}} \big[ W_{4,1}(x) - 4 W_{4,2}(x) + W_{4,3}(x) \big] W_{2,1}(x+i)& \nonumber\\
  & \hspace{4em} \times W_{1,1}(x+j) W_{1,1}(x+k)& \nonumber\\
  +h_2^4 N_f^2 &\smash[b]{\sum_x} \smash[b]{\sum_{i,j,k}}
    \big[ W_{3,1}(x) W_{3,1}(x+i) - 2 W_{3,1}(x) W_{3,2}(x+i) + W_{3,2}(x) W_{3,2}(x+i)\big]& \nonumber\\
  & \hspace{4em} \times W_{1,1}(x+j) W_{1,1}(x+i+k)& \nonumber\\
  +2 h_2^4 N_f^2 &\sum_x \sum_{i,j,k}
    \big[ W_{3,1}(x) - W_{3,2}(x) \big] W_{2,1}(x+i) W_{2,1}(x+j) W_{1,1}(x+j+k)& \nonumber\\
  + h_2^4 N_f^2 &\sum_x \sum_{i,j} W_{2,1}(x) W_{2,1}(x) W_{2,1}(x+i) W_{2,1}(x+j)& \nonumber\\
  + \frac{1}{2} h_2^4 N_f^2 &\sum_x \sum_{i,j} W_{2,1}(x) W_{2,1}(x+i) W_{2,1}(x+j) W_{2,1}(x+i+j)& \nonumber\\
  +\frac{1}{3} h_2^4 N_f &\smash[b]{\sum_x} \smash[b]{\sum_{i,j}}
  \big[ W_{4,1}(x) W_{3,1}(x+i) - 2 W_{4,2}(x) W_{3,1}(x+i) + 2 W_{4,2}(x) W_{3,2}(x+i)& \nonumber\\
  & \hspace{4em} - W_{4,3}(x) W_{3,2}(x+i) \big] W_{1,1}(x+j)& \nonumber\\
  -\frac{4}{3} h_2^4 N_f^3 &\smash[b]{\sum_x} \smash[b]{\sum_{i,j}}
    \big[ 2 W_{4,2}(x) W_{3,1}(x+i) - W_{4,3}(x) W_{3,1}(x+i) + W_{4,1}(x) W_{3,2}(x+i)& \nonumber\\
  & \hspace{4em} - 2 W_{4,2}(x) W_{3,2}(x+i) \big] W_{1,1}(x+j)& \nonumber\\
  + \frac{1}{12} h_2^4 N_f &\sum_x \sum_{i,j}
    \big[ W_{4,1}(x) - 4 W_{4,2}(x) + W_{4,3}(x) \big] W_{2,1}(x+i) W_{2,1}(x+j)& \nonumber\\
  + \frac{2}{3} h_2^4 N_f^3 &\sum_x \sum_{i,j}
    \big[ W_{4,1}(x) - 4 W_{4,2}(x) + W_{4,3}(x) \big] W_{2,1}(x+i) W_{2,1}(x+j)& \nonumber\\
  + \frac{1}{12} h_2^4 N_f^2 &\smash[b]{\sum_x} \smash[b]{\sum_i}
    \big[ W_{4,1}(x) W_{4,1}(x+i) + 12 W_{4,2}(x) W_{4,2}(x+i)& \nonumber\\
  & \hspace{4em} + W_{4,3}(x) W_{4,3}(x+i) \big]& \nonumber\\
  + \frac{2}{3} h_2^4 N_f^4 &\sum_x \sum_i \big[ W_{4,1}(x) W_{4,3}(x+i) + 2 W_{4,2}(x) W_{4,2}(x+i) \big]& \nonumber\\
  - \frac{2}{3} h_2^4 N_f^2 &\sum_x \sum_i \big[ W_{4,1}(x) W_{4,2}(x+i) + W_{4,2}(x) W_{4,3}(x+i) \big].&
\end{flalign}
}
%
The sums over the spatial indices $\{i,j,k,l\}$ are over all spatial directions.
At the leading order in temperature all gauge corrections come from rescaling
the coupling constants $h_1$ and $h_2$. Away from the low temperature limits new
gauge corrections will appear as we saw in \secref{sec:gauge_comment}.

\section{Integrated LCE \texorpdfstring{$n$}{n}-point functions}
\label{apx:z_functions}

Below are all $z$'s needed for the $n$-point functions in use to carry out the
analytic computation to $\mathcal{O}(k^8)$ with $N_c = 3$ and $N_f = 2$. The
definition of $z$ can be found in \meqref{eq:lce_z_definition}. The integrals
can be easily computed with the methods outlined in \apxref{apxA},
%
\begin{subequations}
\begin{alignat}{4}
  &z_0 &&= 1 + 20 h_1^3 + 50 h_1^6 + 20 h_1^9 + h_1^{12}, \\
  &z_{(11)} &&= 15 h_1^3 + 75 h_1^6 + 45 h_1^9 + 3 h_1^{12}, \\
  &z_{(21)} &&= 21 h_1^3 + 70 h_1^6 + 21 h_1^9, \\
  &z_{(31)} &&= 28 h_1^3 + 35 h_1^6 - 7 h_1^9, \\
  &z_{(32)} &&= -7 h_1^3 + 35 h_1^6 + 28 h_1^9, \\
  &z_{(41)} &&= 36 h_1^3 - 40 h_1^6 + h_1^9, \\
  &z_{(42)} &&= -8 h_1^3 + 75 h_1^6 - 8 h_1^9, \\
  &z_{(43)} &&= h_1^3 - 40 h_1^6 + 36 h_1^9, \\
  &z_{(11)^2} &&= 6 h_1^3 + 95 h_1^6 + 96 h_1^9 + 9 h_1^{12}, \\
  &z_{(11)(21)} &&= 7 h_1^3 + 105 h_1^6 + 56 h_1^9, \\
  &z_{(11)(31)} &&= 8 h_1^3 + 100 h_1^6 - 20 h_1^9, \\
  &z_{(11)(32)} &&= -h_1^3 + 5 h_1^6 + 76 h_1^9, \\
  &z_{(11)^3} &&= h_1^3 + 90 h_1^6 + 188 h_1^9 + 27 h_1^{12}, \\
  &z_{(21)^2} &&= 8 h_1^3 + 135 h_1^6 + 8 h_1^9, \\
  &z_{(11)^2(21)} &&= h_1^3 + 100 h_1^6 + 148 h_1^9, \\
  &z_{(11)^4} &&= 60 h_1^6 + 312 h_1^9 + 81 h_1^{12}.
\end{alignat}
\end{subequations}


\section{Higher order \texorpdfstring{$z$}{z}-functions}

So far we have only encountered $z$-functions of order $\leq 2 N_f$. This has
the advantage that the resulting integrals over the characters enter only in
polynomial form. This is due to the fact that the $W_{nm}$ term is
%
\begin{equation}
  W_{nm} = \frac{w_{nm}}{\det{}^n(1 + h_1 W)}
\end{equation}
%
where $w_{nm}$ is a polynomial in the characters. As the denominator enters to
power $2 N_f$ in the integrand from the static determinant, the resulting
integral only has polynomials of the characters, which we saw how to solve in
\apxref{sec:character_integrals}. For higher orders, an expansion is needed.
The simplest option is to expand the expression for $W_{nm}$ directly
%
\begin{equation}
  W_{nm} = \tr \bigg( \frac{(h_1 W)^m}{(1 + h_1 W)^n} \bigg) = \tr \bigg( (h_1
  W)^m \sum_{k=0}^{\infty} \binom{k + n - 1}{k} (-h_1 W)^k \bigg)
\end{equation}
%
In this case the expression is a series in $W$, and the higher powers of the $W$
matrices have to be rewritten to lower order ones using the Caylay-Hamilton
equation. Using this, a direct computation of $z_{(11)^5}$ has been carried out
as an example and the order by order results are compared to the full result in
\figref{fig:z_function_expansions} (left). 

\begin{figure}
  \begin{center}
    \begin{adjustbox}{max width=\textwidth}
      \includegraphics{z_function_slow_expansion}
      \includegraphics{z_function_fast_expansion}
    \end{adjustbox}
  \end{center} \vskip -.5cm
  \caption{Expansions schemes for the higher order $z$-functions at $N_f = 2$.
    Left: Direct expansion in the $W$ matrices. Right: Reshuffling of the
    expansion to a series in $h_1$.}
  \label{fig:z_function_expansions}
\end{figure}

Unfortunately, convergence of this series is particularly slow, and it will for
any order diverge from the correct result at $h_1 = 1$. To amend this, one can
shuffle the expansion either into a series in the full characters by including
the lower orders from the static determinant, or one can reorder it into a pure
expansion in $h_1$, also here taking the already existing orders of the static
determinant into account. The latter choice is plotted in
\figref{fig:z_function_expansions} (right), where the improved convergence is
clearly visible. At $12$\textsuperscript{th} order the expression is almost
exact, and it reaches the full result at $15$\textsuperscript{th} order. Above
this order the remainder of the series' coefficients are all zero.
