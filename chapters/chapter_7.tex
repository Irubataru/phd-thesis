\chapter{Research Perspectives}

Although we have answered many questions regarding the properties and
applications of the effective theory in this thesis, we have also asked new
ones, and been organically lead onto the path of new research. First and
foremost, applications of the effective theory to large-$N_c$ studies, opening
up communication with the supersymmetry community was elaborated in
\secref{sec:large_nc_study}. Also, the application of the analytic expressions
to studies of Yang Lee zeros was outlined.

Furthermore, questions regarding the nature of the continuum extrapolated
results remain unanswered, and additional research is in order. One possible
avenue of future investigation is a more thorough analysis of the analytic
contributions to said effect. Additional insight could be reached through the
large-$N_c$ study, as the \emph{polyquark} states have different statistical
properties depending on the gauge group. Finally one could explore the
corrections to the ground state energy of the different baryonic states, and see
whether it is possible to manipulate the degrees of freedom in such a way that
the continuum physics alter, following predictable patterns.

Another research prospect is the computation of susceptibilities, such as
Polyakov loop susceptibilities. These quantities are a paramount ingredient in
the study of phase transitions for series expansion studies. The linked cluster
method proposes a systematic way of computing these quantities in the
thermodynamic limit, and is given by the set of $n$-rooted graphs with
connectivity properties depending on the variable of study. We believe that
these can be extended to the polymer linked cluster in a straightforward manner,
and should prove immediately beneficial to our continued efforts.

Other projects using the same formalism have already been initiated in our
working group, for instance the study of isospin chemical potential. This was
already explored with the $\mathcal{O}(\kappa^4)$ effective theory action in
\citep{Langelage:2014vpa} using numerical methods. It can be extended to the
realm of analytics by applying the methods outline in this thesis. Another
project also in progress is the evaluation of the canonical partition function
constructed through Fourier transforming the effective theory grand canonical
partition function. This could provide an additional extraction point for the
nucleon binding energy, which remains a future goal. It could also help
illuminate the question of the nature of our continuum physics.

In addition to all of this, we have seen that resummation schemes, as well as
systematic improvement of the theory are more important than order by order
results. Effort should therefore be put towards the development of these
methods. This includes applying the cluster expansion resummation strategy, and
looking to it for inspiration when devising resummation formulas on the level of
the effective action. Recursive improvement patterns, such as coarse graining
techniques, are also of great interest. Finally, we have based the study on an
unimproved Wilson fermion action. Order $a$ improvement schemes are therefore
massively helpful in reducing continuum extrapolation errors.

All in all, we conclude that the field is healthy, and is currently being
adopted by other groups, showing a broad interest and faith in the approach
\citep{Scior:2015vra,Scior:2016fso,Rindlisbacher:2015pea}.
