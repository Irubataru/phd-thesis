\chapter*{Deutsche Zusammenfassung}
{\fontsize{12pt}{18pt}\selectfont

In this thesis we will study the properties of Quantum Chromodynamics (QCD), a
theory of the dynamics of the strong nuclear force, and one of the four
fundamental forces of the universe. The quarks, the constituent particles of the
nucleons, actually carry the appropriate quantum numbers to be influenced by all
of these forces. However, of the four forces acting on the quarks, none are more
important than QCD, which is the force that confines them into inseparable
hadrons. The resulting binding energy is in fact responsible for almost 99\% of
the mass of the hadrons, and the strength of the strong nuclear is as a
result the source of most of the visible mass in the universe.

Of particular interest is the phase diagram of QCD for strongly interacting
matter. At very high energies, QCD is asymptotically free, meaning that although
it is strongly interacting at the energy scales of e.g. chemistry, it becomes
weakly interacting at extreme energies (very short distances). QCD has an
internal energy scale $\Lambda_{\mathrm{QCD}} \approx 200 \mathrm{MeV}$, which
determines the fundamental scale of this transition. One possible source for
this energy is macroscopic and collective statistical effects such as
temperature and chemical potential.

At temperatures around $T_c \sim \Lambda_{\mathrm{QCD}}$, a crossover transition
occurs, where the characteristic degrees of freedom change from being the
confined hadronic states, to a weakly interacting soup of pseudo-free quarks and
gluons, known as the \emph{quark-gluon-plasma} (QGP). This transition is expected to
have taken place in the early universe, and the QGP is accessed experimentally
at various heavy ion collision experiments around the world. This includes e.g.
the Relativistic Heavy Ion Collider (RHIC) at Brookhaven National laboratory and
the Large Hadron Collider (LHC) at CERN, in Geneva. There are also future
experiments planned such as the Facility for Antiproton and Ion Research (FAIR)
at GSI in Darmstadt, and the Nuclotron-based Ion Collider fAcility (NICA) at
JINR in Dubna, which aim at extending the QGP transition studies to higher
density systems, and search for the elusive critical endpoint.

In the opposite limit, at low temperatures but high densities one finds the
liquid gas phase transition at quark chemical potentials $\mu_c \sim M_p$ (the
proton mass). Following this transition curve towards higher densities ends in a
critical endpoint, different from the one theorised to exist due to
deconfinement. At even more extreme densities we once more encounter
asymptotic freedom, however in the cold and dense limit this coincides with
a superconducting phase. This is similar to the superconducting phases of
electromagnetism, however the additional colour charge plays a non-trivial role
in the construction of Cooper pairs.

To this work, the aforementioned liquid gas phase transition is the main focus.
The liquid gas transition happens at parameters high enough for low energy
effective theories to give unsatisfactory results, however it is also too small
for perturbation theory to be applicable. We therefore turn to Lattice QCD
(LQCD), in which the space-time is discretised, and put on a finite grid with a
finite extent. This introduces both a smallest distance, $a$ (the lattice
spacing), as well as a largest distance, $a N$. LQCD is a representation of the
path integral formalism in which the number of possible paths have been reduced
from infinity to a smaller amount, depending on the size of the lattice.  This
discretisation was first proposed by Wilson in 1974, and is is constructed in
such a way that in the limit $a \to 0$ (continuum limit) and $N \to \infty$
(such that $a N \to \infty$), it mathematically reproduces full QCD. Differences
can often be categorised into either finite size effects (due to $a \neq 0$), or
finite volume effects (when $a N \neq \infty$). 

LQCD  has without a doubt proven is efficacy with its ability to reproduce
experimental results for vacuum observables, such as the masses of the light
hadrons to within few percent of the experimental ones. Not only does this serve
as a test of the usefulness of LQCD, but also of the correctness of QCD and the
quark model as a theory of nature.

Although LQCD greatly simplify the mathematical burden of computing observables,
additional numerical methods need to be wielded to access the full
non perturbative results. Popular methods revolve around stochastic integration,
such as Monte Carlo integration and Langevin integration. These methods exploit
the fact that one can use the integration measure itself to extract
information on the dominant contributions to the full integral. Still, the
variable space is too large for a straight forward exploration, and one adopt
chains of configurations, snaking across the space of all configurations,
sampling the integrand as it changes.

This is also applicable to thermodynamic systems in which the Minkowski time is
replaced by a periodic Euclidean time, corresponding to the temperature in
thermal field theoretical descriptions. However, attempting to introduce
chemical potential introduce a problem from the numerical side. Although the
fermion determinant (a mathematical structure encoding the full fermionic
dynamics) is a real number at zero chemical potential, the inclusion of a
chemical potential term shifts this object into the complex plane. For continuum
physics, this is not a problem, as the imaginary components cancel out in the
full set of integrals, however the numerical evaluation is spoiled. One cannot
use the complex integrand as a probability weight for the stochastic variables.
Such difficulties could in theory be overcome by using a slightly different, but
close enough distribution instead to draw ones observables from. This practice
is known as \emph{reweighting}. However, at $\mu / T > 1$, finding appropriate
distributions become difficult, and stochastic integration break down. This is
what in the field is known as the sign problem.

Another difficulty that arise in finite chemical potential lattice studies come
from the Pauli exclusion principle. Due to the fermionic nature of the quarks
combined with the discretised lattice, an LQCD system naturally has an upper
limit for the number of quarks it can support. This lattice artefact becomes
critical around half-filling, the point at which half of the available quantum
states have been occupied. This limitation of lattice studies is known as
\emph{lattice saturatio}, and is a phenomenon not often discussed in the lattice
community due to the fact that the sign problem enters at a much earlier stage
of the calculation. As we overcome the sign problem, this becomes a very real
obstacle that we will have to surmount.

To alleviate the sign problem we introduce an effective lattice theory in which
we integrate out a subset of the degrees of freedom analytically. We show that
this procedure is effective at dampening the severity of the sign problem, and
this effective theory can easily be simulated using reweighting techniques. To
carry out this computation, the integrand must be expanded around certain
limits. One convenient expansion point is the theory of strong coupling static
quarks, in which the partition function integrals are analytically solvable. The
strong coupling limit is the limit of $\alpha_s \to \infty$, which in LQCD
corresponds to the Wilson plaquette parameter $\beta \propto 1/\alpha_s \to 0$.
The static limit is the limit of infinitely heavy quarks, $m_q \to \infty$.
Using LQCD implemented in terms of Wilson fermions, the quark mass enter only in
the \emph{hopping parameter}, $\kappa = \frac{1}{2(4 + a m_q)}$. The static
limit expansion is thus an series around $\kappa \to 0$, also known as the
hopping expansion.

The continuum physics of such a lattice theory must be access by rescaling
the internal parameters of the system. Due to renormalisation effects, the
coupling constant picks up a dependence on the scale of the system, $\beta \to
\beta(a)$, and we can explore the $a$-dependence of observables through
variations of $\beta$. The continuum limit dependence of $\beta$ can be accessed
by perturbation theory, and the true continuum limit can be shown to be
$\beta(a \to 0) \to \infty$, which is the opposite limit of our expansion
starting point. However, by going to high orders in this parameter, we are able
to probe the scaling region of LQCD, from which an analytical extrapolation can
be carried out to the continuum. More details on this in the text.

The effective theory is guaranteed to reproduce the correct expansion
coefficients of full LQCD around these limits, and thus represents a systematic
approach to accessing this very interesting region of the QCD phase diagram from
first principle methods. This thesis is focussed on the derivation of the
effective theory itself, and has a full chapter dedicated to introducing a
scheme for evaluating the correct expansion coefficients on an order by order
basis. We find that the combinatorics of the expansion simplify at low
temperatures, which on the lattice corresponds to a large temporal lattice
extent. This results in a third expansion coefficient, namely $T$ (or more
specifically $a T = 1/N_t$). Finally we also take advantage of the
simplifications that arise for very dense systems. If the chemical potentials
are large enough, the anti-quarks have a very small contribution, and can be
neglected. The resulting degrees of freedom are the so called \emph{Polyakov
  loops}, closed quark lines in the temporal direction. The effective action is
built up of nearest neighbour-, next to nearest neighbour- and so on,
-interactions between these variables. The system can in a sense be seen as a
complicated non-local system of continuous spins having certain symmetry
properties. In the end we present a $\mathcal{O}(\kappa^8 u^5 N_t^4)$ result for
the effective theory.

Although the numerical prowess of the effective theory is discussed, it is not
the main attraction of the current work. Instead we present a novel, purely
analytical treatment of the effective theory. We base the study on the classical
linked cluster expansion, the de facto manner in which one extracts results for
equilibrium thermodynamics from spin systems, and systems that can be rewritten
to resemble one. This method needs to be generalised to fit the requirements of
our non-local action, and we therefore develop a polymer linked cluster
expansion which borrows techniques from graph theory.

We make use of the full analytic power to develop resummation schemes that are
otherwise inaccessible to numerical studies. These include the aptly named
\emph{chain resummation}, which adds together contributions from the effective
theory that form a chain, or can be embedded onto one. Another nearest neighbour
resummation, the \emph{ladder resummation}, is also discussed. We see that
these resummation methods greatly improve the convergence of the theory, which
incentivices future studies to focus on these types of improvements. 

At the end of the thesis we compute various thermodynamic quantities using our
fully analytic grand canonical partition function. This is done at temperatures
in the $\sim 10 \mathrm{MeV}$ range and chemical potentials around the baryon
mass. We still find a crossover transition to nuclear matter due to the fact
that the critical endpoint of the liquid gas transition moves down to lower
temperatures as quark masses increase. A study of the previously mentioned
lattice saturation issue is also presented. The saturation issue manifest itself
most prominently in the continuum extrapolation, and we see that although
continuum limit results do not suffer this lattice artefact, its existence
impede lattice computations of physics beyond the liquid gas phase transition.

Our final and most interesting result is regarding the equation of state for
heavy quark QCD in the cold and dense. We find that although the computation is
based on single quark computations at strong coupling, that the continuum
physics resemble that of weakly coupled fermions. The appearance of these
interactions from a strongly coupled system is a dynamic process, and
demonstrates the approach' strengths.  A study of the contributing degrees of
freedom is conducted, and we are left with various open questions to be answered
in future work. 

The effective theory can also be extended to describing gauge theories of
different local symmetry groups, e.g. SU($N_c$) ($N_c = 3$ for QCD). This
observation somewhat permeates the entire thesis as the symmetry group is rarely
explicitly specified. For instance a world in which $N_c = 4$ would look very
different from the one we live in as for instance every hadron would be a boson.
The foundation of such a study is given in \chapref{chap5}, and further research
is currently in progress. Also an appendix detailing the mathematics and
mathematical tools necessary to carry out such a study supplied.


\par}
