\begingroup%
\let\clearpage\relax%
\let\cleardoublepage\relax%
\fontsize{12pt}{18pt}\selectfont

\chapter*{Deutsche Zusammenfassung}
\markboth{Deutsche Zusammenfassung}{Deutsche Zusammenfassung}
\addcontentsline{toc}{chapter}{Deutsche Zusammenfassung}

In der vorliegenden Arbeit studieren wir die Eigenschaften der
Quantenchromodynamik (QCD), welche die Dynamik der starken Wechselwirkung, eine
der vier fundamentalen Kräfte des Universums, beschreibt. Die zugehörigen
Quantenzahlen werden von den elementaren Bestandteilen der Nukleonen, den
Quarks, getragen. Dabei ist die starke Wechselwirkung dafür verantwortlich, dass
Quarks gebundene Zustände, sogenannten Hadronen, bilden. Die daraus
resultierende Bindungsenergie entspricht 99\% der Hadronen-Mas\-se und somit ist
die Stärke der starken Wechselwirkung der Ursprung des Groß\-teils der
sichtbaren Materie in unserem Universum.

Von großem Interesse ist insbesondere das Phasendiagramm der QCD für stark
wechselwirkende Materie. Die QCD hat die Eigenschaft der asymptotischen
Freiheit, was dazu führt, dass die Stärke der Wechselwirkung bei sehr hohen
Energien (bzw. sehr kleinen Abständen) abnimmt. Die zugehörige Energieskala für
dieses Phänomen ist $\Lambda_{\mathrm{QCD}} \approx 200 \:\mathrm{MeV}$. Derartig
hohe Energien können bei Kollisionen in Teilchenbeschleunigern entstehen, sind
aber auch in thermodynamischen Systemen mit hinreichend hohen Temperaturen
und/oder Dichten erreichbar.

Bei Temperaturen um $T_c \sim \Lambda_{\mathrm{QCD}}$ tritt ein
Crossover-Übergang auf, bei dem die charakteristischen Freiheitsgrade nicht mehr
gebundene, hadronische Zustände bilden, sondern in einen Zustand quasi-freier
Quarks und Gluonen, bekannt als das \emph{Quark-Gluon-Plasma} (QGP), übergehen.
Man geht davon aus, dass dieser Übergang in den anfänglichen Phasen unseres
Universums vorzufinden war. Experimentell wird dieser Zustand in den
verschiedenen Kollisionsexperimenten, die es weltweit gibt, untersucht. Dazu
gehören beispielsweise der Relativistic Heavy Ion Collider (RHIC) am Brookhaven
National Laboratory und der Large Hadron Collider (LHC) am CERN in Genf.
Außerdem sind weitere Experimente an Einrichtungen, wie die Facility for
Antiproton and Ion Research (FAIR) an der GSI in Darmstadt oder die
Nuclotron-based Ion Collider fAcility (NICA) am JINR in Dubna, in der
Entstehung. In letzteren wird insbesondere der QGP-Übergang bei höheren Dichten
und der sogenannte \emph{critical endpoint} (CEP) untersucht.

Im Grenzwert niedriger Temperaturen und hoher Dichten findet man dagegen den
Flüssigkeit-Gas-Phasenübergang bei chemischem Quark-Po\-ten\-ti\-al $\mu_c \sim
M_p$ (Protonenmasse). Folgt man der Kurve des Phasendiagramms dieses Übergangs
hin zu höheren Dichten, so gelangt man ebenfalls an einen CEP, der jedoch nicht
mit dem hypothetischen CEP des QGP zu verwechseln ist. Bei noch höheren Dichten
wird erneut die asymptotische Freiheit deutlich. In diesem Grenzwert kalter und
sehr dichter Materie befindet man sich zudem in der sogenannten supraleitenden
Phase. Diese ist vergleichbar mit der supraleitenden Phase in der
Elektrodynamik, wobei die zusätzliche Farbladung eine nicht triviale Rolle in
der Konstruktion der Cooper-Paare einnimmt.

Das Hauptaugenmerk liegt in dieser Arbeit auf dem erwähnten
Flüssigkeit-Gas-\-Pha\-sen\-über\-gang. Dieser Übergang liegt im Bereich von Energien,
die zu groß für effektive Theorien, jedoch zu klein zur Anwendung von
Stö\-rungs\-theo\-rie, sind. Deshalb wenden wir Gittereichtheorie im Rahmen der
QCD an, in der die Raumzeit diskretisiert und durch ein endliches Gitter mit
endlichem Gitterabstand beschrieben wird. Dadurch wird ein kleinster Abstand,
der Gitterabstand $a$, genauso wie ein größter Abstand, $a N$, definiert.
Mithilfe der Gitter-QCD lässt sich der Pfadintegralformalismus approximieren, in
dem die unendliche Anzahl von möglichen Pfaden auf eine endliche, von der
Gittergröße abhängige Anzahl reduziert wird. Diese Diskretisierung wurde
erstmals von Wilson im Jahre 1974 formuliert. Sie ist so konstruiert, dass sie
im Grenzwert $a \to 0$ (Kontinuumslimes) und $N \to \infty$ (sodass $a N \to
\infty$) die volle QCD reproduziert. Abweichungen werden meist in zwei
Kategorien unterteilt, einerseits \emph{finite size} Effekte (aufgrund von $a
\neq 0$) und andererseits \emph{finite volume} Effekte (da $a N \neq \infty$).

Die Gitter-QCD hat zweifelsfrei ihre Wirkungskraft durch die Bestätigung einer
Viel\-zahl experimenteller Ergebnisse von Vakuum-Observablen unter Beweis
gestellt. Bei\-spie\-le dafür sind die Massen leichter Hadronen, die bis auf wenige
Prozent genau übereinstimmen. Dies dient nicht nur als eine Bestätigung der
Gitter-QCD, sondern auch als eine Bestätigung der QCD und des Quark-Modells als
eine Theorie zur Beschreibung der Natur.

Auch wenn die Gitter-QCD die mathematischen Herausforderungen der Be\-rech\-nung
von Observablen deutlich vereinfacht, muss man sich dennoch numerischer Methoden
bedienen, um an die vollen nicht-perturbativen Ergebnisse zu gelangen. Gängige
Methoden beruhen auf der stochastischen Integration, wie beispielsweise der
Monte-Carlo-Integration oder der Langevin-Integration. Diese Methoden nutzen die
Tatsache, dass man das Integrationsmaß selbst verwenden kann, um Informationen
über die dominanten Beiträge zum vollen Integral zu erlangen.

Diese Methode lässt sich auch auf thermodynamische Systeme anwenden, in denen
die Minkowski-Zeit durch die periodische euklidische Zeit ersetzt wird. Die
euklidische Zeit steht dabei im Zusammenhang mit der Temperatur in
thermodynamischen Feldtheorien. Der Versuch ein chemisches Potential
hinzuzufügen bringt jedoch Probleme numerischer Natur mit sich. Auch wenn die
Fermionen-Determinante (ein mathematisches Objekt, das die volle fermionische
Dynamik beinhaltet) bei verschwindendem chemischen Potential eine reelle Zahl
ist, wird diese in die komplexe Ebene gerückt, sobald das chemische Potential
von Null verschieden ist. Das stellt in der Kontinuumsphysik keine Probleme dar,
da sich die imaginären Komponenten in der Betrachtung des vollen Satzes der
Integrale gegenseitig eliminieren. Für die numerische Implementation ist das
hingegen nicht der Fall, da man den komplexwertigen Integranden nicht mehr als
Wahrscheinlichkeitsgewicht der stochastischen Variablen verwenden kann. Dieses
Problem kann theoretisch umgangen werden, indem man eine leicht modifizierte,
jedoch ähnliche Verteilung zur Berechnung der Observablen wählt. Dieses
Verfahren ist bekannt als \emph{reweighting}. Für $\mu / T > 1$ lassen sich
jedoch häufig nur sehr schwer passende Verteilungen finden, wodurch eine
stochastische Integration nicht mehr möglich ist. Dieser Sachverhalt wird im
Allgemeinen als \emph{sign problem} bezeichnet.

Eine weitere Schwierigkeit, die bei Gitterberechnungen mit endlichem chemischen
Potential auftritt, hat ihren Ursprung im Pauli-Prinzip. Aufgrund der
fermionischen Natur von Quarks kombiniert mit dem diskreten Gitter besitzt ein
System in der Gitter-QCD eine obere Schranke für die Anzahl der zugelassenen
Quarks. Dieses Gitter-Artefakt wird problematisch an dem Punkt, an dem das
Gitter zur Hälfte gefüllt ist, also dem Punkt, an dem die Hälfte der möglichen
Quantenzustände besetzt sind. Dieses Phänomen wird als \emph{lattice saturation}
bezeichnet und wird selten diskutiert, da das \emph{sign problem} deutlich
früher bei der Berechnung auftritt. Sobald wir letzteres überwunden haben, wird
ersteres zu einer echten Herausforderung, die wir ebenfalls bewältigen müssen.

Um das \emph{sign problem} abzuschwächen, verwenden wir eine effektive
Gittertheorie, in der wir das auftretende Integral über einen Teil der
Freiheitsgrade analytisch auswerten. Wir werden zeigen, dass diese Methode
erfolgreich die Stärke des \emph{sign problem} mildert und dass die daraus
resultierende effektive Theorie leicht mithilfe von \emph{reweighting}-Tech\-ni\-ken
simuliert werden kann. Um diese Rechnung durchführen zu können, muss der
Integrand in bestimmten Grenzwerten entwickelt werden. Ein geeigneter
Entwicklungs\-punkt ist die Theorie der stark gekoppelten statischen Quarks, in
der die Integrale der Zu\-stands\-summe analytisch lösbar sind. Der zugehörige
Grenzwert wird als \emph{strong coupling limit} bezeichnet und entspricht dem
Grenzwert $\alpha_s \to \infty$, was in der Gitter-QCD dem Verschwinden des
Parameters $\beta \propto 1/\alpha_s$ der Wilson-Plakette entspricht. Des
Weiteren bezeichnen statische Quarks jene mit Masse  $m_q \to \infty$. Verwendet
man Gitter-QCD mit Wilson-Fermionen, so taucht die Masse nur im sogenannten
\emph{hopping parameter} $\kappa = \frac{1}{2(4 + a m_q)}$ auf. Der Limes
statischer Quarks ist somit eine Entwicklung um den Punkt $\kappa \to 0$, auch
bekannt als \emph{hopping expansion}.

Durch geeignete Reskalierung der Gittertheorie muss die zugehörige
Kontinuums\-phy\-sik zugänglich sein. Aufgrund von Renormierungseffekten erhält die
Kopplungskonstante eine Abhängigkeit von der Skala des Systems, das bedeutet
$\beta \to \beta(a)$, wodurch wir die $a$-Abhängigkeit der Observablen durch
Variation von $\beta$ bestimmen können. Das Verhalten des Kontinuumslimes in
Abhängigkeit von $\beta$ kann mithilfe von Störungstheorie studiert werden und
den tatsächlichen Kontinuumslimes erlangt man durch $\beta(a \to 0)$ $\to \infty$,
welches der entgegengesetzte Grenzwert zu unserer Entwicklung ist. Betrachtet
man jedoch höhere Ordnungen dieses Parameters, kann man eine Aussage über die
Skala der Gitter-QCD treffen, von der eine analytische Erweiterung zum Kontinuum
möglich ist. Weitere Einzelheiten dazu folgen später im Text.

Die effektive Theorie gewährleistet die Bestimmung der korrekten
Entwicklungskoeffizienten der vollen Gitter-QCD in den angesprochenen
Grenzwerten. Damit ermöglicht sie es uns diese interessante Region des
QCD-Phasenraumdiagrammes von fundamentaler Basis aus zu untersuchen. Der
Hauptbestandteil der vorliegenden Arbeit ist die Herleitung dieser effektiven
Theorie und beinhaltet ein volles Kapitel, das sich allein damit beschäftigt,
die korrekten Entwicklungskoeffizienten Ordnung für Ordnung zu bestimmen. Es
wird sich herausstellen, dass sich die Kombinatorik der Entwicklung bei
niedrigen Temperaturen deutlich vereinfacht, was bezogen auf das Gitter einer
großen zeitlichen Gitterausdehnung entspricht. Daraus resultiert ein dritter
Entwicklungsparameter $T$ (oder genauer  $a T = 1/N_t$). Schlussendlich
profitieren wir auch noch von den Vereinfachungen, die bei sehr dichten Systemen
auftreten. Wenn die chemischen Potentiale hoch genug sind, leisten die
Antiquarks keinen großen Beitrag und können daher vernachlässigt werden. Die
resultierenden Freiheitsgrade sind die sogenannten  \emph{Polyakov  loops},
welche geschlossene Quarklinien in zeitlicher Richtung sind. Die effektive
Wirkung setzt sich dann aus nächster-Nachbar-Wechselwirkungen,
übernächster-Nachbar-Wechselwirkungen, etc. dieser Variablen zusammen. Man kann
das System demnach als ein kompliziertes nicht-lokales System von
kontinuierlichen Spins mit bestimmten Symmetrien betrachten. Wir werden am Ende
der Arbeit ein $\mathcal{O}(\kappa^8 u^5 N_t^4)$-Ergeb\-nis dieser effektiven
Theorie präsentieren.

Auch wenn die numerische Fähigkeit der effektiven Theorie diskutiert wird, ist
sie nicht der Hauptreiz der gegenwärtigen Arbeit. Stattdessen präsentieren wir
eine neue, rein analytische Behandlung der effektiven Theorie. Dabei basieren
unsere Untersuchungen auf der klassischen \emph{linked cluster} Entwicklung,
welche die in der Praxis gängige Methode ist, um Ergebnisse für die
Gleichgewichts-Thermodynamik von Spin-Systemen zu erhalten. Diese muss jedoch
verallgemeinert werden, um den Anforderungen unserer nicht-lokalen Wirkung
gerecht zu werden. Deshalb erarbeiten wir eine \emph{polymer linked cluster}
Entwicklung, welche sich Techniken der Graphentheorie bedient.

Wir verwenden eine Vielzahl analytischer Methoden, um die
\emph{resummation}-Verfahren, die mit numerischen Methoden unerreichbar sind, zu
entwickeln. Dazu zählt die sogenannte \emph{chain resummation}, die alle
Beiträge der effektiven Theorie zusammenzählt, die eine Kette bilden oder in
eine solche eingeschlossen werden können. Eine weitere wäre die \emph{ladder
  resummation}, die wir ebenfalls behandeln werden. Wir werden sehen, dass diese
\emph{resummation}-Methoden die Konvergenz der Theorie deutlich steigern, was
den Anreiz schaffen sollte, zukünftige Arbeiten auf diese Art der Verbesserungen
zu fokussieren.

Am Ende dieser Arbeit werden wir verschiedene thermodynamische Grö\-ßen unter
Verwendung unserer rein analytischen großkanonischen Zu\-stands\-summe
berechnen. Dies wird im Temperaturbereich $\sim 10 \:\mathrm{MeV}$ und bei
chemischen Potentialen nahe der Baryonenmasse geschehen. Dabei sehen wir einen
Crossover-Übergang zu atomarer Materie, da sich der CEP des
Flüssigkeit-Gas-Überganges in Richtung niedrigerer Temperaturen bewegt, wenn man
die Quarkmasse erhöht. Eine Untersuchung der zuvor erwähnten \emph{lattice
  saturation} wird ebenfalls gezeigt. Dieses Problem zeigt sich insbesondere in
der Kontinuumsextrapolation und wir werden sehen, dass Gitter-Berechnungen
bezüglich der Physik jenseits des Flüssigkeit-Gas-Pha\-sen\-über\-ganges
erschwert werden, obwohl Ergebnisse im Kontinuumslimes nicht von diesen
Gitterartefakten be\-ein\-träch\-tigt werden.

Unser letztes und interessantestes Ergebnis bezieht sich auf die
Zustandsgleichung für die QCD schwerer Quarks in kalter und dichter Materie. Es
zeigt sich, dass die Kontinuumsphysik, die auf unseren Berechnungen mit schweren
Quarks und starker Kopplung beruht, sehr ähnlich zur Physik schwach gekoppelter
Fermionen ist. Der auftretende Wechsel der Freiheitsgrade des Systems ist ein
dynamischer Prozess und demonstriert die Stärke unseres Ansatzes. Wir führen
zwar Untersuchungen der beitragenden Freiheitsgrade durch, dennoch sind Fragen
für weiterführende Arbeiten offen gelassen.

Die effektive Theorie kann zusätzlich erweitert werden, um Eichtheorien
verschiede\-ner lokaler Symmetriegruppen, wie zum Beispiel SU($N_c$) ($N_c = 3$
für QCD), zu beschrei\-ben. Diese Tatsache macht sich in der gesamten Arbeit
bemerkbar, da die Symmetriegruppe nur sehr selten explizit festgelegt wird. Eine
Welt mit etwa $N_c = 4$ würde sich sehr von unserer unterscheiden, da
beispielsweise jedes Hadron ein Boson wäre. Der Grundstock einer solchen
Untersuchung ist in \chapref{chap5} zu finden und weiterführende
Berechnungen sind aktuell im Gange. Es wird außerdem ein Appendix angeboten, in
dem die zugehörigen mathematischen Werkzeuge detailliert beschrieben werden.

\par \endgroup%
