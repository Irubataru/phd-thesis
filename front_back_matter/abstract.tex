\chapter*{Abstract}
{\fontsize{12pt}{18pt}\selectfont

In this thesis we explore the characteristics of strongly interacting matter,
described by \emph{Quantum Chromodynamics} (QCD). In particular, we investigate
the properties of QCD at extreme densities, a region yet to be explored by first
principle methods. We base the study on lattice gauge theory with Wilson
fermions in the strong coupling, heavy quark regime. We expand the lattice
action around this limit, and carry out analytic integrals over the gauge links
to obtain an effective, dimensionally reduced, theory of Polyakov loop
interactions.

The 3D effective theory suffers only a mild sign problem, and we briefly outline
how it can be simulated using either Monte Carlo techniques with reweighting, or
the Complex Langevin flow. We then continue to the main topic of the thesis,
namely the analytic treatment of the effective theory. We introduce the linked
cluster expansion, a method ideal for studying thermodynamic expansions.  The
complex nature of the effective theory action requires the development of a
generalisation of the linked cluster expansion.  We find a mapping between
generalised linked cluster expansion and our effective theory, and use this to
compute the thermodynamic quantities.

Lastly, various resummation techniques are explored, and a chain resummation is
implemented on the level of the effective theory itself. The resummed effective
theory describes not only nearest neighbour, next to nearest
neighbour, and so on, interactions, but couplings at all distances, making it
well suited for describing macroscopic effects. We compute the equation of state
for cold and dense heavy QCD, and find a correspondence with that of
non-relativistic free fermions, indicating a shift of the dynamics in the
continuum.

We conclude this thesis by presenting two possible extensions to new physics
using the techniques outlined within. First is the application of the effective
theory in the large-$N_c$ limit, of particular interest to the study of
conformal field theory. Second is the computation of analytic Yang Lee zeros,
which can be applied in the search for real phase transitions.

}
